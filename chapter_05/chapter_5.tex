\documentclass[a4paper]{book}

\usepackage{amsmath}
\usepackage{amssymb}
\usepackage[hypcap=false]{caption}
\usepackage{enumitem}	% 定制enumerate标号
\usepackage{geometry}
\geometry{
	left=2cm,
	right=2cm,
	top=2cm,
	bottom=2cm,
}
\usepackage{hyperref}
\hypersetup{
    colorlinks=true,            %链接颜色
    linkcolor=blue,             %内部链接
    filecolor=magenta,          %本地文档
    urlcolor=cyan,              %网址链接
}
\usepackage[none]{hyphenat}		% 阻止长单词分在两行
\usepackage{mathrsfs}
\usepackage[version=4]{mhchem}
\usepackage{subcaption}
\usepackage{titlesec}

\RequirePackage[many]{tcolorbox}
\tcbset{
    boxed title style={colback=magenta},
	breakable,
	enhanced,
	sharp corners,
	attach boxed title to top left={yshift=-\tcboxedtitleheight,  yshifttext=-.75\baselineskip},
	boxed title style={boxsep=1pt,sharp corners},
    fonttitle=\bfseries\sffamily,
}

\definecolor{skyblue}{rgb}{0.54, 0.81, 0.94}

\newtcolorbox[auto counter, number within=chapter, number format=\arabic]{exercise}[1][]{
    title={Exercise~\thetcbcounter},
    colframe=skyblue,
    colback=skyblue!12!white,
    boxed title style={colback=skyblue},
    overlay unbroken and first={
        \node[below right,font=\small,color=skyblue,text width=.8\linewidth]
        at (title.north east) {#1};
    }
}

\newtcolorbox[auto counter, number within=chapter, number format=\arabic]{solution}[1][]{
    title={Solution~\thetcbcounter},
    colframe=teal!60!green,
    colback=green!12!white,
    boxed title style={colback=teal!60!green},
    overlay unbroken and first={
        \node[below right,font=\small,color=red,text width=.8\linewidth]
        at (title.north east) {#1};
    }
}

% special new commands for common symbols used in the article
\newcommand\la{\langle}
\newcommand\ra{\rangle}
\newcommand\lr[2]{\langle#1\|#2\rangle}
\newcommand\tr[1]{\mathrm{tr(#1)}}
\newcommand\Tr[3]{#1\mathrm\#2#3}
\newcommand*{\dif}{\mathop{}\!\mathrm{d}}
\renewcommand\det[1]{\mathrm{det\left(#1\right)}}
\newcommand{\HF}{{\rm HF}}

\newcommand{\A}{{\bf A}}
\newcommand{\B}{{\bf B}}
\newcommand{\C}{{\bf C}}
\newcommand{\I}{{\bf 1}}
\newcommand{\U}{{\bf U}}
\newcommand{\Op}{{\bf O}}

\titleformat{\chapter}[display]
  {\bfseries\Large}
  {\filright\MakeUppercase{\chaptertitlename} \Huge\thechapter}
  {1ex}
  {\titlerule\vspace{1ex}\filleft}
  [\vspace{1ex}\titlerule]
  
\allowdisplaybreaks

\begin{document}

	\stepcounter{chapter}\stepcounter{chapter}\stepcounter{chapter}\stepcounter{chapter}

	\chapter{Pair and Coupled-Pair Theories}
	
	\section{The Independent Electron Pair Approximation (IEPA)}
	
	\begin{exercise}
	111
	\end{exercise}
	
	\begin{solution}
		5-1 so
	\end{solution}
	
	\begin{exercise}
	111
	\end{exercise}
	
	\begin{solution}
		5-2 so
	\end{solution}
	
	\begin{exercise}
	111
	\end{exercise}
	
	\begin{solution}
		5-3 so
	\end{solution}
	
	\subsection{Invariance under Unitary Transformations: An Example}
	
	\begin{exercise}
	111
	\end{exercise}
	
	\begin{solution}
		5-4 so
	\end{solution}
	
	\begin{exercise}
	111
	\end{exercise}
	
	\begin{solution}
		5-5 so
	\end{solution}
	
	\begin{exercise}
	111
	\end{exercise}
	
	\begin{solution}
		5-6 so
	\end{solution}
	
	\begin{exercise}
	111
	\end{exercise}
	
	\begin{solution}
		5-7 so
	\end{solution}
	
	\begin{exercise}
	111
	\end{exercise}
	
	\begin{solution}
		5-8 so
	\end{solution}
	
	\begin{exercise}
	111
	\end{exercise}
	
	\begin{solution}
		5-9 so
	\end{solution}
	
	\begin{exercise}
	111
	\end{exercise}
	
	\begin{solution}
		5-10 so
	\end{solution}
	
	\subsection{Some Illustrative Calculations}
	
	\section{Coupled-Pair Theories}
	
	\subsection{The Coupled Cluster Approximation (CCA)}
	
	\subsection{The Cluster Expansion of the Wave Function}
	
	\begin{exercise}
	111
	\end{exercise}
	
	\begin{solution}
		5-11 so
	\end{solution}
	
	\subsection{Linear CCA and the Coupled Electron Pair Approximation (CEPA)}
	
	\begin{exercise}
	111
	\end{exercise}
	
	\begin{solution}
		5-12 so
	\end{solution}
	
	\subsection{Some Illustrative Calculations}
	
	\section{Many-Electron Theories with Single Particle Hamiltonians}
	
	\begin{exercise}
	111
	\end{exercise}
	
	\begin{solution}
		5-13 so
	\end{solution}
	
	\subsection{The Relaxation Energy via CI, IEPA, CEPA, and CCA}
	
	\begin{exercise}
	111
	\end{exercise}
	
	\begin{solution}
		5-14 so
	\end{solution}
	
	\begin{exercise}
	111
	\end{exercise}
	
	\begin{solution}
		5-15 so
	\end{solution}
	
	\subsection{The Resonance Energy of Polyenes in H{\"u}ckel Theory}
	
	\begin{exercise}
	111
	\end{exercise}
	
	\begin{solution}
		5-16 so
	\end{solution}
	
	\begin{exercise}
	111
	\end{exercise}
	
	\begin{solution}
		5-17 so
	\end{solution}
	\begin{exercise}
	111
	\end{exercise}
	
	\begin{solution}
		5-18 so
	\end{solution}
	
	\begin{exercise}
	111
	\end{exercise}
	
	\begin{solution}
		5-19 so
	\end{solution}
	
	\begin{exercise}
	111
	\end{exercise}
	
	\begin{solution}
		5-20 so
	\end{solution}
	
	\begin{exercise}
	111
	\end{exercise}
	
	\begin{solution}
		5-21 so
	\end{solution}
	

\end{document}
