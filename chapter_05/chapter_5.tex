\documentclass[a4paper]{book}
%\usepackage{amsmath}\special{dvipdfmx:config z 0} %取消PDF压缩,加快速度,最终版本生成的时候最好把这句话注释掉

\usepackage{amsmath}
\usepackage{amssymb}
\usepackage[hypcap=false]{caption}
\usepackage{enumitem}	% 定制enumerate标号
\usepackage{geometry}
\geometry{
	left=2cm,
	right=2cm,
	top=2cm,
	bottom=2cm,
}
\usepackage{hyperref}
\hypersetup{
    colorlinks=true,            %链接颜色
    linkcolor=blue,             %内部链接
    filecolor=magenta,          %本地文档
    urlcolor=cyan,              %网址链接
}
\usepackage[none]{hyphenat}		% 阻止长单词分在两行
\usepackage{mathrsfs}
\usepackage[version=4]{mhchem}
\usepackage{subcaption}
\usepackage{titlesec}

\RequirePackage[many]{tcolorbox}
\tcbset{
    boxed title style={colback=magenta},
	breakable,
	enhanced,
	sharp corners,
	attach boxed title to top left={yshift=-\tcboxedtitleheight,  yshifttext=-.75\baselineskip},
	boxed title style={boxsep=1pt,sharp corners},
    fonttitle=\bfseries\sffamily,
}

\definecolor{skyblue}{rgb}{0.54, 0.81, 0.94}

\newcounter{exercise}[chapter]
\newcounter{solution}[chapter]
\newcounter{eqs}[solution]

\newenvironment{sequation}
  {\begin{equation}\stepcounter{eqs}\tag{\thesolution-\theeqs}}
  {\end{equation}}

\newtcolorbox[use counter=exercise, number within=chapter, number format=\arabic]{exercise}[1][]{
    title={Exercise~\thetcbcounter},
    colframe=skyblue,
    colback=skyblue!12!white,
    boxed title style={colback=skyblue},
    overlay unbroken and first={
        \node[below right,font=\small,color=skyblue,text width=.8\linewidth]
        at (title.north east) {#1};
    }
}

\newtcolorbox[use counter=solution, number within=chapter, number format=\arabic]{solution}[1][]{
    title={Solution~\thetcbcounter},
    colframe=teal!60!green,
    colback=green!12!white,
    boxed title style={colback=teal!60!green},
    overlay unbroken and first={
        \node[below right,font=\small,color=red,text width=.8\linewidth]
        at (title.north east) {#1};
    }
}

% special new commands for common symbols used in the article
\newcommand\la{\langle}
\newcommand\ra{\rangle}
\newcommand\lr[2]{\langle#1\|#2\rangle}
\newcommand\tr[1]{\mathrm{tr(#1)}}
\newcommand\Tr[3]{#1\mathrm\#2#3}
\newcommand*{\dif}{\mathop{}\!\mathrm{d}}
\renewcommand\det[1]{{\rm det}\left(#1\right)}
\newcommand{\HF}{{\rm HF}}
\newcommand{\corr}{{\rm corr}}
\newcommand{\DCI}{{\rm DCI}}
\newcommand{\heff}{h_{\rm eff}}

\newcommand{\A}{{\bf A}}
\newcommand{\B}{{\bf B}}
\newcommand{\C}{{\bf C}}
\newcommand{\HH}{{\bf H}}
\newcommand{\I}{{\bf 1}}
\newcommand{\U}{{\bf U}}
\newcommand{\Op}{{\bf O}}

\titleformat{\chapter}[display]
  {\bfseries\Large}
  {\filright\MakeUppercase{\chaptertitlename} \Huge\thechapter}
  {1ex}
  {\titlerule\vspace{1ex}\filleft}
  [\vspace{1ex}\titlerule]
  
\allowdisplaybreaks

\begin{document}

	\stepcounter{chapter}\stepcounter{chapter}\stepcounter{chapter}\stepcounter{chapter}

	\chapter{Pair and Coupled-Pair Theories}
	
	\section{The Independent Electron Pair Approximation (IEPA)}
	
	% 5.1
	\begin{exercise}
	The application of pair theory  to minimal basis $\ce{H2}$ is trivial since we are dealing with a two-electron system for which the IEPA is exact, i.e., it gives the full CI result obtained in the last chapter, viz.
	\[
		{}^{1}E_{\corr} = \Delta - ( \Delta^2 + K^2_{12} )^{1/2}
	\]
	where (see Eq.(4.20))
	\[
		\Delta = (\varepsilon_2 - \varepsilon_1) + \frac{1}{2}( J_{11} + J_{22} - 4J_{12} + 2K_{12} ).
	\]
	\begin{enumerate}
	
	\item[a.] Calculate the correlation energy using first-order pairs. Remember that the summations in Eq.(5.19) go over spin orbitals (i.e., $a=1$, $\bar{1}$, and $r=2$, $\bar{2}$). Show that
	\[
		{}^{1} E_{\corr}({\rm FO}) = \frac{K^2_{12}}{ 2 ( \varepsilon_1 - \varepsilon_2) }.
	\]
	
	\item[b.] Approximate $\Delta$ in the exact correlation energy by $\varepsilon_2 - \varepsilon_1$ and recover the first-order pair correlation energy by expanding the exact answer to first order using the relation $(1+x)^{1/2} \approx 1 + x/2$.
	\end{enumerate}
	\end{exercise}
	
	\begin{solution}
	
	\begin{enumerate}
	
	\item[a.] At this time, there is only one pair of electrons and one pair of virtual spin orbitals. Thus,
	\begin{sequation}
		E_\corr ({\rm FO}) = \sum_{ \substack{ a<b \\ r<s } } \frac{ | \langle \Psi_0 | \mathscr{H} | \Psi^{rs}_{ab} \rangle |^2 }{ \varepsilon_a + \varepsilon_b - \varepsilon_r - \varepsilon_s } = \frac{ | \langle 1 \bar{1} || 2 \bar{2} \rangle |^2 }{ \varepsilon_1 + \varepsilon_{\bar{1}} - \varepsilon_2 - \varepsilon_{\bar{2}} } = - \frac{ K^2_{12} }{ 2( \varepsilon_2 - \varepsilon_1 ) }.
	\end{sequation}
		
	\item[b.] As $K_{12} \ll \Delta$, we find that
	\[
		{}^1 E_\corr = \Delta \left[ 1 - \sqrt{ 1 + \frac{ K^2_{12} }{ \Delta^2 } } \right] = \Delta \left[ 1 - \left( 1 + \frac{ K^2_{12} }{ 2\Delta^2 } + \cdots \right) \right] = - \frac{ K^2_{12} }{ 2\Delta } + \cdots.
	\]
	Here, the truth that when $|x| \ll 1$,
	\[
		(1+x)^{\frac{1}{2}} \approx 1 + \frac{x}{2},
	\]
	has been used.
	
	After substitute $\Delta = \varepsilon_2 - \varepsilon_1$, we obtain	
	\begin{equation}
		{}^1 E_\corr ({\rm FO}) = - \frac{ K^2_{12} }{ 2( \varepsilon_2 - \varepsilon_1 ) } = \frac{ K^2_{12} }{ 2( \varepsilon_1 - \varepsilon_2 ) }.
	\end{equation}
	
	
	\end{enumerate}		
	
	\end{solution}

	% 5.2
	\begin{exercise}
	Derive Eqs.(5.22a) and (5.22b).
	\end{exercise}
	
	\begin{solution}
	
	From (5.9a)
	
	\begin{sequation}
		K_{12} c^{2_i \bar{2}_i}_{1_i \bar{1}_i} = e_{1_i \bar{1}_i }
	\end{sequation}
	
	\begin{sequation}
		K_{12} + \langle \Psi^{2_i \bar{2}_i}_{1_i \bar{1}_i} | \mathscr{H} - E_0 | \Psi^{2_i \bar{2}_i}_{1_i \bar{1}_i} \rangle = e_{ 1_i \bar{1}_i } c^{2_i \bar{2}_i}_{1_i \bar{1}_i}
	\end{sequation}

	\[
		h_{11} = \varepsilon_1 - J_{11} , \quad h_{22} = \varepsilon_2 - 2 J_{12} + K_{12}
	\]
	
	\[
		\langle \Psi^{2_i \bar{2}_i}_{1_i \bar{1}_i} | \mathscr{H} - E_0 | \Psi^{2_i \bar{2}_i}_{1_i \bar{1}_i} \rangle = 2 \Delta
	\]
	
	\end{solution}

\end{document}
