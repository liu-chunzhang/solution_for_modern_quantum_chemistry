\documentclass[a4paper]{book}
%\usepackage{amsmath}\special{dvipdfmx:config z 0} %取消PDF压缩,加快速度,最终版本生成的时候最好把这句话注释掉

\usepackage{amsmath}
\usepackage{amssymb}
\usepackage[hypcap=false]{caption}
\usepackage{enumitem}	% 定制enumerate标号
\usepackage{geometry}
\geometry{
	left=2cm,
	right=2cm,
	top=2cm,
	bottom=2cm,
}
\usepackage{hyperref}
\hypersetup{
    colorlinks=true,            %链接颜色
    linkcolor=blue,             %内部链接
    filecolor=magenta,          %本地文档
    urlcolor=cyan,              %网址链接
}
\usepackage[none]{hyphenat}		% 阻止长单词分在两行
\usepackage{mathrsfs}
\usepackage[version=4]{mhchem}
\usepackage{subcaption}
\usepackage{titlesec}

\RequirePackage[many]{tcolorbox}
\tcbset{
    boxed title style={colback=magenta},
	breakable,
	enhanced,
	sharp corners,
	attach boxed title to top left={yshift=-\tcboxedtitleheight,  yshifttext=-.75\baselineskip},
	boxed title style={boxsep=1pt,sharp corners},
    fonttitle=\bfseries\sffamily,
}

\definecolor{skyblue}{rgb}{0.54, 0.81, 0.94}

\newcounter{exercise}[chapter]
\newcounter{solution}[chapter]
\newcounter{eqs}[solution]

\newenvironment{sequation}
  {\begin{equation}\stepcounter{eqs}\tag{\thesolution-\theeqs}}
  {\end{equation}}

\newtcolorbox[use counter=exercise, number within=chapter, number format=\arabic]{exercise}[1][]{
    title={Exercise~\thetcbcounter},
    colframe=skyblue,
    colback=skyblue!12!white,
    boxed title style={colback=skyblue},
    overlay unbroken and first={
        \node[below right,font=\small,color=skyblue,text width=.8\linewidth]
        at (title.north east) {#1};
    }
}

\newtcolorbox[use counter=solution, number within=chapter, number format=\arabic]{solution}[1][]{
    title={Solution~\thetcbcounter},
    colframe=teal!60!green,
    colback=green!12!white,
    boxed title style={colback=teal!60!green},
    overlay unbroken and first={
        \node[below right,font=\small,color=red,text width=.8\linewidth]
        at (title.north east) {#1};
    }
}

% special new commands for common symbols used in the article
\newcommand\la{\langle}
\newcommand\ra{\rangle}
\newcommand\lr[2]{\langle#1\|#2\rangle}
\newcommand\tr[1]{\mathrm{tr(#1)}}
\newcommand\Tr[3]{#1\mathrm\#2#3}
\newcommand*{\dif}{\mathop{}\!\mathrm{d}}
\renewcommand\det[1]{{\rm det}\left(#1\right)}
\newcommand{\HF}{{\rm HF}}
\newcommand{\corr}{{\rm corr}}
\newcommand{\DCI}{{\rm DCI}}
\newcommand{\FO}{{\rm FO}}
\newcommand{\heff}{h_{\rm eff}}

\newcommand{\A}{{\bf A}}
\newcommand{\B}{{\bf B}}
\newcommand{\C}{{\bf C}}
\newcommand{\HH}{{\bf H}}
\newcommand{\I}{{\bf 1}}
\newcommand{\U}{{\bf U}}
\newcommand{\Op}{{\bf O}}

\titleformat{\chapter}[display]
  {\bfseries\Large}
  {\filright\MakeUppercase{\chaptertitlename} \Huge\thechapter}
  {1ex}
  {\titlerule\vspace{1ex}\filleft}
  [\vspace{1ex}\titlerule]
  
\allowdisplaybreaks

\begin{document}

	\stepcounter{chapter}\stepcounter{chapter}\stepcounter{chapter}\stepcounter{chapter}

	\chapter{Pair and Coupled-Pair Theories}
	
	\section{The Independent Electron Pair Approximation (IEPA)}
	
	% 5.1
	\begin{exercise}
	The application of pair theory  to minimal basis $\ce{H2}$ is trivial since we are dealing with a two-electron system for which the IEPA is exact, i.e., it gives the full CI result obtained in the last chapter, viz.
	\[
		{}^{1}E_{\corr} = \Delta - ( \Delta^2 + K^2_{12} )^{1/2}
	\]
	where (see Eq.(4.20))
	\[
		\Delta = (\varepsilon_2 - \varepsilon_1) + \frac{1}{2}( J_{11} + J_{22} - 4J_{12} + 2K_{12} ).
	\]
	\begin{enumerate}
	
	\item[a.] Calculate the correlation energy using first-order pairs. Remember that the summations in Eq.(5.19) go over spin orbitals (i.e., $a=1$, $\bar{1}$, and $r=2$, $\bar{2}$). Show that
	\[
		{}^{1} E_{\corr}({\rm FO}) = \frac{K^2_{12}}{ 2 ( \varepsilon_1 - \varepsilon_2) }.
	\]
	
	\item[b.] Approximate $\Delta$ in the exact correlation energy by $\varepsilon_2 - \varepsilon_1$ and recover the first-order pair correlation energy by expanding the exact answer to first order using the relation $(1+x)^{1/2} \approx 1 + x/2$.
	\end{enumerate}
	\end{exercise}
	
	\begin{solution}
	
	\begin{enumerate}
	
	\item[a.] At this time, there is only one pair of electrons and one pair of virtual spin orbitals. Thus,
	\begin{sequation}
		E_\corr ({\rm FO}) = \sum_{ \substack{ a<b \\ r<s } } \frac{ | \langle \Psi_0 | \mathscr{H} | \Psi^{rs}_{ab} \rangle |^2 }{ \varepsilon_a + \varepsilon_b - \varepsilon_r - \varepsilon_s } = \frac{ | \langle 1 \bar{1} || 2 \bar{2} \rangle |^2 }{ \varepsilon_1 + \varepsilon_{\bar{1}} - \varepsilon_2 - \varepsilon_{\bar{2}} } = - \frac{ K^2_{12} }{ 2( \varepsilon_2 - \varepsilon_1 ) }.
	\end{sequation}
		
	\item[b.] As $K_{12} \ll \Delta$, we find that
	\[
		{}^1 E_\corr = \Delta \left[ 1 - \sqrt{ 1 + \frac{ K^2_{12} }{ \Delta^2 } } \right] = \Delta \left[ 1 - \left( 1 + \frac{ K^2_{12} }{ 2\Delta^2 } + \cdots \right) \right] = - \frac{ K^2_{12} }{ 2\Delta } + \cdots.
	\]
	Here, the truth that when $|x| \ll 1$,
	\[
		(1+x)^{\frac{1}{2}} \approx 1 + \frac{x}{2},
	\]
	has been used.
	
	After substitute $\Delta = \varepsilon_2 - \varepsilon_1$, we obtain	
	\begin{equation}
		{}^1 E_\corr ({\rm FO}) = - \frac{ K^2_{12} }{ 2( \varepsilon_2 - \varepsilon_1 ) } = \frac{ K^2_{12} }{ 2( \varepsilon_1 - \varepsilon_2 ) }.
	\end{equation}
	
	
	\end{enumerate}		
	
	\end{solution}

	% 5.2
	\begin{exercise}
	Derive Eqs.(5.22a) and (5.22b).
	\end{exercise}
	
	\begin{solution}
	
	From (5.9a), with $\langle \Psi^{2_i \bar{2}_i }_{1_i \bar{1}_i} | \mathscr{H} | \Psi_0 \rangle = \langle \Psi_0 | \mathscr{H} | \Psi^{2_i \bar{2}_i }_{1_i \bar{1}_i} \rangle = \langle 1_i \bar{1}_i || 2_i \bar{2}_i \rangle = K_{12}$, we obtain
	\begin{sequation}
		K_{12} c^{2_i \bar{2}_i}_{1_i \bar{1}_i} = e_{1_i \bar{1}_i },
	\end{sequation}
	which is (5.22a). Similarly, with (4.20), we obtain
	\begin{sequation}
		K_{12} + \langle \Psi^{2_i \bar{2}_i}_{1_i \bar{1}_i} | \mathscr{H} - E_0 | \Psi^{2_i \bar{2}_i}_{1_i \bar{1}_i} \rangle = e_{ 1_i \bar{1}_i } c^{2_i \bar{2}_i}_{1_i \bar{1}_i}
	\end{sequation}
	which is (5.22b).
	
	\end{solution}
	
	% 5.3	
	\begin{exercise}
	Calculate the total first-order pair correlation energy for the dinner using Eq.(5.19) and show that it is twice the result obtained in Exercise 5.1.
	\end{exercise}
	
	\begin{solution}
	Note that only $| 2_1 \bar{2}_1 1_2 \bar{1}_2 \rangle$ and $| 1_1 \bar{1}_1 2_2 \bar{2}_2 \rangle$ can interact with $\langle \Phi_0 | $ via the Hamiltonian $\mathscr{H}$, thus
	\begin{sequation}
		E_\corr( {\rm FO}, 2\ce{H2} ) = \frac{ \langle \Psi_0 | \mathscr{H} | 2_1 \bar{2}_1 1_2 \bar{1}_2 \rangle }{ \varepsilon_1 + \varepsilon_{\bar{1}} - \varepsilon_2 - \varepsilon_{\bar{2}} } + \frac{ \langle \Psi_0 | \mathscr{H} | 1_1 \bar{1}_1 2_2 \bar{2}_2 \rangle }{ \varepsilon_1 + \varepsilon_{\bar{1}} - \varepsilon_2 - \varepsilon_{\bar{2}} } = - \frac{ K^2_{12} }{ 2(\varepsilon_1 - \varepsilon_2) } - \frac{ K^2_{12} }{ 2(\varepsilon_1 - \varepsilon_2) } = 2 E_\corr( {\rm FO},\ce{H2} ).
	\end{sequation}		
	It is twice the result obtained in Exercise 5.1.
	
	\end{solution}
	
	\subsection{Invariance under Unitary Transformations: An Example}
	
	% 5.4
	\begin{exercise}
	Show that $| a \bar{a} b \bar{b} \rangle = | 1_1 \bar{1}_1 \bar{1}_2 \bar{1}_2 \rangle$. {\it Hint}: use Eq.(1.40) repeatedly. Eq.(1.40) for Slater determinants is
	\[
		| \chi_1 \chi_2 \cdots \left( \sum_{k} c_k \chi^\prime_k \right) \cdots \chi_N \rangle = \sum_k c_k | \chi_1 \chi_2 \cdots \chi^\prime_k \cdots \chi_N \rangle.
	\]
	\end{exercise}
	
	\begin{solution}
	Note that if any two rows (or columns) of a determinant are equal, the value of the determinant is zero. Therefore, we find that	
	\begin{align*}
		| a \bar{a} b \bar{b} \rangle &= \frac{1}{ \sqrt{2} } \left( | 1_1 \bar{a} b \bar{b} \rangle + | 1_2 \bar{a} b \bar{b} \rangle \right) \\
		&= \frac{1}{ \sqrt{2} } \left( \frac{1}{ \sqrt{2} } | 1_1 \bar{a} 1_1 \bar{b} \rangle - \frac{1}{ \sqrt{2} } | 1_1 \bar{a} 1_2 \bar{b} \rangle \right) + \frac{1}{ \sqrt{2} } \left( \frac{1}{ \sqrt{2} } | 1_2 \bar{a} 1_1 \bar{b} \rangle - \frac{1}{ \sqrt{2} } | 1_2 \bar{a} 1_2 \bar{b} \rangle  \right) \\
		&= - \frac{1}{2} | 1_1 \bar{a} 1_2 \bar{b} \rangle + \frac{1}{2} | 1_2 \bar{a} 1_1 \bar{b} \rangle = - | 1_1 \bar{a} 1_2 \bar{b} \rangle \\
		&= - \frac{1}{ \sqrt{2} } \left( | 1_1 \bar{1}_1 1_2 \bar{b} \rangle + | 1_1 \bar{1}_2 1_2 \bar{b} \rangle \right) \\
		&= - \frac{1}{ \sqrt{2} } \left( \frac{1}{ \sqrt{2} } | 1_1 \bar{1}_1 1_2 \bar{1}_1 \rangle - \frac{1}{ \sqrt{2} } | 1_1 \bar{1}_1 1_2 \bar{1}_2 \rangle \right) - \frac{1}{ \sqrt{2} } \left( \frac{1}{ \sqrt{2} } | 1_1 \bar{1}_2 1_2 \bar{1}_1 \rangle - \frac{1}{ \sqrt{2} } | 1_1 \bar{1}_2 1_2 \bar{1}_2 \rangle \right) \\
		&= \frac{1}{2} | 1_1 \bar{1}_1 1_2 \bar{1}_2 \rangle - \frac{1}{2} | 1_1 \bar{1}_2 1_2 \bar{1}_1 \rangle = | 1_1 \bar{1}_1 1_2 \bar{1}_2 \rangle.
	\end{align*}
	Concisely, 
	\begin{sequation}
		| a \bar{a} b \bar{b} \rangle = | 1_1 \bar{1}_1 1_2 \bar{1}_2 \rangle.
	\end{sequation}
			
	\end{solution}
	
	% 5.5
	\begin{exercise}
	Derive Eqs.(5.31a) and (5.31b).
	\end{exercise}
	
	\begin{solution}
	With (5.28a), (5.28b), and (5.29), we find that
	\begin{align*}
		\langle \Psi_0 | \mathscr{H} | \Psi^{**}_{a \bar{a}} \rangle &= \langle \Psi_0 | \mathscr{H} | \left( \frac{1}{ \sqrt{2} } | \Psi^{ r \bar{r} }_{ a \bar{a} } \rangle + \frac{1}{ \sqrt{2} } | \Psi^{ s \bar{s} }_{ a \bar{a} } \rangle \right) = \frac{1}{ \sqrt{2} } \langle \Psi_0 | \mathscr{H}  | \Psi^{ r \bar{r} }_{ a \bar{a} } \rangle + \frac{1}{ \sqrt{2} } \langle \Psi_0 | \mathscr{H} | \Psi^{ s \bar{s} }_{ a \bar{a} } \rangle \\
		&= \frac{1}{ \sqrt{2} } \frac{ K_{12} }{2} + \frac{1}{ \sqrt{2} } \frac{ K_{12} }{2} = \frac{ K_{12} }{ \sqrt{2} }.
	\end{align*}
	This is (5.31a). From (4.17a) and (4.17d), we know that
	\[
		E_0 = \langle \Psi_0 | \mathscr{H} | \Psi_0 \rangle = \langle \Psi_0 | \mathscr{O}_1 | \Psi_0 \rangle + \langle \Psi_0 | \mathscr{O}_2 | \Psi_0 \rangle = 2 ( 2 h_{11} + J_{11} ) = 4\varepsilon_1 - 2 J_{11}.
	\]
	And	it is evident that
	\begin{align*}
		\langle \Psi^{**}_{a \bar{a}} | \mathscr{H} | \Psi^{**}_{a \bar{a}} \rangle &= \left( \frac{1}{ \sqrt{2} } \langle  \Psi^{ r \bar{r} }_{ a \bar{a} } | + \frac{1}{ \sqrt{2} } \langle \Psi^{s \bar{s}}_{a \bar{a}} | \right) \mathscr{H} \left( \frac{1}{ \sqrt{2} } | \Psi^{ r \bar{r} }_{ a \bar{a} } \rangle + \frac{1}{ \sqrt{2} } | \Psi^{s \bar{s}}_{a \bar{a}} \rangle \right) \\
		&= \frac{1}{2} \langle \Psi^{r \bar{r}}_{a \bar{a}} | \mathscr{H} | \Psi^{r \bar{r}}_{a \bar{a}} \rangle + \frac{1}{2} \langle \Psi^{r \bar{r}}_{a \bar{a}} | \mathscr{H} | \Psi^{s \bar{s}}_{a \bar{a}} \rangle + \frac{1}{2} \langle \Psi^{s \bar{s}}_{a \bar{a}} | \mathscr{H} | \Psi^{r \bar{r}}_{a \bar{a}} \rangle + \frac{1}{2} \langle \Psi^{s \bar{s}}_{a \bar{a}} | \mathscr{H} | \Psi^{s \bar{s}}_{a \bar{a}} \rangle .
	\end{align*}
	We have to calculate $\langle \Psi^{r \bar{r}}_{a \bar{a}} | \mathscr{H} | \Psi^{r \bar{r}}_{a \bar{a}} \rangle$, $\langle \Psi^{r \bar{r}}_{a \bar{a}} | \mathscr{H} | \Psi^{s \bar{s}}_{a \bar{a}} \rangle$ and $\langle \Psi^{s \bar{s}}_{a \bar{a}} | \mathscr{H} | \Psi^{s \bar{s}}_{a \bar{a}} \rangle$. 
	
	Before their formal derivation, note that
	\[
		h_{rr} = h_{ss} = h_{22} , \quad h_{aa} = h_{bb} = h_{11}.
	\]
	
	With (5.26a)-(5.26d), we obtain
	\begin{align*}
		\langle \Psi^{r \bar{r}}_{a \bar{a}} | \mathscr{H} | \Psi^{r \bar{r}}_{a \bar{a}} \rangle &= \langle \Psi^{r \bar{r}}_{a \bar{a}} | \mathscr{O}_1 | \Psi^{r \bar{r}}_{a \bar{a}} \rangle + \langle \Psi^{r \bar{r}}_{a \bar{a}} | \mathscr{O}_2 | \Psi^{r \bar{r}}_{a \bar{a}} \rangle	\\
		&= \langle r | h | r \rangle + \langle \bar{r} | h | \bar{r} \rangle + \langle b | h | b \rangle + \langle \bar{b} | h | \bar{b} \rangle \\
		&\hspace{4em} + \langle r \bar{r} || r \bar{r} \rangle + \langle rb || rb \rangle + \langle r \bar{b} || r \bar{b} \rangle + \langle \bar{r} b || \bar{r} b \rangle + \langle \bar{r} \bar{b} || \bar{r} \bar{b} \rangle + \langle b \bar{b} || b \bar{b} \rangle \\
		&= 2h_{rr} + 2h_{bb} + J_{rr} + \left( J_{rb} - K_{rb} \right) + J_{rb} + J_{rb} + \left( J_{rb} - K_{rb} \right) + J_{bb} \\
		&= 2h_{11} + 2h_{22} + \frac{1}{2} J_{11} + \frac{1}{2} J_{22} + 2 J_{12} - K_{12} \\
		&= 2\varepsilon_1 + 2\varepsilon_2 - \frac{3}{2} J_{11} + \frac{1}{2} J_{22} - 2J_{12} + K_{12}.
	\end{align*}		
	Similarly, we obtain that
	\[
		\langle \Psi^{r \bar{r}}_{a \bar{a}} | \mathscr{H} | \Psi^{s \bar{s}}_{a \bar{a}} \rangle = \langle r \bar{r} || s \bar{s} \rangle = J_{rs} = \frac{1}{2} J_{22},
	\]
	and
	\begin{align*}
		\langle \Psi^{s \bar{s}}_{a \bar{a}} | \mathscr{H} | \Psi^{s \bar{s}}_{a \bar{a}} \rangle &= 2h_{ss} + 2h_{aa} + J_{ss} + J_{sb} - K_{sb} + J_{sb} + J_{sb} + J_{sb} - K_{sb} + J_{bb} \\
		&= 2h_{11} + 2h_{22} + \frac{1}{2} J_{11} + \frac{1}{2} J_{22} + 2J_{12} - K_{12}, \\
		&= 2\varepsilon_1 + 2\varepsilon_2 - \frac{3}{2} J_{11} + \frac{1}{2} J_{22} - 2J_{12} + K_{12}.
	\end{align*}
	Hence,
	\begin{align*}
		\langle \Psi^{**}_{a \bar{a}} | \mathscr{H} | \Psi^{**}_{a \bar{a}} \rangle &= 2\varepsilon_1 + 2\varepsilon_2 - \frac{3}{2} J_{11} + \frac{1}{2} J_{22} - 2J_{12} + K_{12} + \frac{1}{2} J_{22} \\
		&= 2\varepsilon_1 + 2\varepsilon_2 - \frac{3}{2} J_{11} + J_{22} - 2J_{12} + K_{12}, \\
		\langle \Psi^{**}_{a \bar{a}} | \mathscr{H} - E_0 | \Psi^{**}_{a \bar{a}} \rangle &= 2\varepsilon_1 + 2\varepsilon_2 - \frac{3}{2} J_{11} + J_{22} - 2J_{12} + K_{12} - (4\varepsilon_1 - 2J_{11} ) \\
		&= 2( \varepsilon_2 - \varepsilon_1 ) + \frac{1}{2} J_{11} + J_{22} - 2J_{12} + K_{12}.
	\end{align*}
	In conclusion, we obtain
	\begin{sequation}
		\langle \Psi^{**}_{a \bar{a}} | \mathscr{H} | \Psi^{**}_{a \bar{a}} \rangle = 2( \varepsilon_2 - \varepsilon_1 ) + J_{22} + \frac{1}{2} \left( J_{11} - 4 J_{12} + 2 K_{12} \right) \equiv 2 \Delta^\prime.
	\end{sequation}
	This is (5.31b).
		
	\end{solution}
	
	% 5.6
	\begin{exercise}
	Show that $e_{a\bar{b}} = e_{\bar{a}b} = e_{a\bar{a}}$.
	\end{exercise}
	
	\begin{solution}
	The key point is to prove that the equations which determine $e_{a \bar{b}}$ are identical to that of $e_{a \bar{a}}$. Note that similar to Exercise 5.5, we can obtain
	\begin{align*}
		\langle \Psi_0 | \mathscr{H} | \Psi^{**}_{a \bar{b}} \rangle &= \frac{ K_{12} }{ \sqrt{2} }, \\
		\langle \Psi^{**}_{a \bar{b}} | \mathscr{H} | \Psi^{**}_{a \bar{b}} \rangle &= 2\Delta^\prime.
	\end{align*}
	Thus the equations of $e_{a\bar{b}}$ are
	\begin{align*}
		\frac{ K_{12} }{ \sqrt{2} } c &= e_{a \bar{b}} , \\
		\frac{ K_{12} }{ \sqrt{2} } + 2\Delta^\prime c &= e_{a \bar{b}} c.
	\end{align*}
	They are identical to that of $e_{a \bar{a}}$, thus $e_{a\bar{b}} = e_{a\bar{a}}$ and so does $e_{a\bar{b}}$.
	\end{solution}
	
	% 5.7	
	\begin{exercise}
	Show that DCI is invariant to unitary transformations for the above model.
	\begin{enumerate}
	
	\item[a.] The DCI wave function is
	\[
		| \Psi_{\DCI} \rangle = | \Psi_0 \rangle + c_1 | \Psi^{**}_{a\bar{a}} \rangle + c_2 | \Psi^{**}_{b \bar{b}} \rangle + c_3 | \Psi^{**}_{a\bar{b}} \rangle + c_4 | \Psi^{**}_{\bar{a}b}\rangle.
	\]
	Show that the corresponding eigenvalue problem which determines the DCI correlation energy of the dimer (${}^{2}E_{\corr}(\DCI)$) is
	\[
	\begin{pmatrix}
		0 & \frac{1}{\sqrt{2}} K_{12} & \frac{1}{\sqrt{2}}K_{12} & \frac{1}{\sqrt{2}} K_{12} & \frac{1}{\sqrt{2}}K_{12} \\
		\frac{1}{\sqrt{2}}K_{12} & 2\Delta^\prime & \frac{1}{2}J_{11} & \frac{1}{2}K_{12} - J_{12} & \frac{1}{2}K_{12} - J_{12} \\
		\frac{1}{\sqrt{2}}K_{12} & \frac{1}{2}J_{11} & 2\Delta^\prime & \frac{1}{2}K_{12} - J_{12} & \frac{1}{2}K_{12} - J_{12} \\
		\frac{1}{\sqrt{2}}K_{12} & \frac{1}{2}K_{12} - J_{12} & \frac{1}{2}K_{12} - J_{12} & 2\Delta^\prime & \frac{1}{2}J_{11} \\
		\frac{1}{\sqrt{2}}K_{12} & \frac{1}{2}K_{12} - J_{12} & \frac{1}{2}K_{12} - J_{12} & \frac{1}{2}J_{11} & 2\Delta^\prime
	\end{pmatrix} \begin{pmatrix}
	1 \\ c_1 \\ c_2 \\ c_3 \\ c_4
	\end{pmatrix} = {}^2 E_\corr (\DCI) \begin{pmatrix}
	1 \\ c_1 \\ c_2 \\ c_3 \\ c_4
	\end{pmatrix}.
	\]
	
	\item[b.] Show that $c_1 = c_2 = c_3 = c_4 = c$ and then solve the equations to show
	\[
		{}^{2}E_\corr(\DCI) = \Delta - ( \Delta^2 + 2 K^2_{12} )^{1/2},
	\]
	which is the same result as found in the last chapter (see Eq.(4.60)).
	\end{enumerate}
	\end{exercise}
	
	\begin{solution}
	
	\begin{itemize}
	
	\item[a.] Some matrix elements have been solved in Exercise 5.5 and Exercise 5.6, and they are listed as follows. 
	\begin{align*}
		\langle \Psi_0 | \mathscr{H} - E_0 | \Psi_0 \rangle &= 0 , \\
		\langle \Psi_0 | \mathscr{H} | \Psi^{**}_{a\bar{a}} \rangle = \langle \Psi^{**}_{a\bar{a}} | \mathscr{H} | \Psi_0 \rangle &= \frac{1}{ \sqrt{2} } K_{12} , \\
		\langle \Psi_0 | \mathscr{H} | \Psi^{**}_{b \bar{b}} \rangle = \langle \Psi^{**}_{b \bar{b}} | \mathscr{H} | \Psi_0 \rangle &= \frac{1}{ \sqrt{2} } K_{12} , \\
		\langle \Psi_0 | \mathscr{H} | \Psi^{**}_{a\bar{b}} \rangle = \langle \Psi^{**}_{a\bar{b}} | \mathscr{H} | \Psi_0 \rangle &= \frac{1}{ \sqrt{2} } K_{12} , \\
		\langle \Psi_0 | \mathscr{H} | \Psi^{**}_{\bar{a}b}\rangle = \langle \Psi^{**}_{\bar{a}b} | \mathscr{H} | \Psi_0 \rangle &= \frac{1}{ \sqrt{2} } K_{12} , \\
		\langle \Psi^{**}_{a\bar{a}} | \mathscr{H} - E_0 | \Psi^{**}_{a\bar{a}} \rangle &= 2 \Delta^\prime , \\
		\langle \Psi^{**}_{b\bar{b}} | \mathscr{H} - E_0 | \Psi^{**}_{b\bar{b}} \rangle &= 2 \Delta^\prime , \\
		\langle \Psi^{**}_{a\bar{b}} | \mathscr{H} - E_0 | \Psi^{**}_{a\bar{b}} \rangle &= 2 \Delta^\prime , \\
		\langle \Psi^{**}_{\bar{a}b} | \mathscr{H} - E_0 | \Psi^{**}_{\bar{a}b} \rangle &= 2 \Delta^\prime .
	\end{align*}
	Now we pay attention to calculate other matrix elements.
	\begin{align*}
		\langle \Psi^{**}_{a\bar{a}} | \mathscr{H} | \Psi^{**}_{b\bar{b}} \rangle &= \left( \frac{1}{ \sqrt{2} } \langle \Psi^{ r \bar{r} }_{a \bar{a}} | + \frac{1}{ \sqrt{2} } \langle \Psi^{ s \bar{s} }_{a \bar{a}} | \right) \mathscr{H} \left( \frac{1}{ \sqrt{2} } | \Psi^{ r \bar{r} }_{b \bar{b}} \rangle + \frac{1}{ \sqrt{2} } | \Psi^{ s \bar{s} }_{b \bar{b}} \rangle \right) \\
		&= \frac{1}{2} \langle \Psi^{ r \bar{r} }_{a \bar{a}} | \mathscr{H} | \Psi^{ r \bar{r} }_{b \bar{b}} \rangle + \frac{1}{2} \langle \Psi^{ r \bar{r} }_{a \bar{a}} | \mathscr{H} | \Psi^{ s \bar{s} }_{b \bar{b}} \rangle + \frac{1}{2} \langle \Psi^{ s \bar{s} }_{a \bar{a}} | \mathscr{H} | \Psi^{ r \bar{r} }_{b \bar{b}} \rangle +  \frac{1}{2} \langle \Psi^{ s \bar{s} }_{a \bar{a}} | \mathscr{H} | \Psi^{ s \bar{s} }_{b \bar{b}} \rangle \\
		&= \frac{1}{2} \left( \langle b \bar{b} || a \bar{a} \rangle + 0 + 0 + \langle b \bar{b} || a \bar{a} \rangle \right) = \frac{1}{2} \left( K_{ab} + K_{ab} \right) = K_{ab} = \frac{1}{2} J_{11}, \\
		\langle \Psi^{**}_{a\bar{a}} | \mathscr{H} | \Psi^{**}_{a\bar{b}} \rangle &= \left( \frac{1}{ \sqrt{2} } \langle \Psi^{ r \bar{r} }_{a \bar{a}} | + \frac{1}{ \sqrt{2} } \langle \Psi^{ s \bar{s} }_{a \bar{a}} | \right) \mathscr{H} \left( \frac{1}{ \sqrt{2} } | \Psi^{ r \bar{s} }_{a \bar{b}} \rangle + \frac{1}{ \sqrt{2} } | \Psi^{ s \bar{r} }_{a \bar{b}} \rangle \right) \\
		&= \frac{1}{2} \langle \Psi^{ r \bar{r} }_{a \bar{a}} | \mathscr{H} | \Psi^{ r \bar{s} }_{a \bar{b}} \rangle + \frac{1}{2} \langle \Psi^{ r \bar{r} }_{a \bar{a}} | \mathscr{H} | \Psi^{ s \bar{r} }_{a \bar{b}} \rangle + \frac{1}{2} \langle \Psi^{ s \bar{s} }_{a \bar{a}} | \mathscr{H} | \Psi^{ r \bar{s} }_{a \bar{b}} \rangle + \frac{1}{2} \langle \Psi^{ s \bar{s} }_{a \bar{a}} | \mathscr{H} | \Psi^{ s \bar{r} }_{a \bar{b}} \rangle \\
		&= \frac{1}{2} \left( \langle \bar{r} \bar{b} || \bar{a} \bar{s} \rangle - \langle r \bar{b} || s \bar{a} \rangle - \langle s \bar{b} || r \bar{a} \rangle + \langle \bar{s} \bar{b} || \bar{a} \bar{r} \rangle \right) \\
		&= \frac{1}{2} \left( (ra|bs) - (rs|ba) - (rs|ba) - (sr|ba) + (sa|br) - (sr|ba) \right) \\
		&= \frac{1}{2} \left( \frac{ K_{12} }{2} - \frac{ J_{12} }{2} - \frac{ J_{12} }{2} - \frac{ J_{12} }{2} + \frac{ K_{12} }{2} - \frac{ J_{12} }{2} \right) = \frac{1}{2} K_{12} - J_{12}, \\
		\langle \Psi^{**}_{a\bar{a}} | \mathscr{H} | \Psi^{**}_{\bar{a} b} \rangle &= \left( \frac{1}{ \sqrt{2} } \langle \Psi^{ r \bar{r} }_{a \bar{a}} | + \frac{1}{ \sqrt{2} } \langle \Psi^{ s \bar{s} }_{a \bar{a}} | \right) \mathscr{H} \left( \frac{1}{ \sqrt{2} } | \Psi^{ \bar{s} r }_{\bar{a} b} \rangle + \frac{1}{ \sqrt{2} } | \Psi^{ \bar{r} s }_{\bar{a} b} \rangle \right) \\
		&= \frac{1}{2} \langle \Psi^{ r \bar{r} }_{a \bar{a}} | \mathscr{H} | \Psi^{ \bar{s} r }_{\bar{a} b} \rangle + \frac{1}{2} \langle \Psi^{ r \bar{r} }_{a \bar{a}} | \mathscr{H} | \Psi^{ \bar{r} s }_{ \bar{a} b } \rangle + \frac{1}{2} \langle \Psi^{ s \bar{s} }_{a \bar{a}} | \mathscr{H} | \Psi^{ \bar{s} r }_{ \bar{a} b } \rangle + \frac{1}{2} \langle \Psi^{ s \bar{s} }_{a \bar{a}} | \mathscr{H} | \Psi^{ \bar{r} s }_{ \bar{a} b } \rangle \\
		&= \frac{1}{2} \left( - \langle \bar{r} b || \bar{s} a \rangle + \langle r b || a s \rangle + \langle s b || a r \rangle - \langle \bar{s} b || \bar{r} a \rangle \right) \\
		&= \frac{1}{2} \left( - (rs|ba) + (ra|bs) - (rs|ba) + (sa|br) - (sr|ba) - (sr|ba) \right) \\
		&= \frac{1}{2} \left( - \frac{ J_{12} }{2} + \frac{ K_{12} }{2} - \frac{ J_{12} }{2} + \frac{ K_{12} }{2} - \frac{ J_{12} }{2} - \frac{ J_{12} }{2} \right) = \frac{1}{2} K_{12} - J_{12}, \\
		\langle \Psi^{**}_{b\bar{b}} | \mathscr{H} | \Psi^{**}_{a\bar{b}} \rangle &= \left( \frac{1}{ \sqrt{2} } \langle \Psi^{ r \bar{r} }_{b \bar{b}} | + \frac{1}{ \sqrt{2} } \langle \Psi^{ s \bar{s} }_{b \bar{b}} | \right) \mathscr{H} \left( \frac{1}{ \sqrt{2} } | \Psi^{ r \bar{s} }_{a \bar{b}} \rangle + \frac{1}{ \sqrt{2} } | \Psi^{ s \bar{r} }_{a \bar{b}} \rangle \right) \\
		&= \frac{1}{2} \langle \Psi^{ r \bar{r} }_{b \bar{b}} | \mathscr{H} | \Psi^{ r \bar{s} }_{a \bar{b}} \rangle + \frac{1}{2} \langle \Psi^{ r \bar{r} }_{b \bar{b}} | \mathscr{H} | \Psi^{ s \bar{r} }_{a \bar{b}} \rangle + \frac{1}{2} \langle \Psi^{ s \bar{s} }_{b \bar{b}} | \mathscr{H} | \Psi^{ r \bar{s} }_{a \bar{b}} \rangle + \frac{1}{2} \langle \Psi^{ s \bar{s} }_{b \bar{b}} | \mathscr{H} | \Psi^{ s \bar{r} }_{a \bar{b}} \rangle \\
		&= \frac{1}{2} \left( - \langle a \bar{r} || b \bar{s} \rangle + \langle a r || s b \rangle + \langle a s || r b \rangle - \langle a \bar{s} || b \bar{r} \rangle \right) \\
		&= \frac{1}{2} \left( -(ab|rs) + (as|rb) - (ab|rs) + (ar|sb) - (ab|sr) - (ab|sr) \right) \\
		&= \frac{1}{2} \left( - \frac{ J_{12} }{2} + \frac{ K_{12} }{2} - \frac{ J_{12} }{2} + \frac{ K_{12} }{2} - \frac{ J_{12} }{2} - \frac{ J_{12} }{2} \right) = \frac{1}{2} K_{12} - J_{12}, \\
		\langle \Psi^{**}_{b\bar{b}} | \mathscr{H} | \Psi^{**}_{ \bar{a} b} \rangle &= \left( \frac{1}{ \sqrt{2} } \langle \Psi^{ r \bar{r} }_{b \bar{b}} | + \frac{1}{ \sqrt{2} } \langle \Psi^{ s \bar{s} }_{b \bar{b}} | \right) \mathscr{H} \left( \frac{1}{ \sqrt{2} } | \Psi^{ \bar{s} r }_{\bar{a} b} \rangle + \frac{1}{ \sqrt{2} } | \Psi^{ \bar{r} s }_{\bar{a} b} \rangle \right) \\
		&= \frac{1}{2} \langle \Psi^{ r \bar{r} }_{b \bar{b}} | \mathscr{H} | \Psi^{ \bar{s} r }_{\bar{a} b} \rangle + \frac{1}{2} \langle \Psi^{ r \bar{r} }_{b \bar{b}} | \mathscr{H} | \Psi^{ \bar{r} s }_{\bar{a} b} \rangle + \frac{1}{2} \langle \Psi^{ s \bar{s} }_{b \bar{b}} | \mathscr{H} | \Psi^{ \bar{s} r }_{\bar{a} b} \rangle + \frac{1}{2} \langle \Psi^{ s \bar{s} }_{b \bar{b}} | \mathscr{H} | \Psi^{ \bar{r} s }_{\bar{a} b} \rangle \\
		&= \frac{1}{2} \left( \langle \bar{a} \bar{r} || \bar{s} \bar{b} \rangle - \langle \bar{a} r || \bar{b} s \rangle - \langle \bar{a} s || \bar{b} r \rangle + \langle \bar{a} \bar{s} || \bar{r} \bar{b} \rangle \right) \\
		&= \frac{1}{2} \left( (as|rb) - (ab|rs) - (ab|rs) - (ab|sr) + (ar|sb) - (ab|sr) \right) \\
		&= \frac{1}{2} \left( \frac{ K_{12} }{2} - \frac{ J_{12} }{2} - \frac{ J_{12} }{2} - \frac{ J_{12} }{2} + \frac{ K_{12} }{2} - \frac{ J_{12} }{2} \right) = \frac{1}{2} K_{12} - J_{12}, \\
		\langle \Psi^{**}_{a\bar{b}} | \mathscr{H} | \Psi^{**}_{ \bar{a} b} \rangle &= \left( \frac{1}{ \sqrt{2} } \langle \Psi^{ r \bar{s} }_{a \bar{b}} | + \frac{1}{ \sqrt{2} } \langle \Psi^{ s \bar{r} }_{a \bar{b}} | \right) \mathscr{H} \left( \frac{1}{ \sqrt{2} } | \Psi^{ \bar{s} r }_{\bar{a} b} \rangle + \frac{1}{ \sqrt{2} } | \Psi^{ \bar{r} s }_{\bar{a} b} \rangle \right) \\
		&= \frac{1}{2} \langle \Psi^{ r \bar{s} }_{a \bar{b}} | \mathscr{H} | \Psi^{ \bar{s} r }_{\bar{a} b} \rangle + \frac{1}{2} \langle \Psi^{ r \bar{s} }_{a \bar{b}} | \mathscr{H} | \Psi^{ \bar{r} s }_{\bar{a} b} \rangle + \frac{1}{2} \langle \Psi^{ s \bar{r} }_{a \bar{b}} | \mathscr{H} | \Psi^{ \bar{s} r }_{\bar{a} b} \rangle + \frac{1}{2} \langle \Psi^{ s \bar{r} }_{a \bar{b}} | \mathscr{H} | \Psi^{ \bar{r} s }_{\bar{a} b} \rangle \\
		&= \frac{1}{2} \left( \langle \bar{a} b || \bar{b} a \rangle + 0 + 0 + \langle \bar{a} b || \bar{b} a \rangle \right) = \frac{1}{2} \left( K_{ab} + K_{ab} \right) = \frac{ 1 }{2} J_{11}.
	\end{align*}
	Thus, the corresponding DCI eigenvalue problem is
	\begin{sequation}\label{eq:DCI_matrix}
	\begin{pmatrix}
		0 & \frac{1}{\sqrt{2}} K_{12} & \frac{1}{\sqrt{2}}K_{12} & \frac{1}{\sqrt{2}} K_{12} & \frac{1}{\sqrt{2}}K_{12} \\
		\frac{1}{\sqrt{2}}K_{12} & 2\Delta^\prime & \frac{1}{2}J_{11} & \frac{1}{2}K_{12} - J_{12} & \frac{1}{2}K_{12} - J_{12} \\
		\frac{1}{\sqrt{2}}K_{12} & \frac{1}{2}J_{11} & 2\Delta^\prime & \frac{1}{2}K_{12} - J_{12} & \frac{1}{2}K_{12} - J_{12} \\
		\frac{1}{\sqrt{2}}K_{12} & \frac{1}{2}K_{12} - J_{12} & \frac{1}{2}K_{12} - J_{12} & 2\Delta^\prime & \frac{1}{2}J_{11} \\
		\frac{1}{\sqrt{2}}K_{12} & \frac{1}{2}K_{12} - J_{12} & \frac{1}{2}K_{12} - J_{12} & \frac{1}{2}J_{11} & 2\Delta^\prime
	\end{pmatrix} \begin{pmatrix}
	1 \\ c_1 \\ c_2 \\ c_3 \\ c_4
	\end{pmatrix} = {}^2 E_\corr (\DCI) \begin{pmatrix}
	1 \\ c_1 \\ c_2 \\ c_3 \\ c_4
	\end{pmatrix}.
	\end{sequation}
	
	\item[b.] From \eqref{eq:DCI_matrix}, we know that
	\begin{align*}
		\frac{K_{12}}{\sqrt{2}} + 2\Delta^\prime c_1 + \frac{J_{11}}{2} c_2 + \frac{ K_{12} - J_{12} }{2} c_3 + \frac{ K_{12} - J_{12} }{2} c_4 &= {}^2 E_\corr (\DCI) c_1 , \\
		\frac{K_{12}}{\sqrt{2}} + \frac{J_{11}}{2} c_1 + 2\Delta^\prime c_2  + \frac{ K_{12} - J_{12} }{2} c_3 + \frac{ K_{12} - J_{12} }{2} c_4 &= {}^2 E_\corr (\DCI) c_2.
	\end{align*}
	The first equation can be substracted by the second one, viz.,
	\[
		( 2\Delta^\prime - \frac{ J_{11} }{2} - {}^2 E_\corr (\DCI) )( c_1 - c_2 ) = 0.
	\]
	Assume that ${}^2 E_\corr (\DCI) \neq 2\Delta^\prime - \frac{ J_{11} }{2}$ (in fact, this holds true), we find that
	\[
		c_1 = c_2.
	\]		
	In this way, we can prove $c_1 = c_2 = c_3 = c_4$. In fact, we can permute 1 and 2 of the second equation of \eqref{eq:DCI_matrix}, then we will find that the third equation of \eqref{eq:DCI_matrix} is obtained, and vice versa. Thus $c_1 = c_2$. This method is suitable for not only $c_1$ and $c_2$, but also $c_1$ and $c_3$, $c_1$ and $c_4$. Hence we can also conclude that $c_1 = c_2 = c_3 = c_4$.
	
	Thus, we set $c_1 = c$, thus
	\begin{align*}
		{}^2 E_\corr (\DCI) &= 2\sqrt{2} K_{12} c , \\
		{}^2 E_\corr (\DCI) c &= \frac{ K_{12} }{ \sqrt{2} } + ( 2\Delta^\prime + \frac{1}{2} J_{11} + \frac{1}{2} K_{12} - J_{12} + \frac{1}{2} K_{12} - J_{12} )c = \frac{ K_{12} }{ \sqrt{2} } + 2 \Delta c .
	\end{align*}
	In fact, they can be converted to a quadratic equation,
	\[
		( {}^2 E_\corr (\DCI) )^2 - 2 \Delta ( {}^2 E_\corr (\DCI) ) - 2 K^2_{12} = 0,
	\]
	The discriminant $\Delta_E$ of this quadratic equation is
	\[
		\Delta_E = (-2 \Delta)^2 - 4 \times 1 \times ( -2 K^2_{12} ) = 4( \Delta^2 + 2 K^2_{12} ) > 0,
	\]
	and the root are
	\[
		E_1 = \Delta + \sqrt{ \Delta^2 + 2 K^2_{12} }, \quad E_2 = \Delta - \sqrt{ \Delta^2 + 2 K^2_{12} }.
	\]	
	Therefore, the lowest root is the correlation energy, viz.,
	\begin{sequation}
		{}^2 E_\corr = \Delta - \sqrt{ \Delta^2 + 2 K^2_{12} }.
	\end{sequation}
	which is the same result as found in the last chapter (see Eq.(4.60)). 
	
	\end{itemize}		
	
	\end{solution}
	
	% 5.8
	\begin{exercise}
	Show directly from Eq.(5.19) using delocalized orbitals and the two-electron integrals in Eq.(5.26) that the total first-order pair correlation energy (which is the same as the many-body second-order perturbation energy) of the dimer is given by Eq.(5.46).
	\end{exercise}
	
	\begin{solution}
	In fact, from Eq.(5.18) using delocalized orbitals and the two-electron integrals in Eq.(5.26), it is also easy to derive the the total first-order pair correlation energy. However, I think this method is much better because it shows the absence of $e^\FO_{a\bar{a}}$ and $e^\FO_{b\bar{b}}$ as their corresponding configurations are {\it ungerade} while $\Psi_0$ is {\it gerade}. Thus I will firstly show $e^\FO_{ab}$, $e^\FO_{a\bar{b}}$, $e^\FO_{b\bar{a}}$, $e^\FO_{b\bar{b}}$ and then calculate $^2 E_\corr( {\rm FO}(D) )$. 
	
	Note that the integrals, which have three {\it gerade} orbitals and one {\it ungerade} orbital, or three {\it ungerade} orbitals and one {\it gerade} orbital, vanish. Thus,
	\begin{align*}
		e^\FO_{a\bar{a}} &= \frac{ | \langle a \bar{a} || r \bar{r} \rangle |^2 }{ \varepsilon_a + \varepsilon_{\bar{a}} - \varepsilon_r - \varepsilon_{\bar{r}} }	+ \frac{ | \langle a \bar{a} || s \bar{s} \rangle |^2 }{ \varepsilon_a + \varepsilon_{\bar{a}} - \varepsilon_s - \varepsilon_{\bar{s}} } + \frac{ | \langle a \bar{a} || r \bar{s} \rangle |^2 }{ \varepsilon_a + \varepsilon_{\bar{a}} - \varepsilon_r - \varepsilon_{\bar{s}} } + \frac{ | \langle a \bar{a} || s \bar{r} \rangle |^2 }{ \varepsilon_a + \varepsilon_{\bar{a}} - \varepsilon_s - \varepsilon_{\bar{r}} } \\
		&= \frac{1}{2(\varepsilon_1 - \varepsilon_2)} \left( \frac{1}{4} K^2_{12} + \frac{1}{4} K^2_{12} + 0 + 0 \right) = \frac{ K^2_{12} }{ 4(\varepsilon_1 - \varepsilon_2) }. 
	\end{align*}
	Similarly, we obtain
	\begin{align*}
		e^\FO_{a\bar{b}} &= \frac{ | \langle a \bar{b} || r \bar{r} \rangle |^2 }{ \varepsilon_a + \varepsilon_{\bar{b}} - \varepsilon_r - \varepsilon_{\bar{r}} }	+ \frac{ | \langle a \bar{b} || s \bar{s} \rangle |^2 }{ \varepsilon_a + \varepsilon_{\bar{b}} - \varepsilon_s - \varepsilon_{\bar{s}} } + \frac{ | \langle a \bar{b} || r \bar{s} \rangle |^2 }{ \varepsilon_a + \varepsilon_{\bar{b}} - \varepsilon_r - \varepsilon_{\bar{s}} } + \frac{ | \langle a \bar{b} || s \bar{r} \rangle |^2 }{ \varepsilon_a + \varepsilon_{\bar{b}} - \varepsilon_s - \varepsilon_{\bar{r}} } = \frac{ K^2_{12} }{ 4(\varepsilon_1 - \varepsilon_2) } , \\
		e^\FO_{\bar{b}a} &= \frac{ | \langle b \bar{a} || r \bar{r} \rangle |^2 }{ \varepsilon_b + \varepsilon_{\bar{a}} - \varepsilon_r - \varepsilon_{\bar{r}} }	+ \frac{ | \langle b \bar{a} || s \bar{s} \rangle |^2 }{ \varepsilon_b + \varepsilon_{\bar{a}} - \varepsilon_s - \varepsilon_{\bar{s}} } + \frac{ | \langle b \bar{a} || r \bar{s} \rangle |^2 }{ \varepsilon_b + \varepsilon_{\bar{a}} - \varepsilon_r - \varepsilon_{\bar{s}} } + \frac{ | \langle b \bar{a} || s \bar{r} \rangle |^2 }{ \varepsilon_b + \varepsilon_{\bar{a}} - \varepsilon_s - \varepsilon_{\bar{r}} } = \frac{ K^2_{12} }{ 4(\varepsilon_1 - \varepsilon_2) } , \\
		e^\FO_{\bar{b}b} &= \frac{ | \langle b \bar{b} || r \bar{r} \rangle |^2 }{ \varepsilon_b + \varepsilon_{\bar{b}} - \varepsilon_r - \varepsilon_{\bar{r}} }	+ \frac{ | \langle b \bar{b} || s \bar{s} \rangle |^2 }{ \varepsilon_b + \varepsilon_{\bar{b}} - \varepsilon_s - \varepsilon_{\bar{s}} } + \frac{ | \langle b \bar{b} || r \bar{s} \rangle |^2 }{ \varepsilon_b + \varepsilon_{\bar{b}} - \varepsilon_r - \varepsilon_{\bar{s}} } + \frac{ | \langle b \bar{b} || s \bar{r} \rangle |^2 }{ \varepsilon_b + \varepsilon_{\bar{b}} - \varepsilon_s - \varepsilon_{\bar{r}} } = \frac{ K^2_{12} }{ 4(\varepsilon_1 - \varepsilon_2) }. 
	\end{align*}
	Thus
	\begin{sequation}
		^2 E_\corr( {\rm FO}(D) ) = e^\FO_{a\bar{a}} + e^\FO_{a\bar{b}} + e^\FO_{\bar{b}a} + e^\FO_{\bar{b}b} = \frac{ K^2_{12} }{ \varepsilon_1 - \varepsilon_2 } = 2 \left( \frac{ K^2_{12} }{ 2 ( \varepsilon_1 - \varepsilon_2 ) } \right).
	\end{sequation}

	\end{solution}

	% 5.9
	\begin{exercise}
	Show that the total correlation energy obtained using Epstein-Nesbet pairs is not invariant to unitary transformations.
	\begin{enumerate}
	
	\item[a.] Show, using localized orbitals, that
	\[
		{}^{2} E_\corr({\rm EN}(L)) = - \frac{K^2_{12}}{\Delta}.
	\]
	
	\item[b.] Show, using delocalized spin-orbital pairs, that
	\[
		{}^{2} E_\corr({\rm EN}(D)) = - \frac{K^2_{12}}{\Delta^\prime}.
	\]
	
	\item[c.] Show, using delocalized spin-adapted pairs, that
	\[
		{}^{2} E^{\rm singlet}_\corr({\rm EN}(D)) = - \frac{ K^2_{12} }{ 2 \Delta^\prime } - \frac{ K^2_{12} }{ 2 \Delta^{\prime\prime} }.
	\]
	
	\item[d.] Using the STO-3G integrals for $\ce{H2}$ in Appendix D compare the numerical values of the above expressions at $R=1.4$ a.u.
	\end{enumerate}
	\end{exercise}
	
	\begin{solution}
		5-9 so
	\end{solution}


\end{document}
