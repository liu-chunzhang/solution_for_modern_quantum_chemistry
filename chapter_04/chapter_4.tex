\documentclass[a4paper]{book}
\usepackage{amsmath}\special{dvipdfmx:config z 0} %取消PDF压缩,加快速度,最终版本生成的时候最好把这句话注释掉

\usepackage{amssymb}
\usepackage[hypcap=false]{caption}
\usepackage{enumitem}	% 定制enumerate标号
\usepackage{fancyhdr}
\pagestyle{plain} 				% 此处为fancy时有页眉
\usepackage{geometry}
\geometry{
	left=2cm,
	right=2cm,
	top=2cm,
	bottom=2cm,
}
\usepackage{hyperref}
\hypersetup{
    colorlinks=true,            %链接颜色
    linkcolor=blue,             %内部链接
    filecolor=magenta,          %本地文档
    urlcolor=cyan,              %网址链接
}
\usepackage[none]{hyphenat}		% 阻止长单词分在两行
\usepackage{mathrsfs}
\usepackage[version=4]{mhchem}
\usepackage{subcaption}
\usepackage{titlesec}

\RequirePackage[many]{tcolorbox}
\tcbset{
    boxed title style={colback=magenta},
	breakable,
	enhanced,
	sharp corners,
	attach boxed title to top left={yshift=-\tcboxedtitleheight,  yshifttext=-.75\baselineskip},
	boxed title style={boxsep=1pt,sharp corners},
    fonttitle=\bfseries\sffamily,
}

\definecolor{skyblue}{rgb}{0.54, 0.81, 0.94}

\newtcolorbox[auto counter, number within=chapter, number format=\arabic]{exercise}[1][]{
    title={Exercise~\thetcbcounter},
    colframe=skyblue,
    colback=skyblue!12!white,
    boxed title style={colback=skyblue},
    overlay unbroken and first={
        \node[below right,font=\small,color=skyblue,text width=.8\linewidth]
        at (title.north east) {#1};
    }
}

\newtcolorbox[auto counter, number within=chapter, number format=\arabic]{solution}[1][]{
    title={Solution~\thetcbcounter},
    colframe=teal!60!green,
    colback=green!12!white,
    boxed title style={colback=teal!60!green},
    overlay unbroken and first={
        \node[below right,font=\small,color=red,text width=.8\linewidth]
        at (title.north east) {#1};
    }
}

% special new commands for common symbols used in the article
\newcommand\la{\langle}
\newcommand\ra{\rangle}
\newcommand\lr[2]{\langle#1\|#2\rangle}
\newcommand\tr[1]{\mathrm{tr(#1)}}
\newcommand\Tr[3]{#1\mathrm\#2#3}
\newcommand*{\dif}{\mathop{}\!\mathrm{d}}
\renewcommand\det[1]{{\rm det}\left(#1\right)}
\newcommand{\HF}{{\rm HF}}
\newcommand{\corr}{{\rm corr}}
\newcommand{\au}{{\rm a.u.}}

\newcommand{\A}{{\bf A}}
\newcommand{\B}{{\bf B}}
\newcommand{\C}{{\bf C}}
\newcommand{\I}{{\bf 1}}
\newcommand{\U}{{\bf U}}
\newcommand{\Op}{{\bf O}}
\newcommand{\h}{{\bf h}}

\newcommand\Figref[1]{Fig \ref{#1}}
\newcommand\Tableref[1]{Table \ref{#1}}

\titleformat{\chapter}[display]
  {\bfseries\Large}
  {\filright\MakeUppercase{\chaptertitlename} \Huge\thechapter}
  {1ex}
  {\titlerule\vspace{1ex}\filleft}
  [\vspace{1ex}\titlerule]
  
\allowdisplaybreaks

\begin{document}

	\stepcounter{chapter}\stepcounter{chapter}\stepcounter{chapter}

	\chapter{Configuration Interaction}
	
	\section{Multiconfigurational Wave Functions and the Structure of the Full CI Matrix}
	
	\subsection{Intermediate Normalization and an Expression for the Correlation Energy}
	
	% 4.1
	\begin{exercise}
	Obtain Eq.(4.12) from Eq.(4.11). It will prove convenient to use unrestricted summations.
	\end{exercise}
	
	\begin{solution}
	
	Note that the index $r$ must be included in the set $\{ t, u, v \}$ and the index $a$ must be included in the set $\{ c,d,e\}$ for a matrix element of $\langle \Psi^r_a | \mathscr{H} | \Psi^{tuv}_{cde} \rangle$. Therefore, we find that
	\begin{align*}
		&\hspace{1.4em}\sum_{ \substack{ c<d<e \\ t<u<v } } \langle \Psi^r_a | \mathscr{H} | \Psi^{tuv}_{cde} \rangle c^{tuv}_{cde} = \frac{1}{\left(3!\right)^2} \sum_{ \substack{ cde \\ tuv} } \langle \Psi^r_a | \mathscr{H} | \Psi^{tuv}_{cde} \rangle c^{tuv}_{cde} \\
		&= \frac{1}{\left(3!\right)^2}  \left[ \sum_{ \substack{ de \\ uv } } \langle \Psi^r_a | \mathscr{H} | \Psi^{ruv}_{ade} \rangle c^{ruv}_{ade} + \sum_{ \substack{ de \\ tv } } \langle \Psi^r_a | \mathscr{H} | \Psi^{trv}_{ade} \rangle c^{trv}_{ade} + \sum_{ \substack{ de \\ tu } } \langle \Psi^r_a | \mathscr{H} | \Psi^{tur}_{ade} \rangle c^{tur}_{ade} \right. \\
		&\hspace{ 6em } \left. + \sum_{ \substack{ ce \\ uv } } \langle \Psi^r_a | \mathscr{H} | \Psi^{ruv}_{cae} \rangle c^{ruv}_{cae} + \sum_{ \substack{ ce \\ tv } } \langle \Psi^r_a | \mathscr{H} | \Psi^{trv}_{cae} \rangle c^{trv}_{cae} + \sum_{ \substack{ ce \\ tu } } \langle \Psi^r_a | \mathscr{H} | \Psi^{tur}_{cae} \rangle c^{tur}_{cae} \right. \\
		&\hspace{ 6em } \left. + \sum_{ \substack{ cd \\ uv } } \langle \Psi^r_a | \mathscr{H} | \Psi^{ruv}_{cda} \rangle c^{ruv}_{cda} + \sum_{ \substack{ cd \\ tv } } \langle \Psi^r_a | \mathscr{H} | \Psi^{trv}_{cda} \rangle c^{trv}_{cda} + \sum_{ \substack{ cd \\ tu } } \langle \Psi^r_a | \mathscr{H} | \Psi^{tur}_{cda} \rangle c^{tur}_{cda} \right] .
	\end{align*}
	
	Then, these dummy indices should be converted into the same one, viz.,
	\begin{align*}
		&\hspace{1.4em} \sum_{ \substack{ c<d<e \\ t<u<v } } \langle \Psi^r_a | \mathscr{H} | \Psi^{tuv}_{cde} \rangle c^{tuv}_{cde} \\
		&= \frac{1}{\left(3!\right)^2}  \left[ \sum_{ \substack{ cd \\ tu } } \langle \Psi^r_a | \mathscr{H} | \Psi^{rtu}_{acd} \rangle c^{rtu}_{acd} + \sum_{ \substack{ cd \\ tu } } \langle \Psi^r_a | \mathscr{H} | \Psi^{tru}_{acd} \rangle c^{tru}_{acd} + \sum_{ \substack{ cd \\ tu } } \langle \Psi^r_a | \mathscr{H} | \Psi^{tur}_{acd} \rangle c^{tur}_{acd} \right. \\
		&\hspace{ 6em } + \sum_{ \substack{ cd \\ tu } } \langle \Psi^r_a | \mathscr{H} | \Psi^{rtu}_{cad} \rangle c^{rtu}_{cad} + \sum_{ \substack{ cd \\ tu } } \langle \Psi^r_a | \mathscr{H} | \Psi^{tru}_{cad} \rangle c^{tru}_{cad} + \sum_{ \substack{ cd \\ tu } } \langle \Psi^r_a | \mathscr{H} | \Psi^{tur}_{cad} \rangle c^{tur}_{cad} \\
		&\hspace{ 6em } \left. + \sum_{ \substack{ cd \\ tu } } \langle \Psi^r_a | \mathscr{H} | \Psi^{rtu}_{cda} \rangle c^{rtu}_{cda} + \sum_{ \substack{ cd \\ tu } } \langle \Psi^r_a | \mathscr{H} | \Psi^{tru}_{cda} \rangle c^{tru}_{cda} + \sum_{ \substack{ cd \\ tu } } \langle \Psi^r_a | \mathscr{H} | \Psi^{tur}_{cda} \rangle c^{tur}_{cda} \right] \\
		&= \frac{1}{\left(3!\right)^2} \times 9 \sum_{ \substack{ cd \\ tu } } \langle \Psi^r_a | \mathscr{H} | \Psi^{rtu}_{acd} \rangle c^{rtu}_{acd} = \frac{1}{\left(2!\right)^2} \sum_{ \substack{ cd \\ tu } } \langle \Psi^r_a | \mathscr{H} | \Psi^{rtu}_{acd} \rangle c^{rtu}_{acd} = \sum_{ \substack{ c<d \\ t<u } } \langle \Psi^r_a | \mathscr{H} | \Psi^{rtu}_{acd} \rangle c^{rtu}_{acd}.
	\end{align*}
	Thus, we have proved that
	\begin{equation}
		\sum_{ \substack{ c<d<e \\ t<u<v } } \langle \Psi^r_a | \mathscr{H} | \Psi^{tuv}_{cde} \rangle c^{tuv}_{cde} = \sum_{ \substack{ c<d \\ t<u } } \langle \Psi^r_a | \mathscr{H} | \Psi^{rtu}_{acd} \rangle c^{rtu}_{acd}.
	\end{equation}
	
	With this equation, it is clear that (4.12) can be obtained from (4.11).
	
	\end{solution}
	
	% 4.2
	\begin{exercise}
	Using the secular determinant approach show that the lowest eigenvalue of the matrix
	\[
		\begin{pmatrix}
			0 & K_{12} \\ K_{12} & 2\Delta
		\end{pmatrix}
	\]
	is given by Eq.(4.23).
	\end{exercise}
	
	\begin{solution}
	
	The introduction of the secular determinant approach is demonstrated in the page 18. The matrix in the exercise 4.2 is denoted as $H$, then
	\[
		\det{H - \varepsilon I} = \begin{vmatrix}
		- \varepsilon & K_{12} \\ K_{12} & 2\Delta - \varepsilon
		\end{vmatrix} = \varepsilon^2 - 2 \Delta \varepsilon -K^2_{12} = 0 ,
	\]	
	The discriminant $\Delta_E$ of this quadratic equation is
	\[
		\Delta_E = 4 \Delta^2 - 4 \times ( -K^2_{12} ) = 4( \Delta^2 + K^2_{12} )
	\]	
	Thus, the root are
	\[
		E_1 = \Delta + \sqrt{ \Delta^2 + K^2_{12} }, \quad E_2 = \Delta - \sqrt{ \Delta^2 + K^2_{12} }.
	\]	
	
	Therefore, the lowest root is the correlation energy, viz.,
	\begin{equation}
		E_\corr = \Delta - \sqrt{ \Delta^2 + K^2_{12} }.
	\end{equation}
	
	\end{solution}
	
	% 4.3
	\begin{exercise}
	Calculate the coefficient of the double excitation ($c$) in the intermediate normalized CI wave function at $R$ = 1.4 $\au$, using the STO-3G integrals given in Appendix D. Show analytically that as $R \rightarrow \infty$, $c \rightarrow -1$, and hence that at large distances the Hartree-Fock ground state and the doubly excited configuration have equal weight in the CI ground state. Finally, show that the CI wave function, when normalized to unity, becomes (at $R=\infty$)
	\[
		\frac{1}{\sqrt{2}} \left( | \phi_1 \bar{\phi}_2 \rangle + | \phi_2 \bar{\phi}_1 \rangle \right)
	\]
	where $\phi_1$ and $\phi_2$ are atomic orbitals on centers one and two, respectively.
	\end{exercise}
	
	\begin{solution}
	
	When $R$ = 1.4. $\au$, we know that
	\begin{center}
	\begin{tabular}{ccc}
		$\varepsilon_1 = -0.5782\,\au$, & $\varepsilon_2 = 0.6703\, \au$, & $J_{11} = 0.6746\,\au$, \\
		$J_{12} = 0.6636\,\au$, & $J_{22} = 0.6975\,\au$, & $K_{12} = 0.1813\,\au$
	\end{tabular}
	\end{center}
	
	Firstly, with (4.20), we calculate $2\Delta$ at $R$ = 1.4. $\au$, viz.,
	\[
		2\Delta = \left[ 2( \varepsilon_2 - \varepsilon_1 ) + J_{11} + J_{22} - 4J_{12} + 2K_{12} \right] = 1.5773 \, \au
	\]
	In other words, $\Delta = 0.78865 \, \au$ Thus, the correlation energy $E_\corr$ at $R$ = 1.4. $\au$ is
	\[
		E_\corr = \Delta - \sqrt{ \Delta^2 + K^2_{12} } = -0.02057 \, \au.
	\]
	Therefore,
	\begin{equation}
		c = \frac{ K_{12} }{ E_\corr - 2\Delta } =  \frac{ 0.1813 \, \au }{ -0.02057 \, \au. - 1.5773 \, \au. }  \approx - 0.1135.
	\end{equation}
	
	Indeed, we can find that
	\[
		\Delta = \varepsilon_2 - \varepsilon_1 + \frac{1}{2} J_{11} + \frac{1}{2} J_{22} - 2J_{12} + K_{12} = h_{22} - h_{11} - \frac{1}{2} J_{11} + \frac{1}{2} J_{12}.
	\]
	It is clear that	
	\[
		\lim_{ R \rightarrow \infty} \Delta = \lim_{ R \rightarrow \infty} \left[ h_{22} - h_{11} + \frac{1}{2} J_{22} - \frac{1}{2} J_{11} \right] = E(\ce{H}) - E(\ce{H}) + \frac{1}{4} ( \phi_1 \phi_1 | \phi_1 \phi_1 ) - \frac{1}{4} ( \phi_1 \phi_1 | \phi_1 \phi_1 ) = 0.
	\]
	Thus,
	\begin{align*}
		\lim_{ R \rightarrow \infty } c &= \lim_{ R \rightarrow \infty } \frac{ K_{12} }{ E_\corr - 2\Delta } = \lim_{ R \rightarrow \infty } \frac{ K_{12} }{ \Delta - \sqrt{ \Delta^2 + K^2_{12} } - 2\Delta } = \lim_{ R \rightarrow \infty } \frac{ - K_{12} }{ \Delta + \sqrt{ \Delta^2 + K^2_{12} } } \\
		&= - \lim_{ \Delta \rightarrow 0 } \frac{ 1 }{ \frac{ \Delta }{ K_{12} } + \sqrt{ 1 + \left( \frac{ \Delta }{ K_{12} } \right)^2 } } = - \lim_{ x \rightarrow 0 } \frac{ 1 }{ x + \sqrt{ 1 + x^2 } } = -1.
	\end{align*}
	This conclusion means that at large distances the Hartree-Fock ground state $\Psi_0$ and the doubly excited configuration $\Psi^{2 \bar{2}}_{1 \bar{1}}$ have equal weight in the CI ground state $\Phi$, viz.,
	\[
		\lim_{ R \rightarrow \infty } | \Phi \rangle = | \Psi_0 \rangle - | \Psi^{2 \bar{2}}_{1 \bar{1}} \rangle = | \psi_1 \bar{\psi}_1 \rangle - | \psi_2 \bar{\psi}_2 \rangle.
	\]
	Note that as $R \rightarrow \infty$, from (3.236) and (3.237), we find that
	\[
		\lim_{ R \rightarrow \infty } \psi_1 = \frac{1}{ \sqrt{2} } ( \phi_1 + \phi_2 ) , \quad \lim_{ R \rightarrow \infty } \psi_2 = \frac{1}{ \sqrt{2} } ( \phi_1 - \phi_2 ).
	\]
	Thus,
	\begin{align*}
		\lim_{ R \rightarrow \infty } | \psi_1 \bar{\psi}_1 \rangle &= \frac{1}{2} | ( \phi_1 + \phi_2 ) ( \bar{\phi}_1 + \bar{\phi}_2 ) \rangle = \frac{1}{2} \left( | \phi_1 \bar{\phi}_1 \rangle + | \phi_1 \bar{\phi}_2 \rangle + | \phi_2 \bar{\phi}_1 \rangle + | \phi_2 \bar{\phi}_2 \rangle \right), \\
		\lim_{ R \rightarrow \infty } | \psi_2 \bar{\psi}_2 \rangle &= \frac{1}{ 2 } | ( \phi_1 - \phi_2 ) ( \bar{\phi}_1 - \bar{\phi}_2 ) \rangle = \frac{1}{2} \left( | \phi_1 \bar{\phi}_1 \rangle - | \phi_1 \bar{\phi}_2 \rangle - | \phi_2 \bar{\phi}_1 \rangle + | \phi_2 \bar{\phi}_2 \rangle \right),
	\end{align*}
	and then	
	\[
		\lim_{ R \rightarrow \infty } | \Phi \rangle = \lim_{ R \rightarrow \infty } | \psi_1 \bar{\psi}_1 \rangle - \lim_{ R \rightarrow \infty } | \psi_2 \bar{\psi}_2 \rangle = | \phi_1 \bar{\phi}_2 \rangle + | \phi_2 \bar{\phi}_1 \rangle
	\]
	Thus, at $R = \infty$, the normalized CI wave function is
	\begin{equation}
		\lim_{R \rightarrow \infty} | \Phi \rangle = \lim_{R \rightarrow \infty} \frac{1}{ \langle \Phi_0 | \Phi_0 \rangle } | \Phi_0 \rangle = \frac{1}{ \sqrt{2} } \left( | \phi_1 \bar{\phi}_2 \rangle + | \phi_2 \bar{\phi}_1 \rangle \right) .
	\end{equation}
	
	We have proved two conclusions at $R = \infty$, the equal weight of the Hartree-Fock ground state $\Psi_0$ and the doubly excited configuration $\Psi^{2 \bar{2}}_{1 \bar{1}}$, and the form of normalized CI wave function.
		
	\end{solution}
	
	\section{Doubly Excited CI}
	
	\section{Some Illustrative Calculations}
	
	\section{Natural Orbitals and the One-Particle Reduced Density Matrix}
	
	% 4.4
	\begin{exercise}
	Show that $\gamma$ is a Hermitian matrix.
	\end{exercise}
	
	\begin{solution}
	Firstly, we find that $\gamma( \boldsymbol{x}_1 , \boldsymbol{x}^\prime_1 )$ is Hermite, viz.,
	\begin{align*}
		\gamma^*( \boldsymbol{x}_1 , \boldsymbol{x}^\prime_1 ) &= \left( N \int_{\mathbb{R}^3} \dif \boldsymbol{x}_2 \cdots \int_{\mathbb{R}^3} \dif \boldsymbol{x}_N \Phi^*( \boldsymbol{x}_1 , \boldsymbol{x}_2 , \cdots , \boldsymbol{x}_N ) \Phi( \boldsymbol{x}^\prime_1 , \boldsymbol{x}_2 , \cdots , \boldsymbol{x}_N ) \right)^* \\
		&= N \int_{\mathbb{R}^3} \dif \boldsymbol{x}_2 \cdots \int_{\mathbb{R}^3} \dif \boldsymbol{x}_N \Phi^*( \boldsymbol{x}^\prime_1 , \boldsymbol{x}_2 , \cdots , \boldsymbol{x}_N ) \Phi( \boldsymbol{x}_1 , \boldsymbol{x}_2 , \cdots , \boldsymbol{x}_N ) = \gamma( \boldsymbol{x}^\prime_1 , \boldsymbol{x}_1 ).
	\end{align*}
	Thus,
	\begin{align*}
		\gamma^*_{ji} &= \left( \int_{\mathbb{R}^3} \dif \boldsymbol{x}_1 \int_{\mathbb{R}^3} \dif \boldsymbol{x}^\prime_1 \chi^*_j( \boldsymbol{x}_1 ) \gamma( \boldsymbol{x}_1 , \boldsymbol{x}^\prime_1 ) \chi_i( \boldsymbol{x}^\prime_1 ) \right)^* = \int_{\mathbb{R}^3} \dif \boldsymbol{x}_1 \int_{\mathbb{R}^3} \dif \boldsymbol{x}^\prime_1 \chi^*_i( \boldsymbol{x}^\prime_1 ) \gamma^*( \boldsymbol{x}_1 , \boldsymbol{x}^\prime_1 ) \chi_j( \boldsymbol{x}_1 ) \\
		&= \int_{\mathbb{R}^3} \dif \boldsymbol{x}_1 \int_{\mathbb{R}^3} \dif \boldsymbol{x}^\prime_1 \chi^*_i( \boldsymbol{x}^\prime_1 ) \gamma( \boldsymbol{x}^\prime_1 , \boldsymbol{x}_1 ) \chi_j( \boldsymbol{x}_1 ) = \int_{\mathbb{R}^3} \dif \boldsymbol{x}_1 \int_{\mathbb{R}^3} \dif \boldsymbol{x}_1 \chi^*_i( \boldsymbol{x}_1 ) \gamma( \boldsymbol{x}_1 , \boldsymbol{x}^\prime_1 ) \chi_j( \boldsymbol{x}^\prime_1 ) = \gamma_{ij}.
	\end{align*}
	Hence we have proved that $\gamma$ is a Hermitian matrix.
	
	\end{solution}	
	
	% 4.5
	\begin{exercise}
	Show that $\tr{\gamma}=N$.
	\end{exercise}
	
	\begin{solution}
	
	\begin{align*}
		N &= \int_{ \mathbb{R}^3 } \dif \boldsymbol{x}_1 \rho( \boldsymbol{x}_1 ) = \int_{ \mathbb{R}^3 } \dif \boldsymbol{x}_1 \gamma( \boldsymbol{x}_1 , \boldsymbol{x}_1  ) = \int_{ \mathbb{R}^3 } \dif \boldsymbol{x}_1 \sum_{ i=1 }^N \sum_{ j=1 }^N \chi_i( \boldsymbol{x}_1 ) \gamma_{ij} \chi^*_j( \boldsymbol{x}_1 ) \\
		&= \sum_{ i=1 }^N \sum_{ j=1 }^N \gamma_{ij} \int_{ \mathbb{R}^3 } \dif \boldsymbol{x}_1 \chi_i( \boldsymbol{x}_1 )  \chi^*_j( \boldsymbol{x}_1 ) = \sum_{ i=1 }^N \sum_{ j=1 }^N \gamma_{ij} \delta_{ij} = \sum_{ i=1 }^N \gamma_{ii} = \tr\gamma.
	\end{align*}
	
	\end{solution}
	
	% 4.6
	\begin{exercise}
	Consider the one-electron operator
	\[
		\mathscr{O}_1 = \sum_{i=1}^N h(i).
	\]
	\begin{enumerate}
	
	\item[a.] Show that
	\[
		\langle \Phi | \mathscr{O}_1 | \Phi \rangle = \int \dif \boldsymbol{x}_1 \left[ h( \boldsymbol{x}_1  ) \gamma( \boldsymbol{x}_1 , \boldsymbol{x}^\prime_1 ) \right]_{ \boldsymbol{x}^\prime_1 = \boldsymbol{x}_1}
	\]
	where the notation $[ \quad ]_{ \boldsymbol{x}^\prime_1 = \boldsymbol{x}_1}$ means that $\boldsymbol{x}^\prime_1$ is set equal to $\boldsymbol{x}_1$ after $h(\boldsymbol{x}_1)$ has operated on $\gamma( \boldsymbol{x}_1 , \boldsymbol{x}^\prime_1 )$.
	
	\item[b.] Show that
	\[
		\langle \Phi | \mathscr{O}_1 | \Phi \rangle = \tr{{\bf h}\gamma}
	\]
	where
	\[
		h_{ij} = \langle i | h | j \rangle = \int \dif \boldsymbol{x}_1 \chi^*_i( \boldsymbol{x}_1 ) h( \boldsymbol{x}_1 ) \chi_j( \boldsymbol{x}_1 ).
	\]
	Thus the expectation value of any one-electron operator can be expressed in terms of the one-matrix.
	\end{enumerate}
	
	\end{exercise}
	
	\begin{solution}
	
	\begin{itemize}
	
	\item[a.] From the definition of $\mathscr{O}_1$, we find that	
	\begin{align*}
		\langle \Phi | \mathscr{O}_1 | \Phi \rangle &= \langle \Phi | \sum_{ i=1 }^N h(i) | \Phi \rangle \\
		&= \sum_{ i=1 }^N \int_{ \mathbb{R}^3 } \dif \boldsymbol{x}_1 \int_{ \mathbb{R}^3 } \dif \boldsymbol{x}_2 \cdots \int_{ \mathbb{R}^3 } \dif \boldsymbol{x}_N \Phi^*( \boldsymbol{x}_1, \boldsymbol{x}_2, \cdots , \boldsymbol{x}_N ) h( \boldsymbol{x}_i ) \Phi( \boldsymbol{x}_1, \boldsymbol{x}_2, \cdots , \boldsymbol{x}_N )
	\end{align*}
	Considering that the different integral variables $\dif \boldsymbol{x}_1$ and $\dif \boldsymbol{x}_i$ ($i \neq 1$) have the same integral range, it is clear that	
	\begin{align*}
		\langle \Phi | \mathscr{O}_1 | \Phi \rangle &= \langle \Phi | \sum_{ i=1 }^N h(i) | \Phi \rangle \\
		&= N \int_{ \mathbb{R}^3 } \dif \boldsymbol{x}_1 \int_{ \mathbb{R}^3 } \dif \boldsymbol{x}_2 \cdots \int_{ \mathbb{R}^3 } \dif \boldsymbol{x}_N \Phi^*( \boldsymbol{x}_1, \boldsymbol{x}_2, \cdots , \boldsymbol{x}_N ) h( \boldsymbol{x}_1 ) \Phi( \boldsymbol{x}_1, \boldsymbol{x}_2, \cdots , \boldsymbol{x}_N ) \\
		&= \int_{ \mathbb{R}^3 } \dif \boldsymbol{x}_1 h( \boldsymbol{x}_1 ) \times N \int_{ \mathbb{R}^3 } \dif \boldsymbol{x}_2 \cdots \int_{ \mathbb{R}^3 } \dif \boldsymbol{x}_N \Phi^*( \boldsymbol{x}_1, \boldsymbol{x}_2, \cdots , \boldsymbol{x}_N ) \Phi( \boldsymbol{x}_1, \boldsymbol{x}_2, \cdots , \boldsymbol{x}_N ) \\
		&= \int_{ \mathbb{R}^3 } \dif \boldsymbol{x}_1 h( \boldsymbol{x}_1 ) \rho( \boldsymbol{x}_1 ) = \int_{ \mathbb{R}^3 } \dif \boldsymbol{x}_1 h( \boldsymbol{x}_1 ) \gamma( \boldsymbol{x}_1 , \boldsymbol{x}_1 ) = \int_{ \mathbb{R}^3 } \dif \boldsymbol{x}_1 \left[ h( \boldsymbol{x}_1 ) \gamma( \boldsymbol{x}_1 , \boldsymbol{x}^\prime_1 ) \right]_{ \boldsymbol{x}^\prime_1 = \boldsymbol{x}_1 }.
	\end{align*}
		
	\item[b.] From the former issue, we know that 	
	\begin{align*}
		\langle \Phi | \mathscr{O}_1 | \Phi \rangle &= \int_{ \mathbb{R}^3 } \dif \boldsymbol{x}_1 h( \boldsymbol{x}_1 ) \gamma( \boldsymbol{x}_1 , \boldsymbol{x}_1 ) = \int_{ \mathbb{R}^3 } \dif \boldsymbol{x}_1 h( \boldsymbol{x}_1 ) \sum_{ i=1 }^N \sum_{ j=1 }^N \chi_i( \boldsymbol{x}_1 ) \gamma_{ij} \chi^*_j( \boldsymbol{x}_1 ) \\
		&= \sum_{ i=1 }^N \sum_{ j=1 }^N \gamma_{ij} \int_{ \mathbb{R}^3 } \dif \boldsymbol{x}_1 \chi_i( \boldsymbol{x}_1 ) h( \boldsymbol{x}_1 ) \chi^*_j( \boldsymbol{x}_1 ) = \sum_{ i=1 }^N \sum_{ j=1 }^N \gamma_{ij} h_{ji} = \tr{\gamma \h} = \tr{\h \gamma}.
	\end{align*}
	The last step uses the conclusion of exercise 1.4(a).
	\end{itemize}		
	
	\end{solution}
	
	% 4.7
	\begin{exercise}
	Recall that in second quantization a one-electron operator is
	\[
		\mathscr{O}_1 = \sum_{ij} \langle i | h | j \rangle a^\dagger_i a_j.
	\]
	\begin{enumerate}
	
	\item[a.] Show that
	\[
		\gamma_{ij} = \langle \Phi | a^\dagger_j a_i | \Phi \rangle.
	\]
	
	\item[b.] Show that the matrix elements of $\gamma^{\HF}$ are given by Eq.(4.40).
	
	\end{enumerate}
	\end{exercise}
	
	\begin{solution}
	
	\begin{itemize}
	
	\item[a.] We can use the conclusion in exercise 4.6(b), viz.,
	\begin{align*}
		\tr{\h\gamma} = \sum_{ij} h_{ij} \gamma_{ji} = \langle \Phi | \mathscr{O}_1 | \Phi \rangle .
	\end{align*}
	With the second quantization form, we find that
	\[
		\sum_{ij} h_{ij} \gamma_{ji} = \langle \Phi | \mathscr{O}_1 | \Phi \rangle = \sum_{ij} \langle \Phi | a^\dagger_i a_j | \Phi_1 \rangle h_{ij} \Leftrightarrow \sum_{ij} \left[ \langle \Phi | a^\dagger_i a_j | \Phi \rangle - \gamma_{ji} \right] h_{ij} = 0.
	\]
	For any system, this equation holds true, which means that terms $h_{ij}$ are linearly independent. Thus,
	\begin{equation}
		\gamma_{ji} = \langle \Phi | a^\dagger_i a_j | \Phi \rangle,
	\end{equation}
	which equals $\gamma_{ij} = \langle \Phi | a^\dagger_j a_i | \Phi \rangle$.
	
	\item[b.] If $i$ belongs to unoccupied, $a_i | \Phi \rangle$ vanishes, and so does $ \langle \Phi | a^\dagger_j$ if $j$ belongs to unoccupied. If the indices $i$ and $j$ are occupied, viz., $i=a$ and $j=b$, with $a^\dagger_a | \Phi \rangle = 0$,
	\[
		\gamma^\HF_{ab} = \langle \Phi | a^\dagger_a a_b | \Phi \rangle = \langle \Phi | ( \delta_{ab} - a_b a^\dagger_a) | \Phi \rangle = \delta_{ab} - \langle \Phi | a_b a^\dagger_a | \Phi \rangle = \delta_{ab} - 0 = \delta_{ab}.
	\]
	Thus, we have proved
	\[
		\gamma^\HF_{ij} = \begin{cases}
   			\delta_{ij}, & \text{iff } i, j \in \text{occupied}, \\
   			0, & \text{otherwise}.
		\end{cases}
	\]
	\end{itemize}		
	
	\end{solution}
	
	% 4.8
	\begin{exercise}
	For the special case of a two-electron system, the use of natural orbitals dramatically reduces the size of the full CI expansion. If $\psi_1$ is the occupied Hartree-Fock spatial orbital and $\psi_r$, $r=2,3,...,K$ are virtual spatial orbitals, the normalized full CI singlet wave function has the form
	\[
		|{}^{1}\Phi_0 \rangle = c_0 |1\bar{1}\rangle + \sum_{r=2}^K c^r_1 | {}^{1} \Psi^r_1 \rangle + \frac{1}{2} \sum_{r=2}^K \sum_{s=2}^K c^{rs}_{11} | {}^{1} \Psi^{rs}_{11} \rangle
	\]
	where the singly and doubly excited spin adapted configurations are defined in Subsection 2.5.2.
	
	\begin{enumerate}
	
	\item[a.] Show that $|{}^{1}\Phi_0 \rangle$	can be cast into the form
	\[
		|{}^{1}\Phi_0 \rangle = \sum_{i=1}^K \sum_{j=1}^K C_{ij} | \psi_i \bar{\psi}_j \rangle
	\]
	where $\C$ is a symmetric $K \times K$ matrix.
	
	\item[b.] Show that
	\[
		\gamma( \boldsymbol{x}_1 , \boldsymbol{x}_1^\prime ) = \sum_{ij} (\C \C^\dagger)_{ij} \left( \psi_i(1) \psi^*_j(1^\prime) + \bar{\psi}_i (1) \bar{\psi}_j^* (1^\prime) \right).
	\]
	
	\item[c.] Let $\U$ be the unitary transformation which diagonalizes $\C$
	\[
		\U^\dagger \C \U = {\bf d}
	\]
	where $({\bf d})_{ij} = d_i \delta_{ij}$. Show that
	\[
		\U^\dagger \C \C^\dagger \U = {\bf d}^2.
	\]
	
	\item[d.] Show that
	\[
		\gamma( \boldsymbol{x}_1 , \boldsymbol{x}_1^\prime ) = \sum_i d^2_i \left( \zeta_i(1)\zeta^*_i(1^\prime) + \bar{\zeta}_i(1) \bar{\zeta}^*_i (1^\prime) \right)
	\]
	where
	\[
		\zeta_i = \sum_k \psi_k U_{ki}.
	\]
	Thus $\U$ diagonalizes the one-matrix, and hence $\zeta_i$ are natural spatial orbitals for the two-electron system.
	
	\item[e.] Finally, since $\C$ is symmetric, $\U$ can be chosen as real. Show that in terms of the natural spatial orbitals, $| {}^{1} \Phi_0 \rangle$ given in part (a) can be rewritten as
	\[
		| {}^{1} \Phi_0 \rangle = \sum_{i=1}^K d_i | \zeta_i \bar{\zeta}_i \rangle
	\]
	and note that this expansion contains only $K$ terms.
	\end{enumerate}
	\end{exercise}
	
	\begin{solution}
	
	\begin{enumerate}
	
	\item[a.] From (2.263) at the page 103 and Table 2.7 at the page 104, we know that	
	\begin{align*}
		| ^1 \Psi^r_1 \rangle &= \frac{ 1 }{ \sqrt{2} } \left( | \Psi^{ \bar{r} }_{ \bar{1} } \rangle + | \Psi^r_1 \rangle \right) = \frac{ 1 }{ \sqrt{2} } \left( | 1 \bar{r} \rangle + | r \bar{1} \rangle \right) , \\
		| ^1 \Psi^{rr}_{11} \rangle &= | \Psi^{r \bar{r}}_{1 \bar{1}} \rangle = | r \bar{r} \rangle , \\
		| ^1 \Psi^{rs}_{11} \rangle &= \frac{ 1 }{ \sqrt{2} } \left( | \Psi^{ r \bar{s} }_{ 1 \bar{1} } \rangle + | \Psi^{s \bar{r} }_{ 1 \bar{1} } \rangle \right) = \frac{ 1 }{ \sqrt{2} } \left( | r \bar{s} \rangle + | s \bar{r} \rangle \right) , \, \forall r \neq s.
	\end{align*}
	
	Thus, 
	\begin{align*}
		| ^1 \Phi_0 \rangle = c_0 | 1 \bar{1} \rangle + \sum_{r=2}^K  \frac{ c^r_1 }{ \sqrt{2} } \left( | 1 \bar{r} \rangle + | r \bar{1} \rangle \right) + \frac{1}{2} \sum_{r=2}^K c^{rr}_{11} | r \bar{r} \rangle + \sum_{r=2}^K \sum_{ \substack{ s=2 \\ r > s } }^K c^{rs}_{11} \frac{ 1 }{ \sqrt{2} } \left( | r \bar{s} \rangle + | s \bar{r} \rangle \right).
	\end{align*}
	From this equation, we find that the coefficients are
	\begin{align*}
		C_{11} &= c_0 , \\
		C_{r1} &= C_{1r} = \frac{ c^r_1 }{ \sqrt{2} }, \\
		C_{rr} &= \frac{ c^{rr}_{11} }{ 2 }, \\
		C_{rs} &= C_{sr} = \frac{ c^{rs}_{11} + c^{sr}_{11} }{ 2 } , \, \forall r \neq s ,
	\end{align*}
	which equals
	\begin{equation}
		|{}^{1}\Phi_0 \rangle = \sum_{i=1}^K \sum_{j=1}^K C_{ij} | \psi_i \bar{\psi}_j \rangle,
	\end{equation}		
	where
	\begin{equation}
		C_{ij} = C_{ji}, \, \forall i, j \in \{ 1,2,\cdots, K \}.
	\end{equation}
	In other words, $\C$ is a symmetric $K \times K$ matrix.
	
	\item[b.] Note that there are two electrons in this system,
	\begin{align*}
		\gamma( \boldsymbol{x}_1 , \boldsymbol{x}^\prime_1 ) &= 2 \int_{ \mathbb{R}^3 } \dif \boldsymbol{x}_2 \Phi( \boldsymbol{x}_1 , \boldsymbol{x}_2 ) \Phi^*( \boldsymbol{x}^\prime_1 , \boldsymbol{x}_2 )  \\
		&= 2 \int_{ \mathbb{R}^3 } \dif \boldsymbol{x}_2 \frac{1}{ \sqrt{2} } \sum_{ i=1 }^K \sum_{ j=1 }^K C_{ij} \left[ \psi_i( \boldsymbol{x}_1 ) \bar{\psi}_j( \boldsymbol{x}_2 ) - \psi_i( \boldsymbol{x}_2 ) \bar{\psi}_j( \boldsymbol{x}_1 ) \right] \\
		&\hspace{10em} \times \frac{1}{ \sqrt{2} } \sum_{ k=1 }^K \sum_{ l=1 }^K C^*_{kl} \left[ \psi^*_k( \boldsymbol{x}^\prime_1 ) \bar{\psi}^*_l( \boldsymbol{x}_2 ) - \psi^*_k( \boldsymbol{x}_2 ) \bar{\psi}^*_l( \boldsymbol{x}^\prime_1 ) \right] \\
		&= \sum_{ i=1 }^K \sum_{ j=1 }^K \sum_{ k=1 }^K \sum_{ l=1 }^K C_{ij} C^*_{kl} \\ 
		&\hspace{2em}\left[ \psi_i( \boldsymbol{x}_1 ) \psi^*_k( \boldsymbol{x}^\prime_1 ) \int_{ \mathbb{R}^3 } \dif \boldsymbol{x}_2 \bar{\psi}_j( \boldsymbol{x}_2 ) \bar{\psi}^*_l( \boldsymbol{x}_2 ) - \psi_i( \boldsymbol{x}_1 ) \bar{\psi}^*_l( \boldsymbol{x}^\prime_1 ) \int_{ \mathbb{R}^3 } \dif \boldsymbol{x}_2 \bar{\psi}_j( \boldsymbol{x}_2 ) \psi^*_k( \boldsymbol{x}_2 ) \right. \\
		&\hspace{2em} \left. - \bar{\psi}_j( \boldsymbol{x}_1 ) \psi^*_k( \boldsymbol{x}^\prime_1 ) \int_{ \mathbb{R}^3 } \dif \boldsymbol{x}_2 \psi_i( \boldsymbol{x}_2 ) \bar{\psi}^*_l( \boldsymbol{x}_2 ) + \bar{\psi}_j( \boldsymbol{x}_1 ) \bar{\psi}^*_l( \boldsymbol{x}^\prime_1 ) \int_{ \mathbb{R}^3 } \dif \boldsymbol{x}_2 \psi_i( \boldsymbol{x}_2 ) \psi^*_k( \boldsymbol{x}_2 ) \right] \\
		&= \sum_{ i=1 }^K \sum_{ j=1 }^K \sum_{ k=1 }^K \sum_{ l=1 }^K C_{ij} C^*_{kl} \left[ \psi_i( \boldsymbol{x}_1 ) \psi^*_k( \boldsymbol{x}^\prime_1 ) \delta_{jl} + \bar{\psi}_j( \boldsymbol{x}_1 ) \bar{\psi}^*_l( \boldsymbol{x}^\prime_1 ) \delta_{ik} \right] \\
		&= \sum_{ i=1 }^K \sum_{ j=1 }^K \sum_{ k=1 }^K C_{ij} C^*_{kj} \psi_i( \boldsymbol{x}_1 ) \psi^*_k( \boldsymbol{x}^\prime_1 ) + \sum_{ i=1 }^K \sum_{ j=1 }^K \sum_{ l=1 }^K C_{ij} C^*_{il} \bar{\psi}_j( \boldsymbol{x}_1 ) \bar{\psi}^*_l( \boldsymbol{x}^\prime_1 ) \\
		&= \sum_{ i=1 }^K \sum_{ j=1 }^K \sum_{ k=1 }^K C_{ij} C^\dagger_{jk} \psi_i( \boldsymbol{x}_1 ) \psi^*_k( \boldsymbol{x}^\prime_1 ) + \sum_{ i=1 }^K \sum_{ j=1 }^K \sum_{ k=1 }^K C_{ji} C^\dagger_{ik} \bar{\psi}_j( \boldsymbol{x}_1 ) \bar{\psi}^*_k( \boldsymbol{x}^\prime_1 ) \\
		&= \sum_{ i=1 }^K \sum_{ j=1 }^K \psi_i( \boldsymbol{x}_1 ) \psi^*_j( \boldsymbol{x}^\prime_1 ) \left( \sum_{ k=1 }^K C_{ik} C^\dagger_{kj} \right)  + \sum_{ i=1 }^K \sum_{ j=1 }^K  \bar{\psi}_i( \boldsymbol{x}_1 ) \bar{\psi}^*_j( \boldsymbol{x}^\prime_1 ) \left( \sum_{ k=1 }^K C_{ik} C^\dagger_{kj} \right)\\
		&= \sum_{ i=1 }^K \sum_{ j=1 }^K (\C \C^\dagger)_{ij} \left( \psi_i( \boldsymbol{x}_1 ) \psi^*_j( \boldsymbol{x}^\prime_1 ) + \bar{\psi}_i( \boldsymbol{x}_1 ) \bar{\psi}^*_j( \boldsymbol{x}^\prime_1 ) \right).
	\end{align*}
	Thus, we have proved that
	\begin{equation}
		\gamma( \boldsymbol{x}_1 , \boldsymbol{x}^\prime_1 ) = \sum_{ i=1 }^K \sum_{ j=1 }^K (\C \C^\dagger)_{ij} \left( \psi_i( \boldsymbol{x}_1 ) \psi^*_j( \boldsymbol{x}^\prime_1 ) + \bar{\psi}_i( \boldsymbol{x}_1 ) \bar{\psi}^*_j( \boldsymbol{x}^\prime_1 ) \right).
	\end{equation}
	
	\item[c.] In fact, all $d_i$ are real as all eigenvalues of a real symmetric matrix are real, thus ${\bf d}^\dagger = {\bf d}$, and
	\begin{equation}
		{\bf d}^2 = {\bf d}{\bf d}^\dagger = \U^\dagger \C \U (\U^\dagger \C \U)^\dagger = \U^\dagger \C \U \U^\dagger \C^\dagger \U = \U^\dagger \C \C^\dagger \U.
	\end{equation}
	
	\item[d.] From the former issue, we know
	\[
		\C \C^\dagger = \U {\bf d}^2 \U^\dagger.
	\]
	Thus, we find that
	\begin{align*}
		\gamma( \boldsymbol{x}_1 , \boldsymbol{x}^\prime_1 ) &= \sum_{ i=1 }^K \sum_{ j=1 }^K (\C \C^\dagger)_{ij} \left( \psi_i( \boldsymbol{x}_1 ) \psi^*_j( \boldsymbol{x}^\prime_1 ) + \bar{\psi}_i( \boldsymbol{x}_1 ) \bar{\psi}^*_j( \boldsymbol{x}^\prime_1 ) \right) \\
		&= \sum_{ i=1 }^K \sum_{ j=1 }^K ( \U {\bf d}^2 \U^\dagger )_{ij} \left( \psi_i( \boldsymbol{x}_1 ) \psi^*_j( \boldsymbol{x}^\prime_1 ) + \bar{\psi}_i( \boldsymbol{x}_1 ) \bar{\psi}^*_j( \boldsymbol{x}^\prime_1 ) \right) \\
		&= \sum_{ i=1 }^K \sum_{ j=1 }^K \sum_{ k=1 }^K \U_{ik} d^2_k \U^\dagger_{kj} \left( \psi_i( \boldsymbol{x}_1 ) \psi^*_j( \boldsymbol{x}^\prime_1 ) + \bar{\psi}_i( \boldsymbol{x}_1 ) \bar{\psi}^*_j( \boldsymbol{x}^\prime_1 ) \right) \\
		&= \sum_{ k=1 }^K d^2_k \left( \sum_{ i=1 }^K \psi_i( \boldsymbol{x}_1 ) \U_{ik} \sum_{ j=1 }^K \psi^*_j( \boldsymbol{x}^\prime_1 ) \U^*_{jk} + \sum_{ i=1 }^K \bar{\psi}_i( \boldsymbol{x}_1 ) \U_{ik} \sum_{ j=1 }^K \bar{\psi}^*_j( \boldsymbol{x}^\prime_1 ) \U^*_{jk} \right).
	\end{align*}
	Therefore, we define
	\[
		\zeta_i = \sum_{k=1}^K \psi_k U_{ki},
	\]
	and we obtain
	\begin{equation}
		\gamma( \boldsymbol{x}_1 , \boldsymbol{x}^\prime_1 ) = \sum_{ k=1 }^K d^2_k \left( \zeta_k(\boldsymbol{x}_1) \zeta^*_k( \boldsymbol{x}^\prime_1 ) + \bar{\zeta}_k( \boldsymbol{x}_1 ) \bar{\zeta}^*_k( \boldsymbol{x}^\prime_1 ) \right) = \sum_{ i=1 }^K d^2_i \left( \zeta_i(\boldsymbol{x}_1) \zeta^*_i( \boldsymbol{x}^\prime_1 ) + \bar{\zeta}_i( \boldsymbol{x}_1 ) \bar{\zeta}^*_i( \boldsymbol{x}^\prime_1 ) \right).
	\end{equation}
	We conclude thay $U$ diagonalizes the one-matrix, and hence $\zeta_i$ are natural spatial orbitals for the two-electron system.
	
	\item[e.] Now we convert $\C$ firstly,	
	\[
		\C = \U {\bf d} \U^\dagger \Leftrightarrow C_{ij} = \sum_{k=1}^K \U_{ik} d_{k} \U^\dagger_{kj} = \sum_{k=1}^K \U_{ik} d_{k} \U^*_{jk} 
	\]
	Thus we arrive at
	\begin{equation}
		|{}^{1} \Phi_0 \rangle = \sum_{i=1}^K \sum_{j=1}^K C_{ij} | \psi_i \bar{\psi}_j \rangle = \sum_{i=1}^K \sum_{j=1}^K \sum_{k=1}^K \U_{ik} d_{k} \U^*_{jk} | \psi_i \bar{\psi}_j \rangle = \sum_{k=1}^K d_{k} | \zeta_k \bar{\zeta}_k \rangle = \sum_{i=1}^K d_{i} | \zeta_i \bar{\zeta}_i \rangle.
	\end{equation}
	We find that this expansion contains only $K$ terms.
	
	\end{enumerate}
	
	\end{solution}
	
	\section{The Multiconfiguration Self-Consistent Field (MCSCF) and \texorpdfstring{\\}- Generalized Valence Bond (GVB) Methods}	
	
	% 4.9
	\begin{exercise}
	Consider the transformation
	\begin{align*}
		u &= \frac{1}{\sqrt{ a^2 + b^2 }} \left( a \psi_A + b \psi_B \right), \\
		v &= \frac{1}{\sqrt{ a^2 + b^2 }} \left( a \psi_A - b \psi_B \right).
	\end{align*}
	\begin{enumerate}
	
	\item[a.] Show that
	\[
		\langle u | u \rangle = \langle v | v \rangle = 1
	\]
	and
	\[
		\langle u | v \rangle \equiv S = \frac{ a^2 - b^2 }{ a^2 + b^2 }.
	\]
	
	\item[b.] Show that $|\Psi_{\rm GVB}\rangle$ in Eq.(4.52) can be rewritten as
	\[
		|\Psi_{\rm GVB}\rangle = \frac{1}{\sqrt{ a^4 + b^4 }} \left[ a^2 \psi_A(1) \psi_A(2) - b^2 \psi_B(1) \psi_B(2) \right] \frac{1}{\sqrt{2}} \left( \alpha(1)\beta(2) - \alpha(2)\beta(1) \right)
	\]
	and conclude that this is identical to $|\Psi_{\rm MCSCF}\rangle$ in Eq.(4.48) if
	\begin{align*}
		c_A &= \frac{ a^2 }{ \sqrt{ a^4 + b^4 } }, \\
		c_B &= -\frac{ b^2 }{ \sqrt{ a^4 + b^4 } }.
	\end{align*}
	\end{enumerate}		
	\end{exercise}
	
	\begin{solution}
	
	\begin{enumerate}
	
	\item[a.] Check $\langle u | u \rangle$, $\langle u | v \rangle$ and $\langle v | v \rangle$.  Note that
	\[
		\langle \psi_A | \psi_A \rangle = \langle \psi_B | \psi_B \rangle = 1, \quad \langle \psi_A | \psi_B \rangle = \langle \psi_B | \psi_A \rangle = 0.
	\]
	My results are as follows.
	\begin{align}
		\langle u | u \rangle &= \frac{1}{ a^2+b^2 } \left[ a^2 \langle \psi_A | \psi_A \rangle + ab \langle \psi_A | \psi_B \rangle + ab \langle \psi_B | \psi_A \rangle + b^2 \langle \psi_B | \psi_B \rangle \right] = 1 , \\
		\langle v | v \rangle &= \frac{1}{ a^2+b^2 } \left[ a^2 \langle \psi_A | \psi_A \rangle - ab \langle \psi_A | \psi_B \rangle - ab \langle \psi_B | \psi_A \rangle + b^2 \langle \psi_B | \psi_B \rangle \right] = 1 , \\
		S \equiv \langle u | v \rangle &= \frac{1}{ a^2+b^2 } \left[ a^2 \langle \psi_A | \psi_A \rangle - ab \langle \psi_A | \psi_B \rangle + ab \langle \psi_B | \psi_A \rangle - b^2 \langle \psi_B | \psi_B \rangle \right] = \frac{ a^2 - b^2 }{ a^2 + b^2 }.
	\end{align}
		
	\item[b.] Substitute these results into (4.52), we find that	
	\begin{align}
		| \Psi_{\rm GVB} \rangle &= \frac{1}{ \sqrt{ 2 \left[ 1 + ( \frac{ a^2 - b^2 }{ a^2 + b^2 } )^2 \right] } } \left[ \frac{1}{ a^2 + b^2 } ( a \psi_A(1) + b \psi_B(1) )( a \psi_A(2) - b \psi_B(2) ) \right. \notag \\
		&\hspace{2em} \left. + \frac{1}{ a^2 + b^2 } ( a \psi_A(2) + b \psi_B(2) )( a \psi_A(1) - b \psi_B(1) ) \right] \frac{1}{ \sqrt{2} } \left[ \alpha(1) \beta(2) - \alpha(2) \beta(1) \right] \notag \\
		&=\frac{ a^2 + b^2 }{ \sqrt{ 2 \left[ ( a^2 + b^2 )^2 + ( a^2 - b^2 )^2 \right] } } \times \frac{1}{ \sqrt{2} \left( a^2 + b^2 \right) } \left[ \alpha(1) \beta(2) - \alpha(2) \beta(1) \right] \notag \\
		&\hspace{2em}\left[ ( a \psi_A(1) + b \psi_B(1) )( a \psi_A(2) - b \psi_B(2) ) + ( a \psi_A(2) + b \psi_B(2) )( a \psi_A(1) - b \psi_B(1) ) \right] \notag \\ 
		&= \frac{ 1 }{ 2 \sqrt{ 2a^4 + 2b^4 } } \left[ 2 a^2 \psi_A(1) \psi_A(2) - 2 b^2 \psi_B(1) \psi_B(2) \right] \left[ \alpha(1) \beta(2) - \alpha(2) \beta(1) \right] \notag \\
		&= \frac{ 1 }{ \sqrt{ a^4 + b^4 } }  \left[ a^2 \psi_A(1) \psi_A(2) - b^2 \psi_B(1) \psi_B(2) \right] \frac{1}{ \sqrt{2} } \left[ \alpha(1) \beta(2) - \alpha(2) \beta(1) \right].
	\end{align}
	Thus, we conclude that this is identical to $|\Psi_{\rm MCSCF}\rangle$ in Eq.(4.48) if
	\[
		c_A = \frac{ a^2 }{ \sqrt{ a^4 + b^4 } } , \quad c_B = -\frac{ b^2 }{ \sqrt{ a^4 + b^4 } }.
	\]
	\end{enumerate}		
		
	\end{solution}
	
	\section{Truncated CI and the Size-Consistency Problem}	
	
	% 4.10
	\begin{exercise}
	Show that $|1_1 \bar{1}_1 2_1 \bar{2}_1 \rangle$ has a zero matrix element with any of the configurations in Eq.(4.55).
	\end{exercise}
	
	\begin{solution}
	
	We only check whether $\langle \Psi_0 | \mathscr{H} | 1_1 \bar{1}_1 2_1 \bar{2}_1 \rangle$, $\langle \Psi^{2_1 \bar{2}_1}_{1_1 \bar{1}_1} | \mathscr{H} | 1_1 \bar{1}_1 2_1 \bar{2}_1 \rangle$ and $\langle \Psi^{2_2 \bar{2}_2}_{1_2 \bar{1}_2} | \mathscr{H} | 1_1 \bar{1}_1 2_1 \bar{2}_1 \rangle$ vanish. That is enough.	
	\begin{align}
		\langle \Psi_0 | \mathscr{H} | 1_1 \bar{1}_1 2_1 \bar{2}_1 \rangle &= \langle 1_1 \bar{1}_1 1_2 \bar{1}_2 | \mathscr{H} | 1_1 \bar{1}_1 2_1 \bar{2}_1 \rangle = \langle 1_2 \bar{1}_2 || 2_1 \bar{2}_1 \rangle = ( 1_2 2_1 | 1_2 2_1 ) = 0, \\
		\langle \Psi^{2_1 \bar{2}_1}_{1_1 \bar{1}_1} | \mathscr{H} | 1_1 \bar{1}_1 2_1 \bar{2}_1 \rangle &= \langle 2_1 \bar{2}_1 1_2 \bar{1}_2 | \mathscr{H} | 1_1 \bar{1}_1 2_1 \bar{2}_1 \rangle = 0 , \\
		\langle \Psi^{2_2 \bar{2}_2}_{1_2 \bar{1}_2} | \mathscr{H} | 1_1 \bar{1}_1 2_1 \bar{2}_1 \rangle &= \langle 1_1 \bar{1}_1 2_2 \bar{2}_2 | \mathscr{H} | 1_1 \bar{1}_1 2_1 \bar{2}_1 \rangle = \langle 2_2 \bar{2}_2 || 2_1 \bar{2}_1 \rangle = ( 2_2 2_1 | 2_2 2_1 ) = 0.
	\end{align}		
	
	\end{solution}
	
	% 4.11
	\begin{exercise}
	Use the integrals for STO-3G $\ce{H2}$ at $R$= 1.4 $\au$, given in Appendix D, to calculate ${}^{N}E_\corr({\rm DCI})/N$ for $N$ = 1, 10, and 100.
	\end{exercise}
	
	\begin{solution}
	
	When $R$ = 1.4. $\au$, we know that
	\begin{center}
	\begin{tabular}{ccc}
		$\varepsilon_1 = -0.5782\,\au$, & $\varepsilon_2 = 0.6703\, \au$, & $J_{11} = 0.6746\,\au$, \\
		$J_{12} = 0.6636\,\au$, & $J_{22} = 0.6975\,\au$, & $K_{12} = 0.1813\,\au$
	\end{tabular}	
	\end{center}
	Similar to exercise 4.3, we obtain $\Delta = 0.78865 \,\au$ Thus, with (4.67), we calculate $\tfrac{ ^N E_\corr({\rm DCI})}{N}$ at various $N$. These results are listed in \Tableref{tab:per_correlation_energy}. We conclude that as $N$ increases, $\tfrac{ ^N E_\corr({\rm DCI})}{N}$ vanishes.

	\begin{minipage}{\textwidth}
    \centering
    \captionsetup{type=table}
    \captionof{table}{The table of $\dfrac{ ^N E_\corr({\rm DCI})}{N}$ to $N$. Here $\Delta = 0.78865 \,\au$}\label{tab:per_correlation_energy}
    \renewcommand{\arraystretch}{2.5}
    \begin{tabular}{c|c|c|c|c|c} \hline
		$N$		&	1	&	10	& 100	& 1000	& 10000 \\ \hline
	$\dfrac{ ^N E_\corr({\rm DCI})}{N}$	& -0.02057 & -0.01864 & -0.01188 & -0.00500 & -0.00174  \\ \hline
	\end{tabular}
	\renewcommand{\arraystretch}{1.0}
	\end{minipage}
		
	\end{solution}
	
	% 4.12
	\begin{exercise}
	Show that full CI is size consistent for a dimer of non-interacting minimal basis $\ce{H_2}$ molecules. A full CI calculation includes, in addition to the excitations in Eq.(4.55), the quadruply excited state $| 2_1 \bar{2}_1 2_2 \bar{2}_2 \rangle = | \Psi^{2_1 \bar{2}_1 2_2 \bar{2}_2}_{1_1 \bar{1}_1 1_2 \bar{1}_2} \rangle$,
	\[
		| \Phi_0 \rangle = | \Psi_0 \rangle + c_1 | 2_1 \bar{2}_1 1_2 \bar{1}_2 \rangle + c_2 | 1_1 \bar{1}_1 2_2 \bar{2}_2 \rangle + c_3 | 2_1 \bar{2}_1 2_2 \bar{2}_2 \rangle.
	\]
	\begin{enumerate}
	
	\item[a.] Show that the full CI matrix equation is
	\[
		\begin{pmatrix}
			0 & K_{12} & K_{12} & 0 \\
			K_{12} & 2\Delta & 0 & K_{12} \\
			K_{12} & 0 & 2\Delta & K_{12} \\
			0 & K_{12} & K_{12} & 4\Delta
		\end{pmatrix} \begin{pmatrix}
			1 \\ c_1 \\ c_2 \\c_3
		\end{pmatrix} = {}^{2} E_{\rm corr} \begin{pmatrix}
			1 \\ c_1 \\ c_2 \\ c_3
		\end{pmatrix}.
	\]	
	Go directly to part (e). If you need help return to part (b).
	
	\item[b.] Show that $c_1 = c_2$ and hence ${}^{2} E_{\rm corr} = 2 K_{12} c_1$.
	
	\item[c.] Show that
	\[
		c_3 = \frac{ {}^{2} E_\corr }{ {}^{2} E_\corr - 4\Delta}.
	\]
	
	\item[d.] Show that
	\[
		c_1 = \frac{ 2 K_{12} }{ {}^{2} E_\corr - 4\Delta }.
	\]
	
	\item[e.] Finally, show that
	\[
		{}^{2} E_\corr = 2\left( \Delta - \sqrt{ \Delta^2 + K^2_{12} } \right)
	\]
	which is indeed exact for the model.
	\end{enumerate}
	
	It is interesting to note that we can express the coefficient of the quadruple excitation ($c_3$) as the product of the coefficients of the double excitations:
	\[
		c_3 = \frac{ {}^{2} E_\corr }{ {}^{2} E_\corr - 4\Delta } = \frac{ 2 K_{12} c_1 }{ {}^{2} E_\corr - 4\Delta } = (c_1)^2
	\]
	where we have used results in parts (b), (c), and (d). This result is not true in general but is a consequence of the fact that the two monomers are independent. However, it suggests that it might be reasonable to {\it approximate} the coefficient of a quadruply excited configuration as a product of the coefficients of the double excitations that combine to give the quadruply excited configuration. This idea plays a central role in the next chapter.
	\end{exercise}
	
	\begin{solution}
	
	\begin{itemize}
		
	\item[a.] Similar to the instance shown at the page 263, we obtain that
	\begin{align*}
		\langle \Psi_0 | \mathscr{H} - E_0 | \Psi_0 \rangle &= 0 , \\
		 \langle \Psi_0 | \mathscr{H} - E_0 | 2_1 \bar{2}_1 1_2 \bar{1}_2 \rangle &= \langle 2_1 \bar{2}_1 1_2 \bar{1}_2 | \mathscr{H} - E_0 | \Psi_0 \rangle = \langle 1_1 \bar{1}_1 || 2_1 \bar{2}_1 \rangle = ( 1_1 2_1 | 1_1 2_1 ) = K_{12} , \\
		 \langle \Psi_0 | \mathscr{H} - E_0 | 1_1 \bar{1}_1 2_2 \bar{2}_2 \rangle &= \langle 1_1 \bar{1}_1 2_2 \bar{2}_2 | \mathscr{H} - E_0 | \Psi_0 \rangle = \langle 1_2 \bar{1}_2 ||  2_2 \bar{2}_2 \rangle = ( 1_2 2_2 | 1_2 2_2 ) = K_{12} , \\
		 \langle \Psi_0 | \mathscr{H} - E_0 | 2_1 \bar{2}_1 2_2 \bar{2}_2 \rangle &= \langle 2_1 \bar{2}_1 2_2 \bar{2}_2 | \mathscr{H} - E_0 | \Psi_0 \rangle = \langle 1_1 \bar{1}_1 1_2 \bar{1}_2 | \mathscr{H} | 2_1 \bar{2}_1 2_2 \bar{2}_2 \rangle = 0, \\
		 \langle 2_1 \bar{2}_1 1_2 \bar{1}_2 | \mathscr{H} - E_0 | 2_1 \bar{2}_1 1_2 \bar{1}_2 \rangle &= ( 2h_{11} + 2h_{22} + J_{11} + J_{12} ) - ( 4 h_{11} + 2J_{11} ) \\
		 &= 2( \varepsilon_1 + \varepsilon_2 ) - J_{11} + J_{22} - 4J_{12} + 2K_{12} - 2( 2\varepsilon_1 - J_{11} ) \\
		 &= 2( \varepsilon_2 - \varepsilon_1 ) + J_{11} + J_{22} + 2K_{12} - 4J_{12} = 2 \Delta , \\
		 \langle 2_1 \bar{2}_1 1_2 \bar{1}_2 | \mathscr{H} - E_0 | 1_1 \bar{1}_1 2_2 \bar{2}_2 \rangle &= \langle 1_1 \bar{1}_1 2_2 \bar{2}_2 | \mathscr{H} - E_0 | 2_1 \bar{2}_1 1_2 \bar{1}_2 \rangle = \langle 1_1 \bar{1}_1 2_2 \bar{2}_2 | \mathscr{H} | 2_1 \bar{2}_1 1_2 \bar{1}_2 \rangle = 0 , \\
		 \langle 2_1 \bar{2}_1 1_2 \bar{1}_2 | \mathscr{H} - E_0 | 2_1 \bar{2}_1 2_2 \bar{2}_2 \rangle &= \langle 2_1 \bar{2}_1 2_2 \bar{2}_2 | \mathscr{H} - E_0 | 2_1 \bar{2}_1 1_2 \bar{1}_2 \rangle = \langle 2_1 \bar{2}_1 2_2 \bar{2}_2 | \mathscr{H} | 2_1 \bar{2}_1 1_2 \bar{1}_2 \rangle \\
		 &= \langle 2_2 \bar{2}_2 || 1_2 \bar{1}_2 \rangle = ( 2_2 1_2 | 2_2 1_2 ) = K_{12}, \\
		 \langle 1_1 \bar{1}_1 2_2 \bar{2}_2 | \mathscr{H} - E_0 | 1_1 \bar{1}_1 2_2 \bar{2}_2 \rangle &= ( 2h_{11} + 2h_{22} + J_{11} + J_{22} ) - ( 4 h_{11} + 2J_{11} ) \\
		 &= 2( \varepsilon_1 + \varepsilon_2 ) - J_{11} + J_{22} - 4J_{12} + 2K_{12} - 2( 2\varepsilon_1 - J_{11} ) \\
		 &= 2( \varepsilon_2 - \varepsilon_1 ) + J_{11} + J_{22} + 2K_{12} - 4J_{12} = 2 \Delta , \\
		 \langle 1_1 \bar{1}_1 2_2 \bar{2}_2 | \mathscr{H} - E_0 | 2_1 \bar{2}_1 2_2 \bar{2}_2 \rangle &= \langle 2_1 \bar{2}_1 2_2 \bar{2}_2 | \mathscr{H} - E_0 | 1_1 \bar{1}_1 2_2 \bar{2}_2 \rangle = \langle 1_1 \bar{1}_1 2_2 \bar{2}_2 | \mathscr{H} | 2_1 \bar{2}_1 2_2 \bar{2}_2 \rangle \\
		 &= \langle 1_1 \bar{1}_1 || 2_1 \bar{2}_1 \rangle = ( 1_1 2_1 | 1_1 2_1 ) = K_{12} , \\
		 \langle 2_1 \bar{2}_1 2_2 \bar{2}_2 | \mathscr{H} - E_0 | 2_1 \bar{2}_1 2_2 \bar{2}_2 \rangle &= (4h_{22} + 2J_{22}) - (4h_{11} + 2J_{11}) \\
		 &= ( 4\varepsilon_2 - 8J_{12} + 4K_{12} + 2J_{12} ) - ( 4\varepsilon_1 - 2J_{11} ) \\
		 &= 4( \varepsilon_2 - \varepsilon_1 ) + 2J_{11} + 2J_{12} + 4K_{12} - 8J_{12} = 4\Delta.
	\end{align*}
	Here, note that the definition of $\Delta$ is located on page 240. Thus, we obtain the full CI matrix equation
	\begin{equation}
		\begin{pmatrix}
			0 		& K_{12} 	& K_{12} 	& 0 		\\
			K_{12} 	& 2\Delta 	& 0 		& K_{12} 	\\
			K_{12} 	& 0			& 2\Delta 	& K_{12}	 \\
			0 		& K_{12} 	& K_{12} 	& 4\Delta
		\end{pmatrix}
		\begin{pmatrix}
			1 \\ c_1 \\ c_2 \\ c_3
		\end{pmatrix} = ^2 E_\corr \begin{pmatrix}
			1 \\ c_1 \\ c_2 \\ c_3
		\end{pmatrix}.
	\end{equation}
	It also equals
	\begin{align*}
		K_{12} c_1 + K_{12} c_2 &= ^2 E_\corr , \\
		K_{12} + 2\Delta c_1 + K_{12} c_3 &= ^2 E_\corr c_1 , \\
		K_{12} + 2\Delta c_2 + K_{12} c_3 &= ^2 E_\corr c_2 , \\
		K_{12} c_1 + K_{12} c_2 + 4\Delta c_3 &= ^2 E_\corr c_3.
	\end{align*}
	
	\item[b.] From the second and the third equations, it is evident that $c_1 = c_2$. From the first equation,
	\begin{equation}
		^2 E_\corr = K_{12} c_1 + K_{12} c_2 = 2 K_{12} c_1.
	\end{equation}
		
	\item[c.] The fourth equation can be substracted by the first one, viz.,
	\[
		4\Delta c_3 = ^2 E_\corr c_3 - ^2 E_\corr,
	\]
	which is equal to
	\begin{equation}
		c_3 = \frac{ ^2 E_\corr }{ ^2 E_\corr - 4 \Delta }.
	\end{equation}
	
	\item[d.] By substituting $c_3$ into the second equation, we find that
	\begin{align*}
		c_1 &= \frac{ K_{12} + K_{12} c_3 }{  ^2 E_\corr - 2\Delta } =  \frac{ K_{12} }{ ^2 E_\corr - 2\Delta } \left( 1 + \frac{ ^2 E_\corr }{ ^2 E_\corr - 4 \Delta } \right) \notag \\
		&= \frac{ K_{12}  \left( 2 ^2 E_\corr - 4 \Delta \right) }{ ( ^2 E_\corr - 2\Delta )( ^2 E_\corr - 4 \Delta ) } = \frac{ 2K_{12} }{ ^2 E_\corr - 4 \Delta }.
	\end{align*}
	
	\item[e.] Thus, by substituting $c_1$ into the first equation, we find that
	\[
		^2 E_\corr = 2 K_{12} c_1 = \frac{ 4K^2_{12} }{ ^2 E_\corr - 4 \Delta } \Rightarrow ( ^2 E_\corr )^2 - 4 \Delta ( ^2 E_\corr ) - 4 K^2_{12} = 0.
	\]
	The discriminant $\Delta_E$ of this quadratic equation is
	\[
		\Delta_E = ( 4 \Delta )^2 - 4 \times 1 \times ( -4 K^2_{12} ) = 16( \Delta^2 + K^2_{12} ) > 0
	\]	
	Thus, the root are
	\[
		E_1 = 2\left( \Delta + \sqrt{ \Delta^2 + K^2_{12} } \right), \quad E_2 = 2 \left( \Delta - \sqrt{ \Delta^2 + K^2_{12} } \right).
	\]
	In current problem, the correlation energy is the lowest root,
	\begin{equation}
		^2 E_\corr = 2 \left( \Delta - \sqrt{ \Delta^2 + K^2_{12} } \right),
	\end{equation}
	which is indeed exact for the model.
	\end{itemize}		
	
	\end{solution}
	
	% 4.13
	\begin{exercise}
	Consider the exact basis set correlation energy of minimal basis $\ce{H2}$ given in Eq.(4.23). Assuming that $K^2_{12}/\Delta^2 \ll 1$ show that
	\[
		{}^{1} E_\corr ({\rm exact}) \approx -\frac{K^2_{12}}{2\Delta}.
	\]
	{\it Hint}: $(1+x)^{1/2} \approx 1 + \frac{1}{2}x$ when $x \ll 1$. This approximate result is the same as the simplest expression for the correlation energy obtained via a form of perturbation theory. Show that by expanding ${}^{N} E_\corr ({\rm DCI})$ in the same way, assuming that $N K^2_{12} / \Delta^2 \ll 1$, one obtains simply $N$ times the above result. This approximation is equivalent to a perturbation result for the correlation energy of a supermolecule, and then form of the result is a reflection of the fact that, in contrast to truncated CI, perturbation theory is size consistent (see Chapter 6).
	\end{exercise}
	
	\begin{solution}
	
	With $(1+x)^{\frac{1}{2}} \approx 1+\frac{1}{2} x$ as $x \ll 1$, when $K^2_{12}/\Delta^2 \ll 1$, we find that
	\[
		^1 E_\corr ({\rm exact}) = \Delta \left( 1 - \sqrt{ 1 + \frac{ K^2_{12} }{ \Delta^2 } } \right) \approx \Delta \left( 1 - \left( 1 + \frac{ K^2_{12} }{ 2\Delta^2 } \right) \right) = -\frac{ K^2_{12} }{ 2\Delta }.
	\]
	The conclusion has been proved.
		
	\end{solution}
	
	% 4.14
	\begin{exercise}
	DCI calculations have become relatively routine; therefore, it is of interest to ask whether one can correct the DCI correlation energy so that it becomes approximately size consistent. A simple prescription for doing this, which is approximately valid {\it only} for relatively small systems, is to write
	\[
		E_\corr = E_\corr( {\rm DCI} ) + \Delta E_{\rm Davidson}
	\]
	where the Davidson correction is given by
	\[
		\Delta E_{\rm Davidson} = ( 1 - c^2_0 ) E_\corr( {\rm DCI} )
	\]
	where $c_0$ is the coefficient of the Hartree-Fock wave function in the {\it normalized} DCI wave function. The Davidson correction can be computed without additional labor since $c_0$ is available in a DCI calculation. Moreover, there is numerical evidence that it leads to an improvement over DCI for relatively small molecules. For example, for $\ce{H2O}$, using the 39-STO basis and the Davidson correction, $\theta_e$ = 104.6$^\circ$, $R_e$ = 1.809 $\au$, $f_{RR}$ = 8.54, and $f_{\theta\theta}$ = 0.80 (see Table 4.7). Also, the ionization potentials of $\ce{N2}$, using the same basis as in Table 4.9 and the Davidson correction, are 0.575 $\au$ ($3\sigma_g$) and 0.617 $\au$ ($1\pi_u$). The purpose of this exercise is to explore the nature of the Davidson correction.
	\begin{enumerate}
	
	\item[a.] For the model of $N$ independent minimal basis $\ce{H2}$ molecules, assume that $N$ is large, yet small enough that $NK^2_{12}/\Delta^2$ is still less than unity. In addition, remember that $\Delta \gg K_{12}$. Using the identity $( 1 + x )^{1/2} \approx 1 + \frac{1}{2} x - \frac{1}{8} x^2 + \cdots$ show that
	\[
		{}^N E_\corr( {\rm DCI} ) = - \frac{ N K^2_{12} }{ 2\Delta } + \frac{ N^2 K^4_{12} }{ 8\Delta^3 } + \cdots.
	\]	
	The term proportional to $N^2$ is spurious and is not present in the similar expansion of ${}^N E_\corr( {\rm exact} )$.
	
	\item[b.] Show that
	\[
		1 - c^2_0 = \frac{ N c^2_1 }{ 1 + N c^2_1 }.
	\]
	
	\item[c.] Show that
	\[
		c_1 = - \frac{ K_{12} }{ 2\Delta } + \cdots.
	\]	
	
	\item[d.] Show that
	\[
		\Delta E_{\rm Davidson} = - \frac{ N^2 K^4_{12} }{ 8 \Delta^3 } + \cdots .
	\]

	Finally, note that the Davidson correction exactly cancels the term proportional to $N^2$ in the expansion of ${}^N E_\corr( {\rm DCI} )$. However, spurious terms containing higher powers of $N$ still remain. For large $N$, the whole analysis breaks down since $N K^2_{12} / \Delta^2$ eventually becomes greater than unity. For $N$ = 1, $E_\corr( {\rm DCI} )$ is exact within the model, yet $\Delta E_{\rm Davidson}$ is not zero.
	
	\item[e.] Numerically investigate the range of validity of the Davidson correction for $N$ independent $\ce{H2}$ molecules. Calculate ${}^N E_\corr( {\rm DCI} )/ {}^N E_\corr( {\rm exact} )$ and $({}^N E_\corr( {\rm DCI} ) + \Delta E_{\rm Davidson} )/ {}^N E_\corr( {\rm exact} )$ for $N$ = 1, $\ldots$, 20 using the values of the two-electron integrals for $R$ = 1.4 a.u. given in Appendix D. You will find that the correlation energy calculated using the Davidson correction is within 1\% of the exact value for $3<N<11$. The DCI correlation energy, on the other hand, errs by 2.5\% for $N$ = 3 and by 10\% for $N$ = 11. For $N$ = 100, the correlation energy with and without the Davidson correction errs by 25\% and 42\%, respectively. {\it Hint}: show that
	\[
		\Delta E_{\rm Davidson} = \frac{ ({}^N E_\corr( {\rm DCI} ))^3 }{ NK^2_{12} + ( {}^N E_\corr( {\rm DCI} ) )^2 }.
	\]
	
	\item[f.] Calculate the Davidson-corrected correlation energy of $\ce{H2O}$ using $E_\corr( {\rm DCI} )$ and $c_0$ given by Saxe et al. (see Further Reading). Compare your result with $E_\corr( {\rm DQCI} )$ and the exact basis set correlation energy.
	\end{enumerate}
	\end{exercise}
	
	\begin{solution}
	
	\begin{itemize}
	
	\item[a.] From (4.67), we find that
	\[
		\frac{ ^N E_\corr ({\rm DCI})}{ \Delta } = 1 - \sqrt{ 1 + \frac{ N K^2_{12} }{ \Delta^2 } } = 1 - \left( 1 + \frac{ N K^2_{12} }{ 2\Delta^2 } - \frac{ N^2 K^4_{12} }{ 8\Delta^4 } + \cdots \right) = - \frac{ N K^2_{12} }{ 2\Delta^2 } + \frac{ N^2 K^4_{12} }{ 8\Delta^4 } + \cdots.
	\]
	
	It equals
	
	\begin{equation}
		^N E_\corr ({\rm DCI}) = - \frac{ N K^2_{12} }{ 2\Delta } + \frac{ N^2 K^4_{12} }{ 8\Delta^3 } + \cdots.
	\end{equation}
	
	\item[b.] From (4.55), we know the norm of $| \Phi_0 \rangle$ of the two independent minimal basis, viz.,
	\[
		\langle \Phi_0 | \Phi_0 \rangle = \left( \langle \Psi_0 | + c^*_1 \langle \Psi^{2_1 \bar{2}_1}_{1_1 \bar{1}_1} | + c^*_2  \langle \Psi^{2_2 \bar{2}_2}_{1_2 \bar{1}_2} | \right) \left( | \Psi_0 \rangle + c_1 | \Psi^{2_1 \bar{2}_1}_{1_1 \bar{1}_1} \rangle + c_2 | \Psi^{2_2 \bar{2}_2}_{1_2 \bar{1}_2} \rangle \right) = 1 + c^2_1 + c^2_2 = 1 + 2c^2_1.
	\]
	Similarly, we can derive that the norm of $| \Phi_0 \rangle$ of $N$ independent minimal basis, viz.,
	\[
		\langle \Phi_0 | \Phi_0 \rangle = 1 + Nc^2_1.
	\]
	Therefore, we can obtain the coefficient of $|\Psi_0\rangle$, 
	\[
		c_0 = \frac{1}{ \sqrt{ \langle \Phi_0 | \Phi_0 \rangle } } = \frac{1}{ \sqrt{ 1 + Nc^2_1 } }
	\]	
	and
	\begin{equation}
		1 - c^2_0 = 1 - \left( \frac{1}{ \sqrt{ 1 + Nc^2_1 } } \right)^2 = \frac{ N c^2_1 }{ 1 + Nc^2_1 }.
	\end{equation}
	
	\item[c.] From (4.64) and (4.67), we obtain
	\begin{equation}
		c_1 = \frac{ K_{12} }{ ^N E_\corr( {\rm DCI}  ) - 2\Delta } = \frac{ K_{12} }{ \Delta - \sqrt{ \Delta^2 + N K^2_{12} } - 2 \Delta } = - \frac{ \frac{ K_{12} }{ \Delta } }{ 1 + \sqrt{ 1 + N \left( \frac{ K_{12} }{ \Delta } \right)^2 } } \approx - \frac{ K_{12} }{ 2 \Delta },
	\end{equation}
	where $N \left( \frac{ K_{12} }{ \Delta } \right)^2 \approx 0$. Note that as $N \left( \frac{ K_{12} }{ \Delta } \right)^2 \approx 0$,
	\[
		1 + Nc^2_1 \approx 1 + N \left( - \frac{ K_{12} }{ 2 \Delta } \right)^2 \approx 1.
	\]
	
	\item[d.] From (4.65), we find that
	\[
		\Delta E_{\rm Davidson} = \frac{ N c^2_1 }{ 1 + Nc^2_1 } N K_{12} c_1 \approx N^2 K_{12} c^3_1 \approx N^2 K_{12} \left( - \frac{ K_{12} }{ 2 \Delta } \right)^3 = - \frac{ N^2 K^4_{12} }{ 8\Delta^3 }.
	\]
	Higher terms are omitted. In other words,
	\begin{equation}
		\Delta E_{\rm Davidson} = - \frac{ N^2 K^4_{12} }{ 8\Delta^3 } + \cdots.
	\end{equation}
	
	\item[e.] Substitute (4.64) into the equation of $\Delta E_{\rm Davidson}$, we find that
	\begin{align}
		\Delta E_{\rm Davidson} &= \frac{ N c^2_1 }{ 1 + Nc^2_1 }    {}^N E_\corr( {\rm DCI} ) \notag \\
		&= \frac{ N \left( \frac{ {}^N E_\corr( {\rm DCI} ) }{ N K_{12} } \right)^2 }{ 1 + N \left( \frac{ {}^N E_\corr( {\rm DCI} ) }{ N K_{12} } \right)^2 } {}^N E_\corr( {\rm DCI} ) = \frac{ ( {}^N E_\corr( {\rm DCI} )^3 }{ N K^2_{12} + ( {}^N E_\corr( {\rm DCI} ) )^2 }.
	\end{align}
	
	A program named \verb!14.cc! in the directory ``scripts" can  be found. It is designed to calculate the correlation energy for $N=1,2,\cdots,19,20,100$. Relevant results are listed in \Tableref{tab:correlation_energy_error}.
	
	\item[f.] The paper by Saxe et al. can be found in the directory ``references". From this paper, we know
	\begin{align*}
		c_0({\rm DCI}) &= 0.97938, \\
		{}^N E_\corr( {\rm DCI} ) &= -0.139340 \, \au \\
		{}^N E_\corr( {\rm DQCI} ) &= -0.145859 \, \au \\
		{}^N E_\corr( {\rm FCI} ) &= -0.148028 \, \au
	\end{align*}
	Thus,
	\[
		\Delta E_{\rm Davidson} = ( 1 - c^2_0({\rm DCI}) ) {}^N E_\corr( {\rm DCI} ) = -0.005687 \, \au
	\]
	and
	\[
		{}^N E_\corr( {\rm DCI} ) + \Delta E_{\rm Davidson} = -0.145027 \, \au.
	\]
	
	Compared to the exact basis set correlation energy ${}^N E_\corr( {\rm FCI} ) = -0.148028 \, \au$, the error of DCI with the Davidson correction is about 2.03\% while the error of QDCI is about 1.47\%.
	
	\begin{minipage}{0.95\linewidth}
    \centering
    \captionsetup{type=table}
    \captionof{table}{The table of the correlation energy to $N$.}\label{tab:correlation_energy_error}
    \begin{tabular}{c|c|c|c|c|c} \hline
		$N$		&	${}^N E_\corr( {\rm DCI} )$	&	error	& ${}^N E_\corr( {\rm DCI} ) + \Delta E_{\rm Davidson}$	& error	& ${}^N E_\corr( {\rm exact} )$ \\ \hline
  1 & -0.020571 &   -0.00\% &  -0.020832 &   1.27\% &   -0.020571 \\
  2 & -0.040632 &   -1.24\% &  -0.041627 &   1.18\% &   -0.041142 \\
  3 & -0.060219 &   -2.42\% &  -0.062355 &   1.04\% &   -0.061713 \\
  4 & -0.079364 &   -3.55\% &  -0.082992 &   0.86\% &   -0.082284 \\
  5 & -0.098095 &   -4.63\% &  -0.103521 &   0.65\% &   -0.102855 \\
  6 & -0.116439 &   -5.66\% &  -0.123929 &   0.41\% &   -0.123426 \\
  7 & -0.134419 &   -6.65\% &  -0.144206 &   0.16\% &   -0.143997 \\
  8 & -0.152055 &   -7.60\% &  -0.164344 &   -0.14\% &  -0.164567 \\
  9 & -0.169367 &   -8.52\% &  -0.184338 &   -0.43\% &  -0.185138 \\
 10 & -0.186371 &   -9.40\% &  -0.204183 &   -0.74\% &  -0.205709 \\
 11 & -0.203084 &   -10.25\% & -0.223877 &   -1.06\% &  -0.226280 \\
 12 & -0.219519 &   -11.07\% & -0.243418 &   -1.39\% &  -0.246851 \\
 13 & -0.235691 &   -11.87\% & -0.262806 &   -1.73\% &  -0.267422 \\
 14 & -0.251612 &   -12.63\% & -0.282041 &   -2.07\% &  -0.287993 \\
 15 & -0.267292 &   -13.38\% & -0.301122 &   -2.41\% &  -0.308564 \\
 16 & -0.282743 &   -14.09\% & -0.320052 &   -2.76\% &  -0.329135 \\
 17 & -0.297975 &   -14.79\% & -0.338830 &   -3.11\% &  -0.349706 \\
 18 & -0.312996 &   -15.47\% & -0.357459 &   -3.46\% &  -0.370277 \\
 19 & -0.327814 &   -16.13\% & -0.375941 &   -3.81\% &  -0.390848 \\
 20 & -0.342439 &   -16.77\% & -0.394276 &   -4.17\% &  -0.411419 \\ 
100 & -1.188453 &   -42.23\% & -1.545648 &   -24.86\% & -2.057093	\\ \hline
	\end{tabular}
	\end{minipage}
	
	\end{itemize}
	
	\end{solution}	
	
	% 4.15
	\begin{exercise}
	The normalized full CI wave function for a minimal basis $\ce{H2}$ molecule is
	\[
		| \Phi_0 \rangle = \frac{1}{ \sqrt{1+c^2} } | 1 \bar{1} \rangle + \frac{c}{ \sqrt{1+c^2} } | 2 \bar{2} \rangle
	\]
	where $c = {}^{1}E_{\rm corr}/K_{12}$. Show that for $N$ independent minimal $\ce{H2}$ molecules, the overlap between the Hartree-Fock wave functions, $| \Psi_0 \rangle$, and the exact {\it normalized} ground state wave function is
	\[
		\langle \Psi_0 | \Phi_0 \rangle = \frac{1}{ (1+c^2)^{\frac{N}{2}} }.
	\]
	Using the values of the two-electron integrals for $R$= 1.4 $\au$, given in Appendix D, calculate $\langle \Psi_0 | \Phi_0 \rangle$ for $N=$1, 10, and 100. Note that this overlap decreases quickly (in fact, exponentially) as $N$ increases. Thus the overlap between the Hartree-Fock and the exact wave functions of the system exponentially approaches zero as the size of the system increases, even though the Hartree-Fock energy is size consistent. {\it Hint}: Because the $N$ independent $\ce{H2}$ molecules are infinitely separated we can, for all intents and purposes, ignore the requirement that the wave function of this system be antisymmetric with respect to the interchange of electrons which belong to different $\ce{H2}$ molecules. Thus we can write
	\[
		| \Phi_0 \rangle \sim \prod_{i=1}^N \left( \frac{1}{ \sqrt{ 1 + c^2 } } | 1_i \bar{1}_i \rangle + \frac{c}{ \sqrt{ 1 + c^2 } } | 2_i \bar{2}_i \rangle \right)
	\]
	and
	\[
		| \Psi_0 \rangle \sim \prod_{i=1}^N | 1_i \bar{1}_i \rangle.
	\]
	\end{exercise}
	
	\begin{solution}
	
	With the wave functions supplied in hints, we find that	
	\begin{equation}
		\langle \Psi_0 | \Phi_0 \rangle = \frac{1}{(1+c^2)^{N/2}} \prod_{ i=1 }^N \prod_{ j=1 }^N \left[ \langle 1_i \bar{1}_i | 1_j \bar{1}_j \rangle + \langle 1_i \bar{1}_i | 2_j \bar{2}_j \rangle \right] = \frac{1}{(1+c^2)^{N/2}} \prod_{ i=1 }^N \prod_{ j=1 }^N \delta_{ij} = \frac{1}{(1+c^2)^{N/2}}.
	\end{equation}
	As $R$ = 1.4 $\au$, $\Delta = 0.78865 \, \au$ and
	\[
		c = \frac{ ^1 E_\corr }{ K_{12} } = \frac{ \Delta - \sqrt{ \Delta^2 + K^2_{12} } }{ K_{12} } = -0.1135.
	\]
	The table of $\langle \Psi_0 | \Phi_0 \rangle$ and $\ln{\langle \Psi_0 | \Phi_0 \rangle}$ to $N$ can be seen below. We conclude that this overlap decreases exponentially as $N$ increases.

	\begin{minipage}{\textwidth}
    \centering
    \captionsetup{type=table}
    \captionof{table}{The table of $\langle \Psi_0 | \Phi_0 \rangle$ and $\ln{\langle \Psi_0 | \Phi_0 \rangle}$ to $N$.}
    \begin{tabular}{c|c|c} \hline
		$N$	& $\langle \Psi_0 | \Phi_0 \rangle$ & $\ln{\langle \Psi_0 | \Phi_0 \rangle}$ \\ \hline
		 1 	& 0.9936 & -0.00642 \\
		 10 & 0.9380 & -0.06401 \\
		100 & 0.5273 & -0.63999 \\
		1000& 0.0017 & -6.37713 \\ \hline
	\end{tabular}
	\end{minipage}

	\end{solution}
	
	
\end{document}
