\documentclass[a4paper]{book}

\usepackage{amsmath}
\usepackage{amssymb}
\usepackage[hypcap=false]{caption}
\usepackage{enumitem}	% 定制enumerate标号
\usepackage{fancyhdr}
\pagestyle{plain} 				% 此处为fancy时有页眉
\usepackage{geometry}
\geometry{
	left=2cm,
	right=2cm,
	top=2cm,
	bottom=2cm,
}
\usepackage{hyperref}
\hypersetup{
    colorlinks=true,            %链接颜色
    linkcolor=blue,             %内部链接
    filecolor=magenta,          %本地文档
    urlcolor=cyan,              %网址链接
}
\usepackage[none]{hyphenat}		% 阻止长单词分在两行
\usepackage{mathrsfs}
\usepackage[version=4]{mhchem}
\usepackage{subcaption}
\usepackage{titlesec}

\RequirePackage[many]{tcolorbox}
\tcbset{
    boxed title style={colback=magenta},
	breakable,
	enhanced,
	sharp corners,
	attach boxed title to top left={yshift=-\tcboxedtitleheight,  yshifttext=-.75\baselineskip},
	boxed title style={boxsep=1pt,sharp corners},
    fonttitle=\bfseries\sffamily,
}

\definecolor{skyblue}{rgb}{0.54, 0.81, 0.94}

\newtcolorbox[auto counter, number within=chapter, number format=\arabic]{exercise}[1][]{
    title={Exercise~\thetcbcounter},
    colframe=skyblue,
    colback=skyblue!12!white,
    boxed title style={colback=skyblue},
    overlay unbroken and first={
        \node[below right,font=\small,color=skyblue,text width=.8\linewidth]
        at (title.north east) {#1};
    }
}

\newtcolorbox[auto counter, number within=chapter, number format=\arabic]{solution}[1][]{
    title={Solution~\thetcbcounter},
    colframe=teal!60!green,
    colback=green!12!white,
    boxed title style={colback=teal!60!green},
    overlay unbroken and first={
        \node[below right,font=\small,color=red,text width=.8\linewidth]
        at (title.north east) {#1};
    }
}

% special new commands for common symbols used in the article
\newcommand\la{\langle}
\newcommand\ra{\rangle}
\newcommand\lr[2]{\langle#1\|#2\rangle}
\newcommand\tr[1]{\mathrm{tr(#1)}}
\newcommand\Tr[3]{#1\mathrm\#2#3}
\newcommand*{\dif}{\mathop{}\!\mathrm{d}}
\renewcommand\det[1]{{\rm det}\left(#1\right)}
\newcommand{\HF}{{\rm HF}}
\newcommand{\corr}{{\rm corr}}
\newcommand{\au}{{\rm a.u.}}

\newcommand{\A}{{\bf A}}
\newcommand{\B}{{\bf B}}
\newcommand{\C}{{\bf C}}
\newcommand{\I}{{\bf 1}}
\newcommand{\U}{{\bf U}}
\newcommand{\Op}{{\bf O}}
\newcommand{\h}{{\bf h}}


\titleformat{\chapter}[display]
  {\bfseries\Large}
  {\filright\MakeUppercase{\chaptertitlename} \Huge\thechapter}
  {1ex}
  {\titlerule\vspace{1ex}\filleft}
  [\vspace{1ex}\titlerule]
  
\allowdisplaybreaks

\begin{document}

	\stepcounter{chapter}\stepcounter{chapter}\stepcounter{chapter}

	\chapter{Configuration Interaction}
	
	\section{Multiconfigurational Wave Functions and the Structure of the Full CI Matrix}
	
	\subsection{Intermediate Normalization and an Expression for the Correlation Energy}
	
	% 4.1
	\begin{exercise}
	Obtain Eq.(4.12) from Eq.(4.11). It will prove convenient to use unrestricted summations.
	\end{exercise}
	
	\begin{solution}
	
	Note that the index $r$ must be included in the set $\{ t, u, v \}$ and the index $a$ must be included in the set $\{ c,d,e\}$ for a matrix element of $\langle \Psi^r_a | \mathscr{H} | \Psi^{tuv}_{cde} \rangle$. Therefore, we find that
	\begin{align*}
		&\hspace{1.4em}\sum_{ \substack{ c<d<e \\ t<u<v } } \langle \Psi^r_a | \mathscr{H} | \Psi^{tuv}_{cde} \rangle c^{tuv}_{cde} = \frac{1}{\left(3!\right)^2} \sum_{ \substack{ cde \\ tuv} } \langle \Psi^r_a | \mathscr{H} | \Psi^{tuv}_{cde} \rangle c^{tuv}_{cde} \\
		&= \frac{1}{\left(3!\right)^2}  \left[ \sum_{ \substack{ de \\ uv } } \langle \Psi^r_a | \mathscr{H} | \Psi^{ruv}_{ade} \rangle c^{ruv}_{ade} + \sum_{ \substack{ de \\ tv } } \langle \Psi^r_a | \mathscr{H} | \Psi^{trv}_{ade} \rangle c^{trv}_{ade} + \sum_{ \substack{ de \\ tu } } \langle \Psi^r_a | \mathscr{H} | \Psi^{tur}_{ade} \rangle c^{tur}_{ade} \right. \\
		&\hspace{ 6em } \left. + \sum_{ \substack{ ce \\ uv } } \langle \Psi^r_a | \mathscr{H} | \Psi^{ruv}_{cae} \rangle c^{ruv}_{cae} + \sum_{ \substack{ ce \\ tv } } \langle \Psi^r_a | \mathscr{H} | \Psi^{trv}_{cae} \rangle c^{trv}_{cae} + \sum_{ \substack{ ce \\ tu } } \langle \Psi^r_a | \mathscr{H} | \Psi^{tur}_{cae} \rangle c^{tur}_{cae} \right. \\
		&\hspace{ 6em } \left. + \sum_{ \substack{ cd \\ uv } } \langle \Psi^r_a | \mathscr{H} | \Psi^{ruv}_{cda} \rangle c^{ruv}_{cda} + \sum_{ \substack{ cd \\ tv } } \langle \Psi^r_a | \mathscr{H} | \Psi^{trv}_{cda} \rangle c^{trv}_{cda} + \sum_{ \substack{ cd \\ tu } } \langle \Psi^r_a | \mathscr{H} | \Psi^{tur}_{cda} \rangle c^{tur}_{cda} \right] .
	\end{align*}
	
	Then, these dummy indices should be converted into the same one, viz.,
	\begin{align*}
		&\hspace{1.4em} \sum_{ \substack{ c<d<e \\ t<u<v } } \langle \Psi^r_a | \mathscr{H} | \Psi^{tuv}_{cde} \rangle c^{tuv}_{cde} \\
		&= \frac{1}{\left(3!\right)^2}  \left[ \sum_{ \substack{ cd \\ tu } } \langle \Psi^r_a | \mathscr{H} | \Psi^{rtu}_{acd} \rangle c^{rtu}_{acd} + \sum_{ \substack{ cd \\ tu } } \langle \Psi^r_a | \mathscr{H} | \Psi^{tru}_{acd} \rangle c^{tru}_{acd} + \sum_{ \substack{ cd \\ tu } } \langle \Psi^r_a | \mathscr{H} | \Psi^{tur}_{acd} \rangle c^{tur}_{acd} \right. \\
		&\hspace{ 6em } + \sum_{ \substack{ cd \\ tu } } \langle \Psi^r_a | \mathscr{H} | \Psi^{rtu}_{cad} \rangle c^{rtu}_{cad} + \sum_{ \substack{ cd \\ tu } } \langle \Psi^r_a | \mathscr{H} | \Psi^{tru}_{cad} \rangle c^{tru}_{cad} + \sum_{ \substack{ cd \\ tu } } \langle \Psi^r_a | \mathscr{H} | \Psi^{tur}_{cad} \rangle c^{tur}_{cad} \\
		&\hspace{ 6em } \left. + \sum_{ \substack{ cd \\ tu } } \langle \Psi^r_a | \mathscr{H} | \Psi^{rtu}_{cda} \rangle c^{rtu}_{cda} + \sum_{ \substack{ cd \\ tu } } \langle \Psi^r_a | \mathscr{H} | \Psi^{tru}_{cda} \rangle c^{tru}_{cda} + \sum_{ \substack{ cd \\ tu } } \langle \Psi^r_a | \mathscr{H} | \Psi^{tur}_{cda} \rangle c^{tur}_{cda} \right] \\
		&= \frac{1}{\left(3!\right)^2} \times 9 \sum_{ \substack{ cd \\ tu } } \langle \Psi^r_a | \mathscr{H} | \Psi^{rtu}_{acd} \rangle c^{rtu}_{acd} = \frac{1}{\left(2!\right)^2} \sum_{ \substack{ cd \\ tu } } \langle \Psi^r_a | \mathscr{H} | \Psi^{rtu}_{acd} \rangle c^{rtu}_{acd} = \sum_{ \substack{ c<d \\ t<u } } \langle \Psi^r_a | \mathscr{H} | \Psi^{rtu}_{acd} \rangle c^{rtu}_{acd}.
	\end{align*}
	Thus, we have proved that
	\begin{equation}
		\sum_{ \substack{ c<d<e \\ t<u<v } } \langle \Psi^r_a | \mathscr{H} | \Psi^{tuv}_{cde} \rangle c^{tuv}_{cde} = \sum_{ \substack{ c<d \\ t<u } } \langle \Psi^r_a | \mathscr{H} | \Psi^{rtu}_{acd} \rangle c^{rtu}_{acd}.
	\end{equation}
	
	With this equation, it is clear that (4.12) can be obtained from (4.11).
	
	\end{solution}
	
	% 4.2
	\begin{exercise}
	Using the secular determinant approach show that the lowest eigenvalue of the matrix
	\[
		\begin{pmatrix}
			0 & K_{12} \\ K_{12} & 2\Delta
		\end{pmatrix}
	\]
	is given by Eq.(4.23).
	\end{exercise}
	
	\begin{solution}
	
	The introduction of the secular determinant approach is demonstrated in the page 18. The matrix in the exercise 4.2 is denoted as $H$, then
	\[
		\det{H - \varepsilon I} = \begin{vmatrix}
		- \varepsilon & K_{12} \\ K_{12} & 2\Delta - \varepsilon
		\end{vmatrix} = \varepsilon^2 - 2 \Delta \varepsilon -K^2_{12} = 0 ,
	\]	
	The discriminant $\Delta_E$ of this quadratic equation is
	\[
		\Delta_E = 4 \Delta^2 - 4 \times ( -K^2_{12} ) = 4( \Delta^2 + K^2_{12} )
	\]	
	Thus, the root are
	\[
		E_1 = \Delta + \sqrt{ \Delta^2 + K^2_{12} }, \quad E_2 = \Delta - \sqrt{ \Delta^2 + K^2_{12} }.
	\]	
	
	Therefore, the lowest root is the correlation energy, viz.,
	\begin{equation}
		E_\corr = \Delta - \sqrt{ \Delta^2 + K^2_{12} }.
	\end{equation}
	
	\end{solution}
	
	% 4.3
	\begin{exercise}
	Calculate the coefficient of the double excitation ($c$) in the intermediate normalized CI wave function at $R$ = 1.4 $\au$, using the STO-3G integrals given in Appendix D. Show analytically that as $R \rightarrow \infty$, $c \rightarrow -1$, and hence that at large distances the Hartree-Fock ground state and the doubly excited configuration have equal weight in the CI ground state. Finally, show that the CI wave function, when normalized to unity, becomes (at $R=\infty$)
	\[
		\frac{1}{\sqrt{2}} \left( | \phi_1 \bar{\phi}_2 \rangle + | \phi_2 \bar{\phi}_1 \rangle \right)
	\]
	where $\phi_1$ and $\phi_2$ are atomic orbitals on centers one and two, respectively.
	\end{exercise}
	
	\begin{solution}
	
	When $R$ = 1.4. $\au$, we know that
	\begin{center}
	\begin{tabular}{ccc}
		$\varepsilon_1 = -0.5782\,\au$, & $\varepsilon_2 = 0.6703\, \au$, & $J_{11} = 0.6746\,\au$, \\
		$J_{12} = 0.6636\,\au$, & $J_{22} = 0.6975\,\au$, & $K_{12} = 0.1813\,\au$
	\end{tabular}
	\end{center}
	
	Firstly, with (4.20), we calculate $2\Delta$ at $R$ = 1.4. $\au$, viz.,
	\[
		2\Delta = \left[ 2( \varepsilon_2 - \varepsilon_1 ) + J_{11} + J_{22} - 4J_{12} + 2K_{12} \right] = 1.5773 \, \au
	\]
	In other words, $\Delta = 0.78865 \, \au$ Thus, the correlation energy $E_\corr$ at $R$ = 1.4. $\au$ is
	\[
		E_\corr = \Delta - \sqrt{ \Delta^2 + K^2_{12} } = -0.02057 \, \au.
	\]
	Therefore,
	\begin{equation}
		c = \frac{ K_{12} }{ E_\corr - 2\Delta } =  \frac{ 0.1813 \, \au }{ -0.02057 \, \au. - 1.5773 \, \au. }  \approx - 0.1135.
	\end{equation}
	
	Indeed, we can find that
	\[
		\Delta = \varepsilon_2 - \varepsilon_1 + \frac{1}{2} J_{11} + \frac{1}{2} J_{22} - 2J_{12} + K_{12} = h_{22} - h_{11} - \frac{1}{2} J_{11} + \frac{1}{2} J_{12}.
	\]
	It is clear that	
	\[
		\lim_{ R \rightarrow \infty} \Delta = \lim_{ R \rightarrow \infty} \left[ h_{22} - h_{11} + \frac{1}{2} J_{22} - \frac{1}{2} J_{11} \right] = E(\ce{H}) - E(\ce{H}) + \frac{1}{4} ( \phi_1 \phi_1 | \phi_1 \phi_1 ) - \frac{1}{4} ( \phi_1 \phi_1 | \phi_1 \phi_1 ) = 0.
	\]
	Thus,
	\begin{align*}
		\lim_{ R \rightarrow \infty } c &= \lim_{ R \rightarrow \infty } \frac{ K_{12} }{ E_\corr - 2\Delta } = \lim_{ R \rightarrow \infty } \frac{ K_{12} }{ \Delta - \sqrt{ \Delta^2 + K^2_{12} } - 2\Delta } = \lim_{ R \rightarrow \infty } \frac{ - K_{12} }{ \Delta + \sqrt{ \Delta^2 + K^2_{12} } } \\
		&= - \lim_{ \Delta \rightarrow 0 } \frac{ 1 }{ \frac{ \Delta }{ K_{12} } + \sqrt{ 1 + \left( \frac{ \Delta }{ K_{12} } \right)^2 } } = - \lim_{ x \rightarrow 0 } \frac{ 1 }{ x + \sqrt{ 1 + x^2 } } = -1.
	\end{align*}
	This conclusion means that at large distances the Hartree-Fock ground state $\Psi_0$ and the doubly excited configuration $\Psi^{2 \bar{2}}_{1 \bar{1}}$ have equal weight in the CI ground state $\Phi$, viz.,
	\[
		\lim_{ R \rightarrow \infty } | \Phi \rangle = | \Psi_0 \rangle - | \Psi^{2 \bar{2}}_{1 \bar{1}} \rangle = | \psi_1 \bar{\psi}_1 \rangle - | \psi_2 \bar{\psi}_2 \rangle.
	\]
	Note that as $R \rightarrow \infty$, from (3.236) and (3.237), we find that
	\[
		\lim_{ R \rightarrow \infty } \psi_1 = \frac{1}{ \sqrt{2} } ( \phi_1 + \phi_2 ) , \quad \lim_{ R \rightarrow \infty } \psi_2 = \frac{1}{ \sqrt{2} } ( \phi_1 - \phi_2 ).
	\]
	Thus,
	\begin{align*}
		\lim_{ R \rightarrow \infty } | \psi_1 \bar{\psi}_1 \rangle &= \frac{1}{2} | ( \phi_1 + \phi_2 ) ( \bar{\phi}_1 + \bar{\phi}_2 ) \rangle = \frac{1}{2} \left( | \phi_1 \bar{\phi}_1 \rangle + | \phi_1 \bar{\phi}_2 \rangle + | \phi_2 \bar{\phi}_1 \rangle + | \phi_2 \bar{\phi}_2 \rangle \right), \\
		\lim_{ R \rightarrow \infty } | \psi_2 \bar{\psi}_2 \rangle &= \frac{1}{ 2 } | ( \phi_1 - \phi_2 ) ( \bar{\phi}_1 - \bar{\phi}_2 ) \rangle = \frac{1}{2} \left( | \phi_1 \bar{\phi}_1 \rangle - | \phi_1 \bar{\phi}_2 \rangle - | \phi_2 \bar{\phi}_1 \rangle + | \phi_2 \bar{\phi}_2 \rangle \right),
	\end{align*}
	and then	
	\[
		\lim_{ R \rightarrow \infty } | \Phi \rangle = \lim_{ R \rightarrow \infty } | \psi_1 \bar{\psi}_1 \rangle - \lim_{ R \rightarrow \infty } | \psi_2 \bar{\psi}_2 \rangle = | \phi_1 \bar{\phi}_2 \rangle + | \phi_2 \bar{\phi}_1 \rangle
	\]
	Thus, at $R = \infty$, the normalized CI wave function is
	\begin{equation}
		\lim_{R \rightarrow \infty} | \Phi \rangle = \lim_{R \rightarrow \infty} \frac{1}{ \langle \Phi_0 | \Phi_0 \rangle } | \Phi_0 \rangle = \frac{1}{ \sqrt{2} } \left( | \phi_1 \bar{\phi}_2 \rangle + | \phi_2 \bar{\phi}_1 \rangle \right) .
	\end{equation}
	
	We have proved two conclusions at $R = \infty$, the equal weight of the Hartree-Fock ground state $\Psi_0$ and the doubly excited configuration $\Psi^{2 \bar{2}}_{1 \bar{1}}$, and the form of normalized CI wave function.
		
	\end{solution}
	
	\section{Doubly Excited CI}
	
	\section{Some Illustrative Calculations}
	
	\section{Natural Orbitals and the One-Particle Reduced Density Matrix}
	
	% 4.4
	\begin{exercise}
	Show that $\gamma$ is a Hermitian matrix.
	\end{exercise}
	
	\begin{solution}
	Firstly, we find that $\gamma( \boldsymbol{x}_1 , \boldsymbol{x}^\prime_1 )$ is Hermite, viz.,
	\begin{align*}
		\gamma^*( \boldsymbol{x}_1 , \boldsymbol{x}^\prime_1 ) &= \left( N \int_{\mathbb{R}^3} \dif \boldsymbol{x}_2 \cdots \int_{\mathbb{R}^3} \dif \boldsymbol{x}_N \Phi^*( \boldsymbol{x}_1 , \boldsymbol{x}_2 , \cdots , \boldsymbol{x}_N ) \Phi( \boldsymbol{x}^\prime_1 , \boldsymbol{x}_2 , \cdots , \boldsymbol{x}_N ) \right)^* \\
		&= N \int_{\mathbb{R}^3} \dif \boldsymbol{x}_2 \cdots \int_{\mathbb{R}^3} \dif \boldsymbol{x}_N \Phi^*( \boldsymbol{x}^\prime_1 , \boldsymbol{x}_2 , \cdots , \boldsymbol{x}_N ) \Phi( \boldsymbol{x}_1 , \boldsymbol{x}_2 , \cdots , \boldsymbol{x}_N ) = \gamma( \boldsymbol{x}^\prime_1 , \boldsymbol{x}_1 ).
	\end{align*}
	Thus,
	\begin{align*}
		\gamma^*_{ji} &= \left( \int_{\mathbb{R}^3} \dif \boldsymbol{x}_1 \int_{\mathbb{R}^3} \dif \boldsymbol{x}^\prime_1 \chi^*_j( \boldsymbol{x}_1 ) \gamma( \boldsymbol{x}_1 , \boldsymbol{x}^\prime_1 ) \chi_i( \boldsymbol{x}^\prime_1 ) \right)^* = \int_{\mathbb{R}^3} \dif \boldsymbol{x}_1 \int_{\mathbb{R}^3} \dif \boldsymbol{x}^\prime_1 \chi^*_i( \boldsymbol{x}^\prime_1 ) \gamma^*( \boldsymbol{x}_1 , \boldsymbol{x}^\prime_1 ) \chi_j( \boldsymbol{x}_1 ) \\
		&= \int_{\mathbb{R}^3} \dif \boldsymbol{x}_1 \int_{\mathbb{R}^3} \dif \boldsymbol{x}^\prime_1 \chi^*_i( \boldsymbol{x}^\prime_1 ) \gamma( \boldsymbol{x}^\prime_1 , \boldsymbol{x}_1 ) \chi_j( \boldsymbol{x}_1 ) = \int_{\mathbb{R}^3} \dif \boldsymbol{x}_1 \int_{\mathbb{R}^3} \dif \boldsymbol{x}_1 \chi^*_i( \boldsymbol{x}_1 ) \gamma( \boldsymbol{x}_1 , \boldsymbol{x}^\prime_1 ) \chi_j( \boldsymbol{x}^\prime_1 ) = \gamma_{ij}.
	\end{align*}
	Hence we have proved that $\gamma$ is a Hermitian matrix.
	
	\end{solution}	
	
	% 4.5
	\begin{exercise}
	Show that $\tr{\gamma}=N$.
	\end{exercise}
	
	\begin{solution}
	
	\begin{align*}
		N &= \int_{ \mathbb{R}^3 } \dif \boldsymbol{x}_1 \rho( \boldsymbol{x}_1 ) = \int_{ \mathbb{R}^3 } \dif \boldsymbol{x}_1 \gamma( \boldsymbol{x}_1 , \boldsymbol{x}_1  ) = \int_{ \mathbb{R}^3 } \dif \boldsymbol{x}_1 \sum_{ i=1 }^N \sum_{ j=1 }^N \chi_i( \boldsymbol{x}_1 ) \gamma_{ij} \chi^*_j( \boldsymbol{x}_1 ) \\
		&= \sum_{ i=1 }^N \sum_{ j=1 }^N \gamma_{ij} \int_{ \mathbb{R}^3 } \dif \boldsymbol{x}_1 \chi_i( \boldsymbol{x}_1 )  \chi^*_j( \boldsymbol{x}_1 ) = \sum_{ i=1 }^N \sum_{ j=1 }^N \gamma_{ij} \delta_{ij} = \sum_{ i=1 }^N \gamma_{ii} = \tr\gamma.
	\end{align*}
	
	\end{solution}
	
	% 4.6
	\begin{exercise}
	Consider the one-electron operator
	\[
		\mathscr{O}_1 = \sum_{i=1}^N h(i).
	\]
	\begin{enumerate}
	
	\item[a.] Show that
	\[
		\langle \Phi | \mathscr{O}_1 | \Phi \rangle = \int \dif \boldsymbol{x}_1 \left[ h( \boldsymbol{x}_1  ) \gamma( \boldsymbol{x}_1 , \boldsymbol{x}^\prime_1 ) \right]_{ \boldsymbol{x}^\prime_1 = \boldsymbol{x}_1}
	\]
	where the notation $[ \quad ]_{ \boldsymbol{x}^\prime_1 = \boldsymbol{x}_1}$ means that $\boldsymbol{x}^\prime_1$ is set equal to $\boldsymbol{x}_1$ after $h(\boldsymbol{x}_1)$ has operated on $\gamma( \boldsymbol{x}_1 , \boldsymbol{x}^\prime_1 )$.
	
	\item[b.] Show that
	\[
		\langle \Phi | \mathscr{O}_1 | \Phi \rangle = \tr{{\bf h}\gamma}
	\]
	where
	\[
		h_{ij} = \langle i | h | j \rangle = \int \dif \boldsymbol{x}_1 \chi^*_i( \boldsymbol{x}_1 ) h( \boldsymbol{x}_1 ) \chi_j( \boldsymbol{x}_1 ).
	\]
	Thus the expectation value of any one-electron operator can be expressed in terms of the one-matrix.
	\end{enumerate}
	
	\end{exercise}
	
	\begin{solution}
	
	\begin{itemize}
	
	\item[a.] From the definition of $\mathscr{O}_1$, we find that	
	\begin{align*}
		\langle \Phi | \mathscr{O}_1 | \Phi \rangle &= \langle \Phi | \sum_{ i=1 }^N h(i) | \Phi \rangle \\
		&= \sum_{ i=1 }^N \int_{ \mathbb{R}^3 } \dif \boldsymbol{x}_1 \int_{ \mathbb{R}^3 } \dif \boldsymbol{x}_2 \cdots \int_{ \mathbb{R}^3 } \dif \boldsymbol{x}_N \Phi^*( \boldsymbol{x}_1, \boldsymbol{x}_2, \cdots , \boldsymbol{x}_N ) h( \boldsymbol{x}_i ) \Phi( \boldsymbol{x}_1, \boldsymbol{x}_2, \cdots , \boldsymbol{x}_N )
	\end{align*}
	Considering that the different integral variables $\dif \boldsymbol{x}_1$ and $\dif \boldsymbol{x}_i$ ($i \neq 1$) have the same integral range, it is clear that	
	\begin{align*}
		\langle \Phi | \mathscr{O}_1 | \Phi \rangle &= \langle \Phi | \sum_{ i=1 }^N h(i) | \Phi \rangle \\
		&= N \int_{ \mathbb{R}^3 } \dif \boldsymbol{x}_1 \int_{ \mathbb{R}^3 } \dif \boldsymbol{x}_2 \cdots \int_{ \mathbb{R}^3 } \dif \boldsymbol{x}_N \Phi^*( \boldsymbol{x}_1, \boldsymbol{x}_2, \cdots , \boldsymbol{x}_N ) h( \boldsymbol{x}_1 ) \Phi( \boldsymbol{x}_1, \boldsymbol{x}_2, \cdots , \boldsymbol{x}_N ) \\
		&= \int_{ \mathbb{R}^3 } \dif \boldsymbol{x}_1 h( \boldsymbol{x}_1 ) \times N \int_{ \mathbb{R}^3 } \dif \boldsymbol{x}_2 \cdots \int_{ \mathbb{R}^3 } \dif \boldsymbol{x}_N \Phi^*( \boldsymbol{x}_1, \boldsymbol{x}_2, \cdots , \boldsymbol{x}_N ) \Phi( \boldsymbol{x}_1, \boldsymbol{x}_2, \cdots , \boldsymbol{x}_N ) \\
		&= \int_{ \mathbb{R}^3 } \dif \boldsymbol{x}_1 h( \boldsymbol{x}_1 ) \rho( \boldsymbol{x}_1 ) = \int_{ \mathbb{R}^3 } \dif \boldsymbol{x}_1 h( \boldsymbol{x}_1 ) \gamma( \boldsymbol{x}_1 , \boldsymbol{x}_1 ) = \int_{ \mathbb{R}^3 } \dif \boldsymbol{x}_1 \left[ h( \boldsymbol{x}_1 ) \gamma( \boldsymbol{x}_1 , \boldsymbol{x}^\prime_1 ) \right]_{ \boldsymbol{x}^\prime_1 = \boldsymbol{x}_1 }.
	\end{align*}
		
	\item[b.] From the former issue, we know that 	
	\begin{align*}
		\langle \Phi | \mathscr{O}_1 | \Phi \rangle &= \int_{ \mathbb{R}^3 } \dif \boldsymbol{x}_1 h( \boldsymbol{x}_1 ) \gamma( \boldsymbol{x}_1 , \boldsymbol{x}_1 ) = \int_{ \mathbb{R}^3 } \dif \boldsymbol{x}_1 h( \boldsymbol{x}_1 ) \sum_{ i=1 }^N \sum_{ j=1 }^N \chi_i( \boldsymbol{x}_1 ) \gamma_{ij} \chi^*_j( \boldsymbol{x}_1 ) \\
		&= \sum_{ i=1 }^N \sum_{ j=1 }^N \gamma_{ij} \int_{ \mathbb{R}^3 } \dif \boldsymbol{x}_1 \chi_i( \boldsymbol{x}_1 ) h( \boldsymbol{x}_1 ) \chi^*_j( \boldsymbol{x}_1 ) = \sum_{ i=1 }^N \sum_{ j=1 }^N \gamma_{ij} h_{ji} = \tr{\gamma \h} = \tr{\h \gamma}.
	\end{align*}
	The last step uses the conclusion of exercise 1.4(a).
	\end{itemize}		
	
	\end{solution}
	
	% 4.7
	\begin{exercise}
	Recall that in second quantization a one-electron operator is
	\[
		\mathscr{O}_1 = \sum_{ij} \langle i | h | j \rangle a^\dagger_i a_j.
	\]
	\begin{enumerate}
	
	\item[a.] Show that
	\[
		\gamma_{ij} = \langle \Phi | a^\dagger_j a_i | \Phi \rangle.
	\]
	
	\item[b.] Show that the matrix elements of $\gamma^{\HF}$ are given by Eq.(4.40).
	
	\end{enumerate}
	\end{exercise}
	
	\begin{solution}
	
	\begin{itemize}
	
	\item[a.] We can use the conclusion in exercise 4.6(b), viz.,
	\begin{align*}
		\tr{\h\gamma} = \sum_{ij} h_{ij} \gamma_{ji} = \langle \Phi | \mathscr{O}_1 | \Phi \rangle .
	\end{align*}
	With the second quantization form, we find that
	\[
		\sum_{ij} h_{ij} \gamma_{ji} = \langle \Phi | \mathscr{O}_1 | \Phi \rangle = \sum_{ij} \langle \Phi | a^\dagger_i a_j | \Phi_1 \rangle h_{ij} \Leftrightarrow \sum_{ij} \left[ \langle \Phi | a^\dagger_i a_j | \Phi \rangle - \gamma_{ji} \right] h_{ij} = 0.
	\]
	For any system, this equation holds true, which means that terms $h_{ij}$ are linearly independent. Thus,
	\begin{equation}
		\gamma_{ji} = \langle \Phi | a^\dagger_i a_j | \Phi \rangle,
	\end{equation}
	which equals $\gamma_{ij} = \langle \Phi | a^\dagger_j a_i | \Phi \rangle$.
	
	\item[b.] If $i$ belongs to unoccupied, $a_i | \Phi \rangle$ vanishes, and so does $ \langle \Phi | a^\dagger_j$ if $j$ belongs to unoccupied. If the indices $i$ and $j$ are occupied, viz., $i=a$ and $j=b$, with $a^\dagger_a | \Phi \rangle = 0$,
	\[
		\gamma^\HF_{ab} = \langle \Phi | a^\dagger_a a_b | \Phi \rangle = \langle \Phi | ( \delta_{ab} - a_b a^\dagger_a) | \Phi \rangle = \delta_{ab} - \langle \Phi | a_b a^\dagger_a | \Phi \rangle = \delta_{ab} - 0 = \delta_{ab}.
	\]
	Thus, we have proved
	\[
		\gamma^\HF_{ij} = \begin{cases}
   			\delta_{ij}, & \text{iff } i, j \in \text{occupied}, \\
   			0, & \text{otherwise}.
		\end{cases}
	\]
	\end{itemize}		
	
	\end{solution}
	
	% 4.8
	\begin{exercise}
	For the special case of a two-electron system, the use of natural orbitals dramatically reduces the size of the full CI expansion. If $\psi_1$ is the occupied Hartree-Fock spatial orbital and $\psi_r$, $r=2,3,...,K$ are virtual spatial orbitals, the normalized full CI singlet wave function has the form
	\[
		|{}^{1}\Phi_0 \rangle = c_0 |1\bar{1}\rangle + \sum_{r=2}^K c^r_1 | {}^{1} \Psi^r_1 \rangle + \frac{1}{2} \sum_{r=2}^K \sum_{s=2}^K c^{rs}_{11} | {}^{1} \Psi^{rs}_{11} \rangle
	\]
	where the singly and doubly excited spin adapted configurations are defined in Subsection 2.5.2.
	
	\begin{enumerate}
	
	\item[a.] Show that $|{}^{1}\Phi_0 \rangle$	can be cast into the form
	\[
		|{}^{1}\Phi_0 \rangle = \sum_{i=1}^K \sum_{j=1}^K C_{ij} | \psi_i \bar{\psi}_j \rangle
	\]
	where $\C$ is a symmetric $K \times K$ matrix.
	
	\item[b.] Show that
	\[
		\gamma( \boldsymbol{x}_1 , \boldsymbol{x}_1^\prime ) = \sum_{ij} (\C \C^\dagger)_{ij} \left( \psi_i(1) \psi^*_j(1^\prime) + \bar{\psi}_i (1) \bar{\psi}_j^* (1^\prime) \right).
	\]
	
	\item[c.] Let $\U$ be the unitary transformation which diagonalizes $\C$
	\[
		\U^\dagger \C \U = {\bf d}
	\]
	where $({\bf d})_{ij} = d_i \delta_{ij}$. Show that
	\[
		\U^\dagger \C \C^\dagger \U = {\bf d}^2.
	\]
	
	\item[d.] Show that
	\[
		\gamma( \boldsymbol{x}_1 , \boldsymbol{x}_1^\prime ) = \sum_i d^2_i \left( \zeta_i(1)\zeta^*_i(1^\prime) + \bar{\zeta}_i(1) \bar{\zeta}^*_i (1^\prime) \right)
	\]
	where
	\[
		\zeta_i = \sum_k \psi_k U_{ki}.
	\]
	Thus $\U$ diagonalizes the one-matrix, and hence $\zeta_i$ are natural spatial orbitals for the two-electron system.
	
	\item[e.] Finally, since $\C$ is symmetric, $\U$ can be chosen as real. Show that in terms of the natural spatial orbitals, $| {}^{1} \Phi_0 \rangle$ given in part (a) can be rewritten as
	\[
		| {}^{1} \Phi_0 \rangle = \sum_{i=1}^K d_i | \zeta_i \bar{\zeta}_i \rangle
	\]
	and note that this expansion contains only $K$ terms.
	\end{enumerate}
	\end{exercise}
	
	\begin{solution}
	
	\begin{enumerate}
	
	\item[a.] From (2.263) at the page 103 and Table 2.7 at the page 104, we know that	
	\begin{align*}
		| ^1 \Psi^r_1 \rangle &= \frac{ 1 }{ \sqrt{2} } \left( | \Psi^{ \bar{r} }_{ \bar{1} } \rangle + | \Psi^r_1 \rangle \right) = \frac{ 1 }{ \sqrt{2} } \left( | 1 \bar{r} \rangle + | r \bar{1} \rangle \right) , \\
		| ^1 \Psi^{rr}_{11} \rangle &= | \Psi^{r \bar{r}}_{1 \bar{1}} \rangle = | r \bar{r} \rangle , \\
		| ^1 \Psi^{rs}_{11} \rangle &= \frac{ 1 }{ \sqrt{2} } \left( | \Psi^{ r \bar{s} }_{ 1 \bar{1} } \rangle + | \Psi^{s \bar{r} }_{ 1 \bar{1} } \rangle \right) = \frac{ 1 }{ \sqrt{2} } \left( | r \bar{s} \rangle + | s \bar{r} \rangle \right) , \, \forall r \neq s.
	\end{align*}
	
	Thus, 
	\begin{align*}
		| ^1 \Phi_0 \rangle = c_0 | 1 \bar{1} \rangle + \sum_{r=2}^K  \frac{ c^r_1 }{ \sqrt{2} } \left( | 1 \bar{r} \rangle + | r \bar{1} \rangle \right) + \frac{1}{2} \sum_{r=2}^K c^{rr}_{11} | r \bar{r} \rangle + \sum_{r=2}^K \sum_{ \substack{ s=2 \\ r > s } }^K c^{rs}_{11} \frac{ 1 }{ \sqrt{2} } \left( | r \bar{s} \rangle + | s \bar{r} \rangle \right).
	\end{align*}
	From this equation, we find that the coefficients are
	\begin{align*}
		C_{11} &= c_0 , \\
		C_{r1} &= C_{1r} = \frac{ c^r_1 }{ \sqrt{2} }, \\
		C_{rr} &= \frac{ c^{rr}_{11} }{ 2 }, \\
		C_{rs} &= C_{sr} = \frac{ c^{rs}_{11} + c^{sr}_{11} }{ 2 } , \, \forall r \neq s ,
	\end{align*}
	which equals
	\begin{equation}
		|{}^{1}\Phi_0 \rangle = \sum_{i=1}^K \sum_{j=1}^K C_{ij} | \psi_i \bar{\psi}_j \rangle,
	\end{equation}		
	where
	\begin{equation}
		C_{ij} = C_{ji}, \, \forall i, j \in \{ 1,2,\cdots, K \}.
	\end{equation}
	In other words, $\C$ is a symmetric $K \times K$ matrix.
	
	\item[b.] Note that there are two electrons in this system,
	\begin{align*}
		\gamma( \boldsymbol{x}_1 , \boldsymbol{x}^\prime_1 ) &= 2 \int_{ \mathbb{R}^3 } \dif \boldsymbol{x}_2 \Phi( \boldsymbol{x}_1 , \boldsymbol{x}_2 ) \Phi^*( \boldsymbol{x}^\prime_1 , \boldsymbol{x}_2 )  \\
		&= 2 \int_{ \mathbb{R}^3 } \dif \boldsymbol{x}_2 \frac{1}{ \sqrt{2} } \sum_{ i=1 }^K \sum_{ j=1 }^K C_{ij} \left[ \psi_i( \boldsymbol{x}_1 ) \bar{\psi}_j( \boldsymbol{x}_2 ) - \psi_i( \boldsymbol{x}_2 ) \bar{\psi}_j( \boldsymbol{x}_1 ) \right] \\
		&\hspace{10em} \times \frac{1}{ \sqrt{2} } \sum_{ k=1 }^K \sum_{ l=1 }^K C^*_{kl} \left[ \psi^*_k( \boldsymbol{x}^\prime_1 ) \bar{\psi}^*_l( \boldsymbol{x}_2 ) - \psi^*_k( \boldsymbol{x}_2 ) \bar{\psi}^*_l( \boldsymbol{x}^\prime_1 ) \right] \\
		&= \sum_{ i=1 }^K \sum_{ j=1 }^K \sum_{ k=1 }^K \sum_{ l=1 }^K C_{ij} C^*_{kl} \\ 
		&\hspace{2em}\left[ \psi_i( \boldsymbol{x}_1 ) \psi^*_k( \boldsymbol{x}^\prime_1 ) \int_{ \mathbb{R}^3 } \dif \boldsymbol{x}_2 \bar{\psi}_j( \boldsymbol{x}_2 ) \bar{\psi}^*_l( \boldsymbol{x}_2 ) - \psi_i( \boldsymbol{x}_1 ) \bar{\psi}^*_l( \boldsymbol{x}^\prime_1 ) \int_{ \mathbb{R}^3 } \dif \boldsymbol{x}_2 \bar{\psi}_j( \boldsymbol{x}_2 ) \psi^*_k( \boldsymbol{x}_2 ) \right. \\
		&\hspace{2em} \left. - \bar{\psi}_j( \boldsymbol{x}_1 ) \psi^*_k( \boldsymbol{x}^\prime_1 ) \int_{ \mathbb{R}^3 } \dif \boldsymbol{x}_2 \psi_i( \boldsymbol{x}_2 ) \bar{\psi}^*_l( \boldsymbol{x}_2 ) + \bar{\psi}_j( \boldsymbol{x}_1 ) \bar{\psi}^*_l( \boldsymbol{x}^\prime_1 ) \int_{ \mathbb{R}^3 } \dif \boldsymbol{x}_2 \psi_i( \boldsymbol{x}_2 ) \psi^*_k( \boldsymbol{x}_2 ) \right] \\
		&= \sum_{ i=1 }^K \sum_{ j=1 }^K \sum_{ k=1 }^K \sum_{ l=1 }^K C_{ij} C^*_{kl} \left[ \psi_i( \boldsymbol{x}_1 ) \psi^*_k( \boldsymbol{x}^\prime_1 ) \delta_{jl} + \bar{\psi}_j( \boldsymbol{x}_1 ) \bar{\psi}^*_l( \boldsymbol{x}^\prime_1 ) \delta_{ik} \right] \\
		&= \sum_{ i=1 }^K \sum_{ j=1 }^K \sum_{ k=1 }^K C_{ij} C^*_{kj} \psi_i( \boldsymbol{x}_1 ) \psi^*_k( \boldsymbol{x}^\prime_1 ) + \sum_{ i=1 }^K \sum_{ j=1 }^K \sum_{ l=1 }^K C_{ij} C^*_{il} \bar{\psi}_j( \boldsymbol{x}_1 ) \bar{\psi}^*_l( \boldsymbol{x}^\prime_1 ) \\
		&= \sum_{ i=1 }^K \sum_{ j=1 }^K \sum_{ k=1 }^K C_{ij} C^\dagger_{jk} \psi_i( \boldsymbol{x}_1 ) \psi^*_k( \boldsymbol{x}^\prime_1 ) + \sum_{ i=1 }^K \sum_{ j=1 }^K \sum_{ k=1 }^K C_{ji} C^\dagger_{ik} \bar{\psi}_j( \boldsymbol{x}_1 ) \bar{\psi}^*_k( \boldsymbol{x}^\prime_1 ) \\
		&= \sum_{ i=1 }^K \sum_{ j=1 }^K \psi_i( \boldsymbol{x}_1 ) \psi^*_j( \boldsymbol{x}^\prime_1 ) \left( \sum_{ k=1 }^K C_{ik} C^\dagger_{kj} \right)  + \sum_{ i=1 }^K \sum_{ j=1 }^K  \bar{\psi}_i( \boldsymbol{x}_1 ) \bar{\psi}^*_j( \boldsymbol{x}^\prime_1 ) \left( \sum_{ k=1 }^K C_{ik} C^\dagger_{kj} \right)\\
		&= \sum_{ i=1 }^K \sum_{ j=1 }^K (\C \C^\dagger)_{ij} \left( \psi_i( \boldsymbol{x}_1 ) \psi^*_j( \boldsymbol{x}^\prime_1 ) + \bar{\psi}_i( \boldsymbol{x}_1 ) \bar{\psi}^*_j( \boldsymbol{x}^\prime_1 ) \right).
	\end{align*}
	Thus, we have proved that
	\begin{equation}
		\gamma( \boldsymbol{x}_1 , \boldsymbol{x}^\prime_1 ) = \sum_{ i=1 }^K \sum_{ j=1 }^K (\C \C^\dagger)_{ij} \left( \psi_i( \boldsymbol{x}_1 ) \psi^*_j( \boldsymbol{x}^\prime_1 ) + \bar{\psi}_i( \boldsymbol{x}_1 ) \bar{\psi}^*_j( \boldsymbol{x}^\prime_1 ) \right).
	\end{equation}
	
	\item[c.] In fact, all $d_i$ are real as all eigenvalues of a real symmetric matrix are real, thus ${\bf d}^\dagger = {\bf d}$, and
	\begin{equation}
		{\bf d}^2 = {\bf d}{\bf d}^\dagger = \U^\dagger \C \U (\U^\dagger \C \U)^\dagger = \U^\dagger \C \U \U^\dagger \C^\dagger \U = \U^\dagger \C \C^\dagger \U.
	\end{equation}
	
	\item[d.] From the former issue, we know
	\[
		\C \C^\dagger = \U {\bf d}^2 \U^\dagger.
	\]
	Thus, we find that
	\begin{align*}
		\gamma( \boldsymbol{x}_1 , \boldsymbol{x}^\prime_1 ) &= \sum_{ i=1 }^K \sum_{ j=1 }^K (\C \C^\dagger)_{ij} \left( \psi_i( \boldsymbol{x}_1 ) \psi^*_j( \boldsymbol{x}^\prime_1 ) + \bar{\psi}_i( \boldsymbol{x}_1 ) \bar{\psi}^*_j( \boldsymbol{x}^\prime_1 ) \right) \\
		&= \sum_{ i=1 }^K \sum_{ j=1 }^K ( \U {\bf d}^2 \U^\dagger )_{ij} \left( \psi_i( \boldsymbol{x}_1 ) \psi^*_j( \boldsymbol{x}^\prime_1 ) + \bar{\psi}_i( \boldsymbol{x}_1 ) \bar{\psi}^*_j( \boldsymbol{x}^\prime_1 ) \right) \\
		&= \sum_{ i=1 }^K \sum_{ j=1 }^K \sum_{ k=1 }^K \U_{ik} d^2_k \U^\dagger_{kj} \left( \psi_i( \boldsymbol{x}_1 ) \psi^*_j( \boldsymbol{x}^\prime_1 ) + \bar{\psi}_i( \boldsymbol{x}_1 ) \bar{\psi}^*_j( \boldsymbol{x}^\prime_1 ) \right) \\
		&= \sum_{ k=1 }^K d^2_k \left( \sum_{ i=1 }^K \psi_i( \boldsymbol{x}_1 ) \U_{ik} \sum_{ j=1 }^K \psi^*_j( \boldsymbol{x}^\prime_1 ) \U^*_{jk} + \sum_{ i=1 }^K \bar{\psi}_i( \boldsymbol{x}_1 ) \U_{ik} \sum_{ j=1 }^K \bar{\psi}^*_j( \boldsymbol{x}^\prime_1 ) \U^*_{jk} \right).
	\end{align*}
	Therefore, we define
	\[
		\zeta_i = \sum_{k=1}^K \psi_k U_{ki},
	\]
	and we obtain
	\begin{equation}
		\gamma( \boldsymbol{x}_1 , \boldsymbol{x}^\prime_1 ) = \sum_{ k=1 }^K d^2_k \left( \zeta_k(\boldsymbol{x}_1) \zeta^*_k( \boldsymbol{x}^\prime_1 ) + \bar{\zeta}_k( \boldsymbol{x}_1 ) \bar{\zeta}^*_k( \boldsymbol{x}^\prime_1 ) \right) = \sum_{ i=1 }^K d^2_i \left( \zeta_i(\boldsymbol{x}_1) \zeta^*_i( \boldsymbol{x}^\prime_1 ) + \bar{\zeta}_i( \boldsymbol{x}_1 ) \bar{\zeta}^*_i( \boldsymbol{x}^\prime_1 ) \right).
	\end{equation}
	We conclude thay $U$ diagonalizes the one-matrix, and hence $\zeta_i$ are natural spatial orbitals for the two-electron system.
	
	\item[e.] Now we convert $\C$ firstly,	
	\[
		\C = \U {\bf d} \U^\dagger \Leftrightarrow C_{ij} = \sum_{k=1}^K \U_{ik} d_{k} \U^\dagger_{kj} = \sum_{k=1}^K \U_{ik} d_{k} \U^*_{jk} 
	\]
	Thus we arrive at
	\begin{equation}
		|{}^{1} \Phi_0 \rangle = \sum_{i=1}^K \sum_{j=1}^K C_{ij} | \psi_i \bar{\psi}_j \rangle = \sum_{i=1}^K \sum_{j=1}^K \sum_{k=1}^K \U_{ik} d_{k} \U^*_{jk} | \psi_i \bar{\psi}_j \rangle = \sum_{k=1}^K d_{k} | \zeta_k \bar{\zeta}_k \rangle = \sum_{i=1}^K d_{i} | \zeta_i \bar{\zeta}_i \rangle.
	\end{equation}
	We find that this expansion contains only $K$ terms.
	
	\end{enumerate}
	
	\end{solution}
	
	\section{The Multiconfiguration Self-Consistent Field (MCSCF) and \texorpdfstring{\\}- Generalized Valence Bond (GVB) Methods}	
	
	\begin{exercise}
	Consider the transformation
	\begin{align*}
		u &= \frac{1}{\sqrt{ a^2 + b^2 }} \left( a \psi_A + b \psi_B \right), \\
		v &= \frac{1}{\sqrt{ a^2 + b^2 }} \left( a \psi_A - b \psi_B \right).
	\end{align*}
	\begin{enumerate}
	
	\item[a.] Show that
	\[
		\langle u | u \rangle = \langle v | v \rangle = 1
	\]
	and
	\[
		\langle u | v \rangle \equiv S = \frac{ a^2 - b^2 }{ a^2 + b^2 }.
	\]
	
	\item[b.] Show that $|\Psi_{\rm GVB}\rangle$ in Eq.(4.52) can be rewritten as
	\[
		|\Psi_{\rm GVB}\rangle = \frac{1}{\sqrt{ a^4 + b^4 }} \left[ a^2 \psi_A(1) \psi_A(2) - b^2 \psi_B(1) \psi_B(2) \right] \frac{1}{\sqrt{2}} \left( \alpha(1)\beta(2) - \alpha(2)\beta(1) \right)
	\]
	and conclude that this is identical to $|\Psi_{\rm MCSCF}\rangle$ in Eq.(4.48) if
	\begin{align*}
		c_A &= \frac{ a^2 }{ \sqrt{ a^4 + b^4 } }, \\
		c_B &= -\frac{ b^2 }{ \sqrt{ a^4 + b^4 } }.
	\end{align*}
	\end{enumerate}		
	\end{exercise}
	
	\begin{solution}
	
	\begin{enumerate}
	
	\item[a.]
	\begin{align}
		\langle u | u \rangle &= \frac{1}{ a^2+b^2 } \left[ a^2 \langle \psi_A | \psi_A \rangle + ab \langle \psi_A | \psi_B \rangle + ab \langle \psi_B | \psi_A \rangle + b^2 \langle \psi_B | \psi_B \rangle \right] \notag \\ 
		&= \frac{1}{ a^2+b^2 } \left[ a^2 \times 1 + ab \times 0 + ab \times 0 + b^2 \times 1 \right] = 1 .
	\end{align}
		
	\begin{align}
		S \equiv \langle u | v \rangle &= \frac{1}{ a^2+b^2 } \left[ a^2 \langle \psi_A | \psi_A \rangle - ab \langle \psi_A | \psi_B \rangle + ab \langle \psi_B | \psi_A \rangle - b^2 \langle \psi_B | \psi_B \rangle \right] \notag \\ 
		&= \frac{1}{ a^2+b^2 } \left[ a^2 \times 1 - ab \times 0 + ab \times 0 - b^2 \times 1 \right] = \frac{ a^2 - b^2 }{ a^2 + b^2 }.
	\end{align}
	
	\item[b.]
	
	\begin{align}
		| \Psi_{\rm GVB} \rangle &= \frac{1}{ \sqrt{ 2 \left[ 1 + ( \frac{ a^2 - b^2 }{ a^2 + b^2 } )^2 \right] } } \left[ \frac{1}{ a^2 + b^2 } ( a \psi_A(1) + b \psi_B(1) )( a \psi_A(2) - b \psi_B(2) ) \right. \notag \\
		&\hspace{2em} \left. + \frac{1}{ a^2 + b^2 } ( a \psi_A(2) + b \psi_B(2) )( a \psi_A(1) - b \psi_B(1) ) \right] \frac{1}{ \sqrt{2} } \left[ \alpha(1) \beta(2) - \alpha(2) \beta(1) \right] \notag \\
		&=\frac{ a^2 + b^2 }{ \sqrt{ 2 \left[ ( a^2 + b^2 )^2 + ( a^2 - b^2 )^2 \right] } } \times \frac{1}{ \sqrt{2} \left( a^2 + b^2 \right) } \left[ \alpha(1) \beta(2) - \alpha(2) \beta(1) \right] \notag \\
		&\hspace{2em}\left[ ( a \psi_A(1) + b \psi_B(1) )( a \psi_A(2) - b \psi_B(2) ) + ( a \psi_A(2) + b \psi_B(2) )( a \psi_A(1) - b \psi_B(1) ) \right] \notag \\ 
		&= \frac{ 1 }{ 2 \sqrt{ 2a^4 + 2b^4 } } \left[ 2 a^2 \psi_A(1) \psi_A(2) - 2 b^2 \psi_B(1) \psi_B(2) \right] \left[ \alpha(1) \beta(2) - \alpha(2) \beta(1) \right] \notag \\
		&= \frac{ 1 }{ \sqrt{ a^4 + b^4 } }  \left[ a^2 \psi_A(1) \psi_A(2) - b^2 \psi_B(1) \psi_B(2) \right] \frac{1}{ \sqrt{2} } \left[ \alpha(1) \beta(2) - \alpha(2) \beta(1) \right] 
	\end{align}
	Thus,
	\[
		c_A = \frac{ a^2 }{ \sqrt{ a^4 + b^4 } } , \quad c_B = -\frac{ b^2 }{ \sqrt{ a^4 + b^4 } }.
	\]
	\end{enumerate}		
		
	\end{solution}
	
	\section{Truncated CI and the Size-Consistency Problem}	
	
	
	
	
\end{document}
