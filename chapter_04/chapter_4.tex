\documentclass[a4paper]{book}

\usepackage{amsmath}
\usepackage{amssymb}
\usepackage[hypcap=false]{caption}
\usepackage{enumitem}	% 定制enumerate标号
\usepackage{fancyhdr}
\pagestyle{plain} 				% 此处为fancy时有页眉
\usepackage{geometry}
\geometry{
	left=2cm,
	right=2cm,
	top=2cm,
	bottom=2cm,
}
\usepackage{hyperref}
\hypersetup{
    colorlinks=true,            %链接颜色
    linkcolor=blue,             %内部链接
    filecolor=magenta,          %本地文档
    urlcolor=cyan,              %网址链接
}
\usepackage[none]{hyphenat}		% 阻止长单词分在两行
\usepackage{mathrsfs}
\usepackage[version=4]{mhchem}
\usepackage{subcaption}
\usepackage{titlesec}

\RequirePackage[many]{tcolorbox}
\tcbset{
    boxed title style={colback=magenta},
	breakable,
	enhanced,
	sharp corners,
	attach boxed title to top left={yshift=-\tcboxedtitleheight,  yshifttext=-.75\baselineskip},
	boxed title style={boxsep=1pt,sharp corners},
    fonttitle=\bfseries\sffamily,
}

\definecolor{skyblue}{rgb}{0.54, 0.81, 0.94}

\newtcolorbox[auto counter, number within=chapter, number format=\arabic]{exercise}[1][]{
    title={Exercise~\thetcbcounter},
    colframe=skyblue,
    colback=skyblue!12!white,
    boxed title style={colback=skyblue},
    overlay unbroken and first={
        \node[below right,font=\small,color=skyblue,text width=.8\linewidth]
        at (title.north east) {#1};
    }
}

\newtcolorbox[auto counter, number within=chapter, number format=\arabic]{solution}[1][]{
    title={Solution~\thetcbcounter},
    colframe=teal!60!green,
    colback=green!12!white,
    boxed title style={colback=teal!60!green},
    overlay unbroken and first={
        \node[below right,font=\small,color=red,text width=.8\linewidth]
        at (title.north east) {#1};
    }
}

% special new commands for common symbols used in the article
\newcommand\la{\langle}
\newcommand\ra{\rangle}
\newcommand\lr[2]{\langle#1\|#2\rangle}
\newcommand\tr[1]{\mathrm{tr(#1)}}
\newcommand\Tr[3]{#1\mathrm\#2#3}
\newcommand*{\dif}{\mathop{}\!\mathrm{d}}
\renewcommand\det[1]{{\rm det}\left(#1\right)}
\newcommand{\HF}{{\rm HF}}
\newcommand{\corr}{{\rm corr}}
\newcommand{\au}{{\rm a.u.}}

\newcommand{\A}{{\bf A}}
\newcommand{\B}{{\bf B}}
\newcommand{\C}{{\bf C}}
\newcommand{\I}{{\bf 1}}
\newcommand{\U}{{\bf U}}
\newcommand{\Op}{{\bf O}}
\newcommand{\h}{{\bf h}}


\titleformat{\chapter}[display]
  {\bfseries\Large}
  {\filright\MakeUppercase{\chaptertitlename} \Huge\thechapter}
  {1ex}
  {\titlerule\vspace{1ex}\filleft}
  [\vspace{1ex}\titlerule]
  
\allowdisplaybreaks

\begin{document}

	\stepcounter{chapter}\stepcounter{chapter}\stepcounter{chapter}

	\chapter{Configuration Interaction}
	
	\section{Multiconfigurational Wave Functions and the Structure of the Full CI Matrix}
	
	\subsection{Intermediate Normalization and an Expression for the Correlation Energy}
	
	\begin{exercise}
	Obtain Eq.(4.12) from Eq.(4.11). It will prove convenient to use unrestricted summations.
	\end{exercise}
	
	\begin{solution}
	
	Note that the index $r$ must be included in the set $\{ t, u, v \}$ and the index $a$ must be included in the set $\{ c,d,e\}$ for a matrix element of $\langle \Psi^r_a | \mathscr{H} | \Psi^{tuv}_{cde} \rangle$. Therefore, we find that
	\begin{align*}
		&\hspace{1.4em}\sum_{ \substack{ c<d<e \\ t<u<v } } \langle \Psi^r_a | \mathscr{H} | \Psi^{tuv}_{cde} \rangle c^{tuv}_{cde} = \frac{1}{\left(3!\right)^2} \sum_{ \substack{ cde \\ tuv} } \langle \Psi^r_a | \mathscr{H} | \Psi^{tuv}_{cde} \rangle c^{tuv}_{cde} \\
		&= \frac{1}{\left(3!\right)^2}  \left[ \sum_{ \substack{ de \\ uv } } \langle \Psi^r_a | \mathscr{H} | \Psi^{ruv}_{ade} \rangle c^{ruv}_{ade} + \sum_{ \substack{ de \\ tv } } \langle \Psi^r_a | \mathscr{H} | \Psi^{trv}_{ade} \rangle c^{trv}_{ade} + \sum_{ \substack{ de \\ tu } } \langle \Psi^r_a | \mathscr{H} | \Psi^{tur}_{ade} \rangle c^{tur}_{ade} \right. \\
		&\hspace{ 6em } \left. + \sum_{ \substack{ ce \\ uv } } \langle \Psi^r_a | \mathscr{H} | \Psi^{ruv}_{cae} \rangle c^{ruv}_{cae} + \sum_{ \substack{ ce \\ tv } } \langle \Psi^r_a | \mathscr{H} | \Psi^{trv}_{cae} \rangle c^{trv}_{cae} + \sum_{ \substack{ ce \\ tu } } \langle \Psi^r_a | \mathscr{H} | \Psi^{tur}_{cae} \rangle c^{tur}_{cae} \right. \\
		&\hspace{ 6em } \left. + \sum_{ \substack{ cd \\ uv } } \langle \Psi^r_a | \mathscr{H} | \Psi^{ruv}_{cda} \rangle c^{ruv}_{cda} + \sum_{ \substack{ cd \\ tv } } \langle \Psi^r_a | \mathscr{H} | \Psi^{trv}_{cda} \rangle c^{trv}_{cda} + \sum_{ \substack{ cd \\ tu } } \langle \Psi^r_a | \mathscr{H} | \Psi^{tur}_{cda} \rangle c^{tur}_{cda} \right] .
	\end{align*}
	
	Then, these dummy indices should be converted into the same one, viz.,
	\begin{align*}
		&\hspace{1.4em} \sum_{ \substack{ c<d<e \\ t<u<v } } \langle \Psi^r_a | \mathscr{H} | \Psi^{tuv}_{cde} \rangle c^{tuv}_{cde} \\
		&= \frac{1}{\left(3!\right)^2}  \left[ \sum_{ \substack{ cd \\ tu } } \langle \Psi^r_a | \mathscr{H} | \Psi^{rtu}_{acd} \rangle c^{rtu}_{acd} + \sum_{ \substack{ cd \\ tu } } \langle \Psi^r_a | \mathscr{H} | \Psi^{tru}_{acd} \rangle c^{tru}_{acd} + \sum_{ \substack{ cd \\ tu } } \langle \Psi^r_a | \mathscr{H} | \Psi^{tur}_{acd} \rangle c^{tur}_{acd} \right. \\
		&\hspace{ 6em } + \sum_{ \substack{ cd \\ tu } } \langle \Psi^r_a | \mathscr{H} | \Psi^{rtu}_{cad} \rangle c^{rtu}_{cad} + \sum_{ \substack{ cd \\ tu } } \langle \Psi^r_a | \mathscr{H} | \Psi^{tru}_{cad} \rangle c^{tru}_{cad} + \sum_{ \substack{ cd \\ tu } } \langle \Psi^r_a | \mathscr{H} | \Psi^{tur}_{cad} \rangle c^{tur}_{cad} \\
		&\hspace{ 6em } \left. + \sum_{ \substack{ cd \\ tu } } \langle \Psi^r_a | \mathscr{H} | \Psi^{rtu}_{cda} \rangle c^{rtu}_{cda} + \sum_{ \substack{ cd \\ tu } } \langle \Psi^r_a | \mathscr{H} | \Psi^{tru}_{cda} \rangle c^{tru}_{cda} + \sum_{ \substack{ cd \\ tu } } \langle \Psi^r_a | \mathscr{H} | \Psi^{tur}_{cda} \rangle c^{tur}_{cda} \right] \\
		&= \frac{1}{\left(3!\right)^2} \times 9 \sum_{ \substack{ cd \\ tu } } \langle \Psi^r_a | \mathscr{H} | \Psi^{rtu}_{acd} \rangle c^{rtu}_{acd} = \frac{1}{\left(2!\right)^2} \sum_{ \substack{ cd \\ tu } } \langle \Psi^r_a | \mathscr{H} | \Psi^{rtu}_{acd} \rangle c^{rtu}_{acd} = \sum_{ \substack{ c<d \\ t<u } } \langle \Psi^r_a | \mathscr{H} | \Psi^{rtu}_{acd} \rangle c^{rtu}_{acd}.
	\end{align*}
	Thus, we have proved that
	\begin{equation}
		\sum_{ \substack{ c<d<e \\ t<u<v } } \langle \Psi^r_a | \mathscr{H} | \Psi^{tuv}_{cde} \rangle c^{tuv}_{cde} = \sum_{ \substack{ c<d \\ t<u } } \langle \Psi^r_a | \mathscr{H} | \Psi^{rtu}_{acd} \rangle c^{rtu}_{acd}.
	\end{equation}
	
	With this equation, it is clear that (4.12) can be obtained from (4.11).
	
	\end{solution}
	
	\begin{exercise}
	Using the secular determinant approach show that the lowest eigenvalue of the matrix
	\[
		\begin{pmatrix}
			0 & K_{12} \\ K_{12} & 2\Delta
		\end{pmatrix}
	\]
	is given by Eq.(4.23).
	\end{exercise}
	
	\begin{solution}
	
	The introduction of the secular determinant approach is demonstrated in the page 18. The matrix in the exercise 4.2 is denoted as $H$, then
	\[
		\det{H - \varepsilon I} = \begin{vmatrix}
		- \varepsilon & K_{12} \\ K_{12} & 2\Delta - \varepsilon
		\end{vmatrix} = \varepsilon^2 - 2 \Delta \varepsilon -K^2_{12} = 0 ,
	\]	
	The discriminant $\Delta_E$ of this quadratic equation is
	\[
		\Delta_E = 4 \Delta^2 - 4 \times ( -K^2_{12} ) = 4( \Delta^2 + K^2_{12} )
	\]	
	Thus, the root are
	\[
		E_1 = \Delta + \sqrt{ \Delta^2 + K^2_{12} }, \quad E_2 = \Delta - \sqrt{ \Delta^2 + K^2_{12} }.
	\]	
	
	Therefore, the lowest root is the correlation energy, viz.,
	\begin{equation}
		E_\corr = \Delta - \sqrt{ \Delta^2 + K^2_{12} }.
	\end{equation}
	
	\end{solution}
	
	\begin{exercise}
	Calculate the coefficient of the double excitation ($c$) in the intermediate normalized CI wave function at $R$ = 1.4 $\au$, using the STO-3G integrals given in Appendix D. Show analytically that as $R \rightarrow \infty$, $c \rightarrow -1$, and hence that at large distances the Hartree-Fock ground state and the doubly excited configuration have equal weight in the CI ground state. Finally, show that the CI wave function, when normalized to unity, becomes (at $R=\infty$)
	\[
		\frac{1}{\sqrt{2}} \left( | \phi_1 \bar{\phi}_2 \rangle + | \phi_2 \bar{\phi}_1 \rangle \right)
	\]
	where $\phi_1$ and $\phi_2$ are atomic orbitals on centers one and two, respectively.
	\end{exercise}
	
	\begin{solution}
	
	When $R$ = 1.4. $\au$, we know that
	\begin{center}
	\begin{tabular}{ccc}
		$\varepsilon_1 = -0.5782\,\au$, & $\varepsilon_2 = 0.6703\, \au$, & $J_{11} = 0.6746\,\au$, \\
		$J_{12} = 0.6636\,\au$, & $J_{22} = 0.6975\,\au$, & $K_{12} = 0.1813\,\au$
	\end{tabular}
	\end{center}
	
	Firstly, with (4.20), we calculate $2\Delta$ at $R$ = 1.4. $\au$, viz.,
	\[
		2\Delta = \left[ 2( \varepsilon_2 - \varepsilon_1 ) + J_{11} + J_{22} - 4J_{12} + 2K_{12} \right] = 1.5773 \, \au
	\]
	In other words, $\Delta = 0.78865 \, \au$ Thus, the correlation energy $E_\corr$ at $R$ = 1.4. $\au$ is
	\[
		E_\corr = \Delta - \sqrt{ \Delta^2 + K^2_{12} } = -0.02057 \, \au.
	\]
	Therefore,
	\begin{equation}
		c = \frac{ K_{12} }{ E_\corr - 2\Delta } =  \frac{ 0.1813 \, \au }{ -0.02057 \, \au. - 1.5773 \, \au. }  \approx - 0.1135.
	\end{equation}
	
	Indeed, we can find that
	\[
		\Delta = \varepsilon_2 - \varepsilon_1 + \frac{1}{2} J_{11} + \frac{1}{2} J_{22} - 2J_{12} + K_{12} = h_{22} - h_{11} - \frac{1}{2} J_{11} + \frac{1}{2} J_{12}.
	\]
	It is clear that	
	\[
		\lim_{ R \rightarrow \infty} \Delta = \lim_{ R \rightarrow \infty} \left[ h_{22} - h_{11} + \frac{1}{2} J_{22} - \frac{1}{2} J_{11} \right] = E(\ce{H}) - E(\ce{H}) + \frac{1}{4} ( \phi_1 \phi_1 | \phi_1 \phi_1 ) - \frac{1}{4} ( \phi_1 \phi_1 | \phi_1 \phi_1 ) = 0.
	\]
	Thus,
	\begin{align*}
		\lim_{ R \rightarrow \infty } c &= \lim_{ R \rightarrow \infty } \frac{ K_{12} }{ E_\corr - 2\Delta } = \lim_{ R \rightarrow \infty } \frac{ K_{12} }{ \Delta - \sqrt{ \Delta^2 + K^2_{12} } - 2\Delta } = \lim_{ R \rightarrow \infty } \frac{ - K_{12} }{ \Delta + \sqrt{ \Delta^2 + K^2_{12} } } \\
		&= - \lim_{ \Delta \rightarrow 0 } \frac{ 1 }{ \frac{ \Delta }{ K_{12} } + \sqrt{ 1 + \left( \frac{ \Delta }{ K_{12} } \right)^2 } } = - \lim_{ x \rightarrow 0 } \frac{ 1 }{ x + \sqrt{ 1 + x^2 } } = -1.
	\end{align*}
	This conclusion means that at large distances the Hartree-Fock ground state $\Psi_0$ and the doubly excited configuration $\Psi^{2 \bar{2}}_{1 \bar{1}}$ have equal weight in the CI ground state $\Phi$, viz.,
	\[
		\lim_{ R \rightarrow \infty } | \Phi \rangle = | \Psi_0 \rangle - | \Psi^{2 \bar{2}}_{1 \bar{1}} \rangle = | \psi_1 \bar{\psi}_1 \rangle - | \psi_2 \bar{\psi}_2 \rangle.
	\]
	Note that as $R \rightarrow \infty$, from (3.236) and (3.237), we find that
	\[
		\lim_{ R \rightarrow \infty } \psi_1 = \frac{1}{ \sqrt{2} } ( \phi_1 + \phi_2 ) , \quad \lim_{ R \rightarrow \infty } \psi_2 = \frac{1}{ \sqrt{2} } ( \phi_1 - \phi_2 ).
	\]
	Thus,
	\begin{align*}
		\lim_{ R \rightarrow \infty } | \psi_1 \bar{\psi}_1 \rangle &= \frac{1}{2} | ( \phi_1 + \phi_2 ) ( \bar{\phi}_1 + \bar{\phi}_2 ) \rangle = \frac{1}{2} \left( | \phi_1 \bar{\phi}_1 \rangle + | \phi_1 \bar{\phi}_2 \rangle + | \phi_2 \bar{\phi}_1 \rangle + | \phi_2 \bar{\phi}_2 \rangle \right), \\
		\lim_{ R \rightarrow \infty } | \psi_2 \bar{\psi}_2 \rangle &= \frac{1}{ 2 } | ( \phi_1 - \phi_2 ) ( \bar{\phi}_1 - \bar{\phi}_2 ) \rangle = \frac{1}{2} \left( | \phi_1 \bar{\phi}_1 \rangle - | \phi_1 \bar{\phi}_2 \rangle - | \phi_2 \bar{\phi}_1 \rangle + | \phi_2 \bar{\phi}_2 \rangle \right),
	\end{align*}
	and then	
	\[
		\lim_{ R \rightarrow \infty } | \Phi \rangle = \lim_{ R \rightarrow \infty } | \psi_1 \bar{\psi}_1 \rangle - \lim_{ R \rightarrow \infty } | \psi_2 \bar{\psi}_2 \rangle = | \phi_1 \bar{\phi}_2 \rangle + | \phi_2 \bar{\phi}_1 \rangle
	\]
	Thus, at $R = \infty$, the normalized CI wave function is
	\begin{equation}
		\lim_{R \rightarrow \infty} | \Phi \rangle = \lim_{R \rightarrow \infty} \frac{1}{ \langle \Phi_0 | \Phi_0 \rangle } | \Phi_0 \rangle = \frac{1}{ \sqrt{2} } \left( | \phi_1 \bar{\phi}_2 \rangle + | \phi_2 \bar{\phi}_1 \rangle \right) .
	\end{equation}
	
	We have proved two conclusions at $R = \infty$, the equal weight of the Hartree-Fock ground state $\Psi_0$ and the doubly excited configuration $\Psi^{2 \bar{2}}_{1 \bar{1}}$, and the form of normalized CI wave function.
		
	\end{solution}
	
	\section{Doubly Excited CI}
	
	\section{Some Illustrative Calculations}
	
	\section{Natural Orbitals and the One-Particle Reduced Density Matrix}
	
	\begin{exercise}
	Show that $\gamma$ is a Hermitian matrix.
	\end{exercise}
	
	\begin{solution}

	\begin{align*}
		\gamma^*( \boldsymbol{x}_1 , \boldsymbol{x}^\prime_1 ) &= \left( N \int_{\mathbb{R}^3} \dif \boldsymbol{x}_2 \cdots \int_{\mathbb{R}^3} \dif \boldsymbol{x}_N \Phi^*( \boldsymbol{x}_1 , \boldsymbol{x}_2 , \cdots , \boldsymbol{x}_N ) \Phi( \boldsymbol{x}^\prime_1 , \boldsymbol{x}_2 , \cdots , \boldsymbol{x}_N ) \right)^* \\
		&= N \int_{\mathbb{R}^3} \dif \boldsymbol{x}_2 \cdots \int_{\mathbb{R}^3} \dif \boldsymbol{x}_N \Phi^*( \boldsymbol{x}^\prime_1 , \boldsymbol{x}_2 , \cdots , \boldsymbol{x}_N ) \Phi( \boldsymbol{x}_1 , \boldsymbol{x}_2 , \cdots , \boldsymbol{x}_N ) = \gamma( \boldsymbol{x}^\prime_1 , \boldsymbol{x}_1 )
	\end{align*}
	
	\begin{align*}
		\gamma^*_{ji} &= \left( \int_{\mathbb{R}^3} \dif \boldsymbol{x}_1 \int_{\mathbb{R}^3} \dif \boldsymbol{x}^\prime_1 \chi^*_j( \boldsymbol{x}_1 ) \gamma( \boldsymbol{x}_1 , \boldsymbol{x}^\prime_1 ) \chi_i( \boldsymbol{x}^\prime_1 ) \right)^* = \int_{\mathbb{R}^3} \dif \boldsymbol{x}_1 \int_{\mathbb{R}^3} \dif \boldsymbol{x}^\prime_1 \chi^*_i( \boldsymbol{x}^\prime_1 ) \gamma^*( \boldsymbol{x}_1 , \boldsymbol{x}^\prime_1 ) \chi_j( \boldsymbol{x}_1 ) \\
		&= \int_{\mathbb{R}^3} \dif \boldsymbol{x}_1 \int_{\mathbb{R}^3} \dif \boldsymbol{x}^\prime_1 \chi^*_i( \boldsymbol{x}^\prime_1 ) \gamma( \boldsymbol{x}^\prime_1 , \boldsymbol{x}_1 ) \chi_j( \boldsymbol{x}_1 ) = \int_{\mathbb{R}^3} \dif \boldsymbol{x}_1 \int_{\mathbb{R}^3} \dif \boldsymbol{x}_1 \chi^*_i( \boldsymbol{x}_1 ) \gamma( \boldsymbol{x}_1 , \boldsymbol{x}^\prime_1 ) \chi_j( \boldsymbol{x}^\prime_1 ) = \gamma_{ij}
	\end{align*}
	Thus, we have proved that $\gamma$ is a Hermitian matrix.
	
	\end{solution}	
	
	
\end{document}
