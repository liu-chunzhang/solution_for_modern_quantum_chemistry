\documentclass[a4paper]{book}
\special{dvipdfmx:config z 0} %取消PDF压缩,加快速度,最终版本生成的时候最好把这句话注释掉

\usepackage{amssymb}
\usepackage[hypcap=false]{caption}
\usepackage{geometry}
\geometry{
	left=2cm,
	right=2cm,
	top=2cm,
	bottom=2cm,
}
\usepackage{hyperref}
\hypersetup{
    colorlinks=true,            %链接颜色
    linkcolor=blue,             %内部链接
    filecolor=magenta,          %本地文档
    urlcolor=cyan,              %网址链接
}
\usepackage[none]{hyphenat}		% 阻止长单词分在两行
\usepackage{mathrsfs}
\usepackage{mathtools}
\usepackage[version=4]{mhchem}
\usepackage{subcaption}
\usepackage{titlesec}

\RequirePackage[many]{tcolorbox}
\tcbset{
    boxed title style={colback=magenta},
	breakable,
	enhanced,
	sharp corners,
	attach boxed title to top left={yshift=-\tcboxedtitleheight,  yshifttext=-.75\baselineskip},
	boxed title style={boxsep=1pt,sharp corners},
    fonttitle=\bfseries\sffamily,
}

\definecolor{skyblue}{rgb}{0.54, 0.81, 0.94}

\newcounter{exercise}[chapter]
\newcounter{solution}[chapter]
\newcounter{eqs}[solution]

\newenvironment{sequation}
  {\begin{equation}\stepcounter{eqs}\tag{\thesolution-\theeqs}}
  {\end{equation}}

\newtcolorbox[use counter=exercise, number within=chapter, number format=\arabic]{exercise}[1][]{
    title={Exercise~\thetcbcounter},
    colframe=skyblue,
    colback=skyblue!12!white,
    boxed title style={colback=skyblue},
    overlay unbroken and first={
        \node[below right,font=\small,color=skyblue,text width=.8\linewidth]
        at (title.north east) {#1};
    }
    label={\unskip},
    before upper={
        \phantomsection
        \addcontentsline{toc}{subsubsection}{Exercise\hspace{1em}\thetcbcounter}
    },
}

\newtcolorbox[use counter=solution, number within=chapter, number format=\arabic]{solution}[1][]{
    title={Solution~\thetcbcounter},
    colframe=teal!60!green,
    colback=green!12!white,
    boxed title style={colback=teal!60!green},
    overlay unbroken and first={
        \node[below right,font=\small,color=red,text width=.8\linewidth]
        at (title.north east) {#1};
    }
}

% special new commands for common symbols used in the article
\newcommand\tr[1]{\mathrm{tr(#1)}}
\newcommand*{\dif}{\mathop{}\!\mathrm{d}}
\renewcommand\det[1]{\mathrm{det\left(#1\right)}}
\newcommand{\bfr}{{\bf r}}
\newcommand{\bfx}{{\bf x}}
\newcommand{\HP}{{\rm HP}}

\titleformat{\chapter}[display]
  {\bfseries\Large}
  {\filright\MakeUppercase{\chaptertitlename} \Huge\thechapter}
  {1ex}
  {\titlerule\vspace{1ex}\filleft}
  [\vspace{1ex}\titlerule]
 

\newcommand\Figref[1]{Fig \ref{#1}}
\newcommand\Tableref[1]{Table \ref{#1}}
 
\allowdisplaybreaks

\begin{document}

	\stepcounter{chapter}

	\chapter{Many Electron Wave Functions and Operators}
	
	\section{The Electron Problem}
	
	\subsection{Atomic Units}
	
	\subsection{The Born-Oppenheimer Approximation}
	
	\subsection{The Antisymmetry or Pauli Exclusion Principle}
	
	\section{Orbitals, Slater Determinants, and Basis Functions}
	
	\subsection{Spin Orbitals and Spatial Orbitals}
	
	% 2.1
	\begin{exercise}
	Given a set of $K$ orthonormal spatial functions, $\{\psi^\alpha_i(\bfr)\}$, and another set of $K$ orthonormal functions, $\{\psi^\beta_i(\bfr)\}$, such that the first set is not orthogonal to the second set, i.e., 
	\[
		\int \dif \bfr \, \psi^{\alpha*}_i( \bfr ) \psi^\beta_j( \bfr ) = S_{ij}
	\]
	where ${\bf S}$ is an overlap matrix, show that the set $\{ \chi_i \}$ of $2K$ spin orbitals, formed by multiplying $\psi^\alpha_i( \bfr )$ by the $\alpha$ spin function and $\psi^\beta_i( \bfr )$ by the $\beta$ spin function, i.e.,
	\[ \left.
	\begin{rgathered}
		\chi_{2i-1}(\bfx) = \psi^\alpha_i( \bfr ) \alpha( \omega ) \quad \\
		\chi_{2i}(\bfx) = \psi^\beta_i( \bfr ) \beta( \omega ) \quad
	\end{rgathered} \right\} i = 1, 2, \ldots, K
	\]
	is an orthonormal set.
	\end{exercise}
	
	\begin{solution}
	It is easy to verify the normalization of any $\chi_{2i-1}$ or $\chi_{2j}$, where $i = 1,2,\ldots,K$ and $j = 1,2,\ldots, K$,
	\begin{align*}
		\langle \chi_{2i-1} | \chi_{2j-1} \rangle &= \int \dif \bfx \, \chi^*_{2i-1}( \bfx ) \chi_{2j-1}( \bfx ) = \int \dif \bfr \, \psi^{\alpha*}_i( \bfr ) \psi^\alpha_j( \bfr ) \int \dif \omega \, \alpha^*( \omega ) \alpha( \omega) = \delta_{ij} \times 1 = \delta_{ij} , \\
		\langle \chi_{2i} | \chi_{2j} \rangle &= \int \dif \bfx \, \chi^*_{2i}( \bfx ) \chi_{2j}( \bfx ) = \int \dif \bfr \, \psi^{\beta*}_i( \bfr ) \psi^\beta_j( \bfr ) \int \dif \omega \, \beta^*( \omega ) \beta( \omega) = \delta_{ij} \times 1 = \delta_{ij}.
	\end{align*}
	and the orthogonality between $\chi_{2i-1}$ and $\chi_{2j}$, where $i = 1,2,\ldots,K$ and $j = 1,2,\ldots, K$,
	\begin{align*}
		\langle \chi_{2i-1} | \chi_{2j} \rangle &= \int \dif \bfx \, \chi^*_{2i-1}( \bfx ) \chi_{2j}( \bfx ) = \int \dif \bfr \, \psi^{\alpha*}_i( \bfr ) \psi^\beta_j( \bfr ) \int \dif \omega \, \alpha^*( \omega ) \beta( \omega) = S_{ij} \times 0 = 0 , \\
		\langle \chi_{2i} | \chi_{2j-1} \rangle &= \int \dif \bfx \, \chi^*_{2i}( \bfx ) \chi_{2j-1}( \bfx ) = \int \dif \bfr \, \psi^{\beta*}_i( \bfr ) \psi^\alpha_j( \bfr ) \int \dif \omega \, \beta^*( \omega ) \alpha( \omega) = S^*_{ji} \times 0 = 0 .
	\end{align*}	
	
	Thus, we can that the set $\{ \chi_i \}$ of $2K$ spin orbitals is an orthonormal set.	
	\end{solution}
	
	\subsection{Hartree Products}
	
	% 2.2
	\begin{exercise}
	Show that the Hartree product of (2.30) is an eigenfunction of $\mathscr{H} = \displaystyle\sum_{i=1}^N h(i)$ with an eigenvalue given by (2.32).
	\end{exercise}
	
	\begin{solution}
	
	The verification is easy. With (2.29), we find that
	\begin{sequation}
		\mathscr{H} \Psi^\HP = \left( \sum_{i=1}^N h(i) \right) \left[ \prod_{ j=1 }^N \chi_{j^\prime}( \bfx_j ) \right] = \sum_{i=1}^N \prod_{ j=1 }^N h(i) \chi_{j^\prime}( \bfx_j ) = \sum_{i=1}^N \prod_{ j=1 }^N \varepsilon_i \chi_{j^\prime}( \bfx_j ) = \sum_{i=1}^N \varepsilon_i \prod_{ j=1 }^N  \chi_{j^\prime}( \bfx_j ).
	\end{sequation}		
	
	\end{solution}
	
	\subsection{Slater Determinants}
	
	% 2.3	
	\begin{exercise}
	Show that $\Psi({\bf x}_1,{\bf x}_2)$ of Eq.(2.34) is normalized.
	\end{exercise}
	
	\begin{solution}
	The verification is direct, viz.,
	\begin{align*}
		&\hspace{1.4em}\langle \Psi | \Psi \rangle = \int \dif \vec{\bfx} \, \langle \Psi | \vec{\bfx} \rangle \langle \vec{\bfx} | \Psi \rangle \\
		&= \int \dif \bfx_1 \int \dif \bfx_2 \, \frac{1}{\sqrt{2}} \left[ \chi_i(\bfx_1) \chi_j(\bfx_2) - \chi_j(\bfx_1) \chi_i(\bfx_2) \right]^* \frac{1}{\sqrt{2}} \left[ \chi_i(\bfx_1) \chi_j(\bfx_2) - \chi_j(\bfx_1) \chi_i(\bfx_2) \right] \\
		&= \frac{1}{2} \left( \int \dif \bfx_1 \int \dif \bfx_2 \chi^*_i(\bfx_1) \chi^*_j(\bfx_2) \chi_i(\bfx_1) \chi_j(\bfx_2) - \int \dif \bfx_1 \int \dif \bfx_2 \chi^*_i(\bfx_1) \chi^*_j(\bfx_2) \chi_j(\bfx_1) \chi_i(\bfx_2) \right. \\
		&\hspace{4em} \left. - \int \dif \bfx_1 \int \dif \bfx_2  \chi_j^*(\bfx_1) \chi_i^*(\bfx_2) \chi_i(\bfx_1) \chi_j(\bfx_2) + \int \dif \bfx_1 \int \dif \bfx_2 \chi_j^*(\bfx_1) \chi_i^*(\bfx_2) \chi_j(\bfx_1) \chi_i(\bfx_2) \right) \\
		&= \frac{1}{2} ( 1 - 0 - 0 + 1 ) = 1.
	\end{align*}
	
	\end{solution}
	
	% 2.4
	\begin{exercise}
	Suppose the spin orbitals $\chi_i$ and $\chi_j$ are eigenfunctions of a one-electron operator $h$ with eigenvalues $\varepsilon_i$ and $\varepsilon_j$ as in Eq.(2.29). Show that the Hartree products in Eqs.(2.33a, b) and the antisymmetrized wave function in Eq.(2.34) are eigenfunctions of the independent-particle Hamiltonian $\mathscr{H} = h(1) + h(2)$ (c.f. Eq.(2.28)) and have the same eigenvalue namely, $\varepsilon_i + \varepsilon_j$. 
	\end{exercise}
	
	\begin{solution}
	Firstly, we check the Hartree products of $\chi_i$ and $\chi_j$. With the conclusion of Exercise 2.2, we get that
	\begin{align*}
		\mathscr{H} | \Psi^\HP_{12} \rangle &= ( \varepsilon_i + \varepsilon_j ) | \Psi^\HP_{21} \rangle, \\
		\mathscr{H} | \Psi^\HP_{21} \rangle &= ( \varepsilon_j + \varepsilon_i ) | \Psi^\HP_{21} \rangle = ( \varepsilon_i + \varepsilon_j ) | \Psi^\HP_{21} \rangle.
	\end{align*}
	Thus, the eigenvalue of the Hartree product of $\chi_i$ and $\chi_j$ is irrelevant to their order. Note that
	\[
		\Psi = \frac{1}{\sqrt{2}} \left[ \chi_i(\bfx_1) \chi_j(\bfx_2) - \chi_j(\bfx_1) \chi_i(\bfx_2) \right] = \frac{1}{\sqrt{2}} \left( \Psi^\HP_{12} - \Psi^\HP_{21} \right),
	\]
	we find that
	\begin{align*}
		\mathscr{H} | \Psi \rangle &= \mathscr{H} \frac{1}{\sqrt{2}} \left( | \Psi^\HP_{12} \rangle - | \Psi^\HP_{21} \rangle \right) = \frac{1}{\sqrt{2}} \left( \mathscr{H} | \Psi^\HP_{12} \rangle - \mathscr{H} | \Psi^\HP_{21} \rangle \right) \\
		&= \frac{1}{\sqrt{2}} \left[ ( \varepsilon_i + \varepsilon_j ) | \Psi^\HP_{12} \rangle - ( \varepsilon_i + \varepsilon_j ) | \Psi^\HP_{21} \rangle \right] \\
		&= ( \varepsilon_i + \varepsilon_j ) \frac{1}{\sqrt{2}} \left( | \Psi^\HP_{12} \rangle - | \Psi^\HP_{21} \rangle \right) = ( \varepsilon_i + \varepsilon_j ) | \Psi \rangle.
	\end{align*}
	Thus, we have proved that the Hartree products in Eqs.(2.33a, b) and the antisymmetrized wave function in Eq.(2.34) are eigenfunctions of the independent-particle Hamiltonian $\mathscr{H} = h(1) + h(2)$ and have the same eigenvalue $\varepsilon_i + \varepsilon_j$. 
	\end{solution}
	
	% 2.5
	\begin{exercise}
	Consider the Slater determinants
	\[
		| K \rangle = | \chi_i \chi_j \rangle, \quad | L \rangle = | \chi_k \chi_l \rangle.
	\]
	Show that
	\[
		\langle K | L \rangle = \delta_{ik} \delta_{jl} - \delta_{il} \delta_{jk}.
	\]
	Note that the overlap is zero unless: 1) $k=i$ and $l=j$, in which case $|L\rangle$ = $|K\rangle$ and the overlap is unity and 2) $k=j$ and $l=i$ in which case $|L\rangle= |\chi_j \chi_i \rangle = -|K\rangle$ and the overlap is minus one.
	\end{exercise}
	
	\begin{solution}
	We calculate the inner product firstly,
	\begin{align*}
		\langle K | L \rangle &= \int \dif \vec{\bfx} \, \langle K | \vec{\bfx} \rangle \langle \vec{\bfx} | L \rangle \\
		&= \int \dif \bfx_1 \int \dif \bfx_2 \, \frac{1}{\sqrt{2}}\left[ \chi_i(\bfx_1) \chi_j(\bfx_2) - \chi_j(\bfx_1) \chi_i(\bfx_2) \right]^* \frac{1}{\sqrt{2}}\left[ \chi_k(\bfx_1) \chi_l(\bfx_2) - \chi_l(\bfx_1) \chi_k(\bfx_2) \right] \\
		&= \frac{1}{2} \left[ \int \dif \bfx_1 \int \dif \bfx_2 \, \chi^*_i(\bfx_1) \chi^*_j(\bfx_2) \chi_k(\bfx_1) \chi_l(\bfx_2) - \int \dif \bfx_1 \int \dif \bfx_2 \, \chi^*_i(\bfx_1) \chi^*_j(\bfx_2) \chi_l(\bfx_1) \chi_k(\bfx_2)\right. \\
		&\hspace{4em} \left. - \int \dif \bfx_1 \int \dif \bfx_2 \, \chi^*_j(\bfx_1) \chi^*_i(\bfx_2) \chi_k(\bfx_1) \chi_l(\bfx_2) + \int \dif \bfx_1 \int \dif \bfx_2 \, \chi^*_j(\bfx_1) \chi^*_i(\bfx_2) \chi_l(\bfx_1) \chi_k(\bfx_2) \right] \\
		&= \frac{1}{2} \left( \delta_{ik} \delta_{jl} - \delta_{il}\delta_{jk} - \delta_{jk} \delta_{il} + \delta_{jl}\delta_{ik}\right) = \delta_{ik}\delta_{jl} - \delta_{jk} \delta_{il}.
	\end{align*}
	The conclusion is obvious. 
	\begin{itemize}
	
	\item When $k=i$ and $l=j$, in which case $|L\rangle$ = $|K\rangle$ and the overlap is $1$.
	
	\item When $k=j$ and $l=i$ in which case $|L\rangle = | \chi_k \chi_l \rangle = |\chi_j \chi_i \rangle = -|K\rangle$ and the overlap is $-1$.
	
	\item Otherwise, the overlap is $0$.
	
	\end{itemize}
	\end{solution}
	
	\subsection{The Hartree-Fock Approximation}
	
	\subsection{The Minimal Basis \texorpdfstring{$\ce{H2}$}- Model}
	
	% 2.6
	\begin{exercise}
	Show that $\psi_1$ and $\psi_2$ form an orthonormal set.
	\end{exercise}
	
	\begin{solution}
	Similar to Exercise 2.1, verify the normalization of $\psi_1$ or $\psi_2$, with $S_{12} = S_{21}$,
	\begin{align*}
		\langle \psi_1 | \psi_1 \rangle &= \int \dif \bfr \, \psi^*_1( \bfr ) \psi_1( \bfr ) = \int \dif \bfr \,  \frac{1}{ \sqrt{ 2 ( 1 + S_{12} ) } } \left( \phi_1(\bfr) + \phi_2(\bfr) \right)^* \frac{1}{ \sqrt{ 2 ( 1 + S_{12} ) } } \left( \phi_1(\bfr) + \phi_2(\bfr) \right) \notag \\
		&= \frac{1}{ 2 ( 1 + S_{12} ) } ( 1 + S_{12} + S_{21} + 1 ) = 1, \\
		\langle \psi_2 | \psi_2 \rangle &= \int \dif \bfr \, \psi^*_2( \bfr ) \psi_2( \bfr ) = \int \dif \bfr \,  \frac{1}{ \sqrt{ 2 ( 1 - S_{12} ) } } \left( \phi_1(\bfr) - \phi_2(\bfr) \right)^* \frac{1}{ \sqrt{ 2 ( 1 - S_{12} ) } } \left( \phi_1(\bfr) - \phi_2(\bfr) \right) \notag \\
		&= \frac{1}{ 2 ( 1 - S_{12} ) } ( 1 - S_{12} - S_{21} + 1 ) = 1,
	\end{align*}
	and the orthogonality between $\psi_1$ and $\psi_2$,
	\begin{align*}
		\langle \psi_1 | \psi_2 \rangle &= \int \dif \bfr \, \psi^*_1( \bfr ) \psi_2( \bfr ) = \int \dif \bfr \,  \frac{1}{ \sqrt{ 2 ( 1 + S_{12} ) } } \left( \phi_1(\bfr) + \phi_2(\bfr) \right)^* \frac{1}{ \sqrt{ 2 ( 1 - S_{12} ) } } \left( \phi_1(\bfr) - \phi_2(\bfr) \right) \notag \\
		&= \frac{1}{ 2 \sqrt{  ( 1 - (S_{12})^2 ) } } ( 1 - S_{12} + S_{21} - 1 ) = 0 = \langle \psi_2 | \psi_1 \rangle^*.
	\end{align*}	
	
	Thus, we conclude that $\psi_1$ and $\psi_2$ form an orthonormal set.
	\end{solution}
	
	\subsection{Excited Determinants}
	
	\subsection{Form of the Exact Wave Function and Configuration Interaction}
	
	% 2.7
	\begin{exercise}
	A minimal basis set for benzene consists of 72 spin orbitals. Calculate the size of the full CI matrix if it would be formed from determinants. How many singly excited determinants are there? How many doubly excited determinants are there?
	\end{exercise}
	
	\begin{solution}
	
	To begin with, a benzene molecule consists of 6 carbon atoms, each contributing 6 electrons, and 6 hydrogen atoms, each contributing 1 electron. Consequently, the total number of electrons in a benzene molecule is calculated as $6 \times 6 + 6 \times 1 = 42$ electrons. Namely, $N=42$.
	
	Secondly, the minimal basis set of benzene includes 36 spatial orbitals. Each carbon atom provides its $1s$, $2s$, and three $2p$ orbitals (i.e., $2p_x$, $2p_y$, and $2p_z$), while each hydrogen atom contributes its $1s$ orbital. These 36 spatial orbitals can be used to construct 72 spin orbitals. Namely, $2K=72$.
	
	Thus, there are $\binom{72}{42}$ = 164307576757973059488 determinants in full CI calculation. Besides, there are $\binom{42}{1} \binom{30}{1}$ = 1260 singly excited determinants and $\binom{42}{2} \binom{30}{2}$ = 374535 doubly excited determinants.
	\end{solution}
	
	\section{Operators and Matrix Elements}
	
	\subsection{Minimal Basis \texorpdfstring{$\ce{H2}$}- Matrix Elements}
	
	% 2.8
	\begin{exercise}
	Show that
	\[
		\langle \Psi^{34}_{12} | \mathscr{O}_1 | \Psi^{34}_{12} \rangle = \langle 3 | h | 3 \rangle + \langle 4 | h | 4 \rangle
	\]
	and
	\[
		\langle \Psi_0 | \mathscr{O}_1 | \Psi^{34}_{12} \rangle = \langle \Psi^{34}_{12} | \mathscr{O}_1 | \Psi_0 \rangle = 0.
	\]
	\end{exercise}
	
	\begin{solution}
	For $\langle \Psi_{12}^{34} | \mathscr{O}_1 | \Psi_{12}^{34} \rangle$, it can be divided into two parts, too, viz.,
	\[
		\langle \Psi_{12}^{34} | \mathscr{O}_1 | \Psi_{12}^{34} \rangle = \langle \Psi_{12}^{34} | h(1) | \Psi_{12}^{34} \rangle + \langle \Psi_{12}^{34} | h(2) | \Psi_{12}^{34} \rangle.
	\]
	Its first part is
	\begin{align*}
		&\hspace{1.4em}\langle \Psi_{12}^{34} | h(1) | \Psi_{12}^{34} \rangle \\
		&=  \int \dif \bfx_1 \int \dif \bfx_2 \, \frac{1}{\sqrt{2}}[ \chi_3(1) \chi_4(2) - \chi_4(1) \chi_3(2) ]^* h(1) \frac{1}{\sqrt{2}}[ \chi_3(1) \chi_4(2) - \chi_4(1) \chi_3(2) ] \\
		&= \frac{1}{2} \left[ \int \dif \bfx_1 \, \chi_3^*(1) h(1) \chi_3(1) \int \dif \bfx_2 \, \chi^*_4(2) \chi_4(2) - \int \dif \bfx_1 \, \chi_3^*(1) h(1) \chi_4(1) \int \dif \bfx_2 \, \chi^*_4(2) \chi_3(2) \right. \\
		&\hspace{2em} \left. - \int \dif \bfx_1 \, \chi_4^*(1) h(1) \chi_3(1) \int \dif \bfx_2 \, \chi^*_3(2) \chi_4(2) + \int \dif \bfx_1 \, \chi_4^*(1) h(1) \chi_4(1) \int \dif \bfx_2 \, \chi^*_3(2) \chi_3(2) \right] \\
		&= \frac{1}{2} \left[ \int \dif \bfx_1 \, \chi_3^*(1) h(1) \chi_3(1) + 0 + 0 + \int \dif \bfx_1 \, \chi_4^*(1) h(1) \chi_4(1) \right] = \frac{1}{2} \left( \langle 3 | h | 3 \rangle + \langle 4 | h | 4 \rangle \right).
	\end{align*}
	Similarly, we can obtain that
	\[
		\langle \Psi_{12}^{34} | h(2) | \psi_{12}^{34} \rangle = \frac{1}{2} \left( \langle 3 | h | 3 \rangle + \langle 4 | h | 4 \rangle \right).
	\]
	Thus,
	\begin{sequation}
		\langle \Psi_{12}^{34} | \mathscr{O}_1 | \Psi_{12}^{34} \rangle = \frac{1}{2} \left( \langle 3 | h | 3 \rangle + \langle 4 | h | 4 \rangle \right) + \frac{1}{2} \left( \langle 3 | h | 3 \rangle + \langle 4 | h | 4 \rangle \right) = \langle 3 | h | 3 \rangle + \langle 4 | h | 4 \rangle .
	\end{sequation}
	Besides, in the same way, we obtain that
	\begin{align*}
		&\hspace{1.4em}\langle \Psi_{12}^{34} | h(1) | \Psi_0 \rangle \\
		&=  \int \dif \bfx_1 \int \dif \bfx_2 \, \frac{1}{\sqrt{2}}[ \chi_3(1) \chi_4(2) - \chi_4(1) \chi_3(2) ]^* h(1) \frac{1}{\sqrt{2}}[ \chi_1(1) \chi_2(2) - \chi_2(1) \chi_1(2) ] \\
		&= \frac{1}{2} \left[ \int \dif \bfx_1 \, \chi_3^*(1) h(1) \chi_1(1) \int \dif \bfx_2 \, \chi^*_4(2) \chi_2(2) - \int \dif \bfx_1 \, \chi_3^*(1) h(1) \chi_2(1) \int \dif \bfx_2 \, \chi^*_4(2) \chi_1(2) \right. \\
		&\hspace{2em} \left. - \int \dif \bfx_1 \, \chi_4^*(1) h(1) \chi_1(1) \int \dif \bfx_2 \, \chi^*_3(2) \chi_2(2) + \int \dif \bfx_1 \, \chi_4^*(1) h(1) \chi_2(1) \int \dif \bfx_2 \, \chi^*_3(2) \chi_1(2) \right] \\
		&= \frac{1}{2} \left[ 0 - 0 - 0 + 0 \right] = 0.
	\end{align*}
	and
	\[
		\langle \Psi_{12}^{34} | h(2) | \Psi_0 \rangle = 0,
	\]
	Therefore,
	\begin{sequation}
		\langle \Psi_{12}^{34} | h(1) | \Psi_{12}^{34} \rangle = \langle \Psi_{12}^{34} | h(1) | \Psi_0 \rangle + \langle \Psi_{12}^{34} | h(2) | \Psi_0 \rangle = 0 + 0 = 0,
	\end{sequation}
	and
	\begin{sequation}
		\langle \Psi_0 | \mathscr{O}_1 | \Psi^{34}_{12} \rangle = \langle \Psi^{34}_{12} | \mathscr{O}_1 | \Psi_0 \rangle^* = 0^* = 0.
	\end{sequation}
	
	\end{solution}
	
	% 2.9
	\begin{exercise}
	Using the above approach, show that the full CI matrix for minimal basis $\ce{H2}$ is
	\[
		\mathscr{H} = \begin{pmatrix}
			\langle 1 | h | 1 \rangle + \langle 2 | h | 2 \rangle + \langle 12 | 12 \rangle - \langle 12 | 21 \rangle & \langle 12 | 34 \rangle - \langle 12 | 43 \rangle \\
			\langle 34 | 12 \rangle - \langle 34 | 21 \rangle & \langle 3 | h | 3 \rangle + \langle 4 | h | 4 \rangle + \langle 34| 34 \rangle - \langle 34 | 43 \rangle
		\end{pmatrix}.
	\]
	and that it is Hermitian.
	\end{exercise}
	
	\begin{solution}
	From (2.92), we know that
	\[
		\langle \Psi_0 | \mathscr{H} | \Psi_0 \rangle = \langle 1 | h | 1 \rangle + \langle 2 | h | 2 \rangle + \langle 12 | 12 \rangle - \langle 12 | 21 \rangle.
	\]
	For $\langle \Psi_0 | \mathscr{H} | \Psi_{12}^{34} \rangle$, from Exercise 2.8, we know
	\[
		\langle \Psi_0 | \mathscr{O}_1 | \Psi^{34}_{12} \rangle = 0.
	\]
	Besides,
	\begin{align*}
		&\hspace{1.4em}\langle \Psi_0 | \mathscr{O}_2 | \Psi_{12}^{34} \rangle \\
		&=  \int \dif \bfx_1 \int \dif \bfx_2 \, \frac{1}{\sqrt{2}}[ \chi_1(1) \chi_2(2) - \chi_2(1) \chi_1(2) ]^* r^{-1}_{12} \frac{1}{\sqrt{2}}[ \chi_3(1) \chi_4(2) - \chi_4(1) \chi_3(2) ] \\
		&= \frac{1}{2} \left[ \int \dif \bfx_1 \int \dif \bfx_2 \,\chi_1^*(1) \chi^*_2(2) r^{-1}_{12} \chi_3(1) \chi_4(2) - \int \dif \bfx_1 \int \dif \bfx_2 \, \chi_1^*(1) \chi^*_2(2) r^{-1}_{12} \chi_4(1) \chi_3(2) \right. \\
		&\hspace{2em} \left. -\int \dif \bfx_1 \int \dif \bfx_2 \,\chi_2^*(1) \chi^*_1(2) r^{-1}_{12} \chi_3(1) \chi_4(2) + \int \dif \bfx_1 \int \dif \bfx_2 \, \chi_2^*(1) \chi^*_1(2) r^{-1}_{12} \chi_4(1) \chi_3(2) \right] \\
		&= \frac{1}{2} \left( \langle 12 | 34 \rangle - \langle 12 |  43 \rangle - \langle 21 | 34 \rangle + \langle 21 | 43 \rangle \right) = \frac{1}{2} \left( \langle 12 | 34 \rangle - \langle 12 |  43 \rangle - \langle 12 | 43 \rangle + \langle 12 | 34 \rangle \right) \\
		&= \langle 12 | 34 \rangle - \langle 12 | 43 \rangle .
	\end{align*}
	Thus, we know that
	\begin{sequation}
		\langle \Psi_0 | \mathscr{H} | \Psi^{34}_{12} \rangle = \langle \Psi_0 | \mathscr{O}_1 | \Psi^{34}_{12} \rangle + \langle \Psi_0 | \mathscr{O}_2 | \Psi^{34}_{12} \rangle = 0 + \langle 12 | 34 \rangle - \langle 12 |  43 \rangle = \langle 12 | 34 \rangle - \langle 12 | 43 \rangle,
	\end{sequation}
	and
	\begin{sequation}
		\langle \Psi^{34}_{12} | \mathscr{H} | \Psi_0  \rangle = ( \langle \Psi_0 | \mathscr{H} | \Psi^{34}_{12} \rangle )^* = \langle 34 | 12 \rangle - \langle 34 | 21 \rangle.
	\end{sequation}
	
	At last, for $\langle \Psi^{34}_{12} | \mathscr{H} | \Psi^{34}_{12} \rangle$, from Exercise 2.8,
	\[
		\langle \Psi^{34}_{12} | \mathscr{O}_1 | \Psi^{34}_{12} \rangle = \langle 3 | h | 3 \rangle + \langle 4 | h | 4 \rangle.
	\]
	Moreover,
	\begin{align*}
		&\hspace{1.4em}\langle \Psi^{12}_{34} | \mathscr{O}_2 | \Psi_{12}^{34} \rangle \\
		&=  \int \dif \bfx_1 \int \dif \bfx_2 \, \frac{1}{\sqrt{2}}[ \chi_3(1) \chi_4(2) - \chi_4(1) \chi_3(2) ]^* r^{-1}_{12} \frac{1}{\sqrt{2}}[ \chi_3(1) \chi_4(2) - \chi_4(1) \chi_3(2) ] \\
		&= \frac{1}{2} \left[ \int \dif \bfx_1 \int \dif \bfx_2 \,\chi_3^*(1) \chi^*_4(2) r^{-1}_{12} \chi_3(1) \chi_4(2) - \int \dif \bfx_1 \int \dif \bfx_2 \, \chi_3^*(1) \chi^*_4(2) r^{-1}_{12} \chi_4(1) \chi_3(2) \right. \\
		&\hspace{2em} \left. -\int \dif \bfx_1 \int \dif \bfx_2 \,\chi_4^*(1) \chi^*_3(2) r^{-1}_{12} \chi_3(1) \chi_4(2) + \int \dif \bfx_1 \int \dif \bfx_2 \, \chi_4^*(1) \chi^*_3(2) r^{-1}_{12} \chi_4(1) \chi_3(2) \right] \\
		&= \frac{1}{2} \left( \langle 34 | 34 \rangle - \langle 34 |  43 \rangle - \langle 43 | 34 \rangle + \langle 43 | 43 \rangle \right) = \frac{1}{2} \left( \langle 34 | 34 \rangle - \langle 34 |  43 \rangle - \langle 34 | 43 \rangle + \langle 34 | 34 \rangle \right) \\
		&= \langle 34 | 34 \rangle - \langle 34 | 43 \rangle .
	\end{align*}
	Hence,
	\begin{sequation}
		\langle \Psi^{34}_{12} | \mathscr{H} | \Psi^{34}_{12} \rangle = \langle \Psi^{34}_{12} | \mathscr{O}_1 | \Psi^{34}_{12} \rangle + \langle \Psi^{34}_{12} | \mathscr{O}_2 | \Psi^{34}_{12} \rangle = \langle 3 | h | 3 \rangle + \langle 4 | h | 4 \rangle + \langle 34 | 34 \rangle - \langle 34 |  43 \rangle.
	\end{sequation}
	In conclusion, we have proved that
	\begin{sequation}
		\mathscr{H} = \begin{pmatrix}
			\langle 1 | h | 1 \rangle + \langle 2 | h | 2 \rangle + \langle 12 | 12 \rangle - \langle 12 | 21 \rangle & \langle 12 | 34 \rangle - \langle 12 | 43 \rangle \\
			\langle 34 | 12 \rangle - \langle 34 | 21 \rangle & \langle 3 | h | 3 \rangle + \langle 4 | h | 4 \rangle + \langle 34| 34 \rangle - \langle 34 | 43 \rangle
		\end{pmatrix}.
	\end{sequation}
	Obviously, it is Hermitian.
	\end{solution}
	
	\subsection{Notations for One- and Two-Electron Integrals}
	
	\subsection{General Rules for Matrix Elements}
	
	% 2.10
	\begin{exercise}
	Derive Eq.(2.110) from Eq.(2.107).
	\end{exercise}
	
	\begin{solution}
	We derive Eqs.(2.109a, b) firstly, then derive other equations. Note that
	\begin{sequation}
		\langle mm || mm \rangle = \langle mm | mm \rangle - \langle mm | mm \rangle = 0.
	\end{sequation}
	Thus (2.109a) has been proved. Besides, note that
	\begin{sequation}
		\langle mn || mn \rangle = \langle mn | mn \rangle - \langle mn | nm \rangle = \langle nm | nm \rangle - \langle nm | mn \rangle = \langle nm || nm \rangle .
	\end{sequation}
	Thus, (2.109b) has been proved. From Eq.(2.107), with $\langle mm || mm \rangle = 0$, we find that
	\begin{align*}
		\langle K | \mathscr{H} | K \rangle &= \sum_m^N \langle m | h | m \rangle + \frac{1}{2} \sum_m^N \sum_n^N \langle mn ||mn \rangle \\
		&= \sum_m^N \langle m | h | m \rangle + \frac{1}{2}\sum_m^N \sum_{n>m}^N \langle mn || mn \rangle + \frac{1}{2}\sum_n^N \sum_{m>n}^N \langle mn || mn \rangle \\
		&= \sum_m^N \langle m | h | m \rangle + \frac{1}{2}\sum_m^N \sum_{n>m}^N \langle mn || mn \rangle + \frac{1}{2}\sum_m^N \sum_{n>m}^N \langle nm || nm \rangle \\
		&= \sum_m^N \langle m | h | m \rangle + \sum_m^N \sum_{n>m}^N \langle mn || mn \rangle .
	\end{align*}
	The first line of (2.110) has been verified. Then,		
	\begin{align*}
		\langle K | \mathscr{H} | K \rangle &= \sum_m^N \langle m | h | m \rangle + \sum_m^N \sum_{n>m}^N \langle mn || mn \rangle = \sum_m^N \langle m | h | m \rangle + \sum_m^N \sum_{n>m}^N \langle mn | mn \rangle - \langle mn | nm \rangle \\
		&= \sum_m^N [ m | h | m ] + \sum_m^N \sum_{n>m}^N [ mm | nn ] - [ mn | nm ] .
	\end{align*}
	
	\end{solution}
	
	% 2.11
	\begin{exercise}
	If $|K\rangle = |\chi_1 \chi_2 \chi_3 \rangle$ show that
	\[
		\langle K | \mathscr{H} | K \rangle = \langle 1 | h | 1 \rangle + \langle 2 | h | 2 \rangle + \langle 3 | h | 3 \rangle + \langle 12 || 12 \rangle + \langle 13 || 13 \rangle + \langle 23 || 23 \rangle .
	\]
	\end{exercise}
	
	\begin{solution}
	
	From the conclusion of Exercise 2.10, we immediately obtain that
	\[
		\langle K | \mathscr{H} | K \rangle = \langle 1 | h | 1 \rangle + \langle 2 | h | 2 \rangle + \langle 3 | h | 3 \rangle + \langle 12 || 12 \rangle + \langle 13 || 13 \rangle + \langle 23 || 23 \rangle .
	\]
	
	\end{solution}
	
	% 2.12
	\begin{exercise}
	Evaluate the matrix elements that occur in the minimal basis $\ce{H2}$ full CI matrix (Eq.(2.79)) using the rules. Compare with the result obtained in Exercise 2.9.
	\end{exercise}
	
	\begin{solution}
	
	$\langle \Psi_0 | \mathscr{H} | \Psi_0 \rangle$ has been delivered by (2.111) and (2.114). Applying the rules, we get that
	\begin{align*}
		\langle \Psi_0 | \mathscr{H} | \Psi^{34}_{12} \rangle &= \langle \Psi_0 | \mathscr{O}_1 | \Psi^{34}_{12} \rangle + \langle \Psi_0 | \mathscr{O}_2 | \Psi^{34}_{12} \rangle = 0 + \langle 12 || 34 \rangle = \langle 12 || 34 \rangle , \\
		\langle \Psi^{34}_{12} | \mathscr{H} | \Psi_0 \rangle &= \langle \Psi^{34}_{12} | \mathscr{O}_1 | \Psi_0 \rangle + \langle \Psi^{34}_{12} | \mathscr{O}_2 | \Psi_0 \rangle = 0 + \langle 34 || 12 \rangle = \langle 34 || 12 \rangle , \\
		\langle \Psi^{34}_{12} | \mathscr{H} | \Psi^{34}_{12} \rangle &= \langle \Psi^{34}_{12} | \mathscr{O}_1 | \Psi^{34}_{12} \rangle + \langle \Psi^{34}_{12} | \mathscr{O}_2 | \Psi^{34}_{12} \rangle = \langle 3 | h | 3 \rangle + \langle 4 | h | 4 \rangle + \langle 34 || 34 \rangle .
	\end{align*}
	This result equals that of Exercise 2.9.
	
	\end{solution}
	
	% 2.13
	\begin{exercise}
	Show that
	\[ \langle \Psi^r_a| \mathscr{O}_1 | \Psi^s_b \rangle =
	\begin{dcases}
		0, & \text{if } a \neq b , \, r \neq s ; \\ 
		\langle r | h | s \rangle, & \text{if } a = b , \, r \neq s ; \\ 
		-\langle b | h | a \rangle, & \text{if } a \neq b , \, r = s ; \\
		\sum_c^N \langle c | h | c \rangle - \langle a | h | a \rangle + \langle r | h | r \rangle & \text{if } a = b , \, r = s. 
	\end{dcases}
	\]	
	\end{exercise}
	
	\begin{solution}
	
	Note that this result is not correct until the two determinants are in maximum coincidence! We talk about the final result according to the different occupation of the bra and the ket.
	\begin{itemize}
	
	\item When $a \neq b$ and $r \neq s$, the two determinants differ by two spin orbitals, and thus
	\[
		\langle \Psi^r_a| \mathscr{O}_1 | \Psi^s_b \rangle = 0.
	\]
	
	\item When $a = b$ and $r \neq s$, the two determinants differ by one spin orbital, and thus 
	\[
		\langle \Psi^r_a| \mathscr{O}_1 | \Psi^s_b \rangle = \langle \Psi^r_a | \mathscr{O}_1 | \Psi^s_a \rangle = \langle r | h | s \rangle.
	\]
	
	\item When $a \neq b$ and $r = s$, the two determinants differ by one spin orbital, and thus $\langle \Psi^r_a | \mathscr{O}_1 | \Psi^s_b \rangle = \langle \Psi^r_a | \mathscr{O}_1 | \Psi^r_b \rangle$. Note that
	\[
		\langle \Psi^r_a | \mathscr{O}_1 | \Psi^r_b \rangle = \langle \cdots \chi_r \chi_b \cdots | \mathscr{O}_1 | \cdots \chi_a \chi_r \cdots \rangle = -\langle \cdots \chi_r \chi_b \cdots | \mathscr{O}_1 | \cdots \chi_r\chi_a  \cdots \rangle = -\langle r | h | a \rangle.
	\]

	\item When $a = b$ and $r = s$, the two determinants are the same, and thus
	\[
		\langle \Psi^r_a | \mathscr{O}_1 | \Psi^s_b \rangle = \langle \Psi^r_a | \mathscr{O}_1 | \Psi^r_a \rangle = \langle \cdots \chi_{a \rightarrow r} \cdots | \mathscr{O}_1 | \cdots \chi_{a \rightarrow r}  \cdots \rangle = \sum_{c\neq a} \langle c | h | c \rangle + \langle r | h | r \rangle .
	\]
	In other words,
	\[
		\langle \Psi^r_a | \mathscr{O}_1 | \Psi^s_b \rangle = \sum_{c} \langle c | h | c \rangle - \langle a | h | a \rangle + \langle r | h | r \rangle.
	\]

	\end{itemize}
	
	In conclusion, we get that
	\begin{sequation}
		\langle \Psi^r_a| \mathscr{O}_1 | \Psi^s_b \rangle =
		\begin{dcases}
			0, & \text{if } a \neq b , \, r \neq s ; \\ 
			\langle r | h | s \rangle, & \text{if } a = b , \, r \neq s ; \\ 
			-\langle b | h | a \rangle, & \text{if } a \neq b , \, r = s ; \\
		\sum_c^N \langle c | h | c \rangle - \langle a | h | a \rangle 	+ \langle r | h | r \rangle & \text{if } a = b , \, r = s. 
		\end{dcases}	
	\end{sequation}
	
	\end{solution}
	
	% 2.14
	\begin{exercise}
	The Hartree-Fock ground state energy for an $N$-electron system is $^N E_0 = \langle ^N \Psi_0 | \mathscr{H} | ^N \Psi_0 \rangle$. Consider a state of the ionized system (in which an electron has been removed from spin orbital $\chi_a$) with energy $^{N-1} E_a = \langle ^{N-1} \Psi_a | \mathscr{H} | ^{N-1} \Psi_a \rangle$, where $| ^{N-1} \Psi_a \rangle$ is a single determinant with all spin orbitals but $\chi_a$ occupied,
	\[
		| ^{N-1} \Psi_a \rangle = | \chi_1 \chi_2 \cdots \chi_{a-1} \chi_{a+1} \cdots \chi_N \rangle.
	\]
	Show, using the rules in the tables, that the energy required for this ionization process is
	\[
		^N E_0 - ^{N-1} E_a = \langle a | h | a \rangle + \sum_b^N \langle ab || ab \rangle.
	\]
	%To show the power and simplicity of the mnemonic device introduced in this subsection, let us derive the above result without doing any algebra. Consider the representation of $|^N \Psi_0 \rangle$ in Fig. 2.4. If we remove an electron from $\chi_a$, we lose the ``one-electron energy" contribution $\langle a | h | a \rangle$ to $^N E_0$. Moreover, we lose the pair-wise contributions arising from the ``interaction" of the electron in $\chi_a$ with the remaining electrons $\displaystyle \left( \text{i.e.,} \sum_{b \neq a}^N \langle ab || ab \rangle \right)$. Because $\langle aa || aa \rangle = 0$, the above result follows immediately.
	\end{exercise}
	
	\begin{solution}
	
	The verification is direct.
	\begin{align*}
		^N E_0 - ^{N-1} E_0 &= \left[ \sum^N_b \langle b | h | b \rangle + \frac{1}{2} \sum^N_c \sum^N_d \langle cd || cd \rangle \right] - \left[ \sum^N_{b \neq a}  \langle b | h | b \rangle + \frac{1}{2} \sum^N_{c \neq a} \sum^N_{d \neq a} \langle cd ||cd \rangle \right] \\
		&= \left[ \sum^N_b \langle b | h | b \rangle - \sum^N_{b \neq a} \langle b | h | b \rangle \right] + \frac{1}{2} \left[  \sum^N_c \sum^N_d \langle cd || cd \rangle - \sum^N_{c \neq a} \sum^N_{d \neq a} \langle cd ||cd \rangle \right] \\
		&= \langle a | h | a \rangle + \frac{1}{2} \left[ \sum^N_d \langle ad || ad \rangle + \sum^N_c \langle ca || ca \rangle - \langle aa || aa \rangle \right] \\
		&= \langle a | h | a \rangle + \frac{1}{2} \left[ \sum^N_b \langle ab || ab \rangle + \sum^N_b \langle ab || ab \rangle \right] = \langle a | h | a \rangle + \sum^N_b \langle ab ||ab \rangle.
	\end{align*}
	
	\end{solution}
	
	\subsection{Derivation of the Rules for Matrix Elements}
	
	% 2.15
	\begin{exercise}
	Generalize the result of Exercise 2.4 to $N$-electron Slater determinants. Show that the Slater determinant $| \chi_i \chi_j \cdots \chi_k \rangle$ formed from spin orbitals, which are eigenfunctions of the one-electron operator $h$ as in Eq.(2.29), is an eigenfunction of the independent-electron Hamiltonian (2.28), $\mathscr{H} = \sum_{i=1}^N h(i)$, with an eigenvalue $\varepsilon_i + \varepsilon_j + \cdots + \varepsilon_k$. {\it Hint}: Since $\mathscr{H}$ is invariant to permutations of the electron labels, it commutes with the permutation operator $\mathscr{P}_n$.
	\end{exercise}
	
	\begin{solution}
	
	Note that a Slater determinant is a linear combination of $N!$ Hartree product. With the conclusion of Exercise 2.2 and the truth that $\mathscr{H}$ commutes with the permutation operator $\mathscr{P}_n$, we find that
	\begin{align*}
		\mathscr{H} \Psi^\HP &= \mathscr{H} \frac{1}{\sqrt{N!}} \sum_{ n }^{ N!} (-1)^{p_n} \mathscr{P}_n \prod_{ j=1 }^N \chi_{j^\prime}( \bfx_j ) = \frac{1}{\sqrt{N!}} \sum_{ n }^{ N!} (-1)^{p_n} \mathscr{P}_n \mathscr{H} \prod_{ j=1 }^N \chi_{j^\prime}( \bfx_j ) \\
		&= \frac{1}{\sqrt{N!}} \sum_{ n }^{ N!} (-1)^{p_n} \mathscr{P}_n \sum_{ i=1 }^N \varepsilon_i \prod_{ j=1 }^N \chi_{j^\prime}( \bfx_j ) = \sum_{ i=1 }^N \varepsilon_i \frac{1}{\sqrt{N!}} \sum_{ n }^{ N!} (-1)^{p_n} \mathscr{P}_n \prod_{ j=1 }^N \chi_{j^\prime}( \bfx_j ) = \sum_{ i=1 }^N \varepsilon_i \Psi^\HP .
	\end{align*}
	
	\end{solution}
	
	% 2.16
	
	
	
	
%	由练习2.15. $ \mathscr{H}|\chi_i\chi_j \cdots \chi_k \rangle = \Big( \sum_{i=1}^N \varepsilon_i \Big) | \chi_i \chi_j \cdots \chi_k \rangle $, 
%		
%		由练习2.2. $ \mathscr{H} |K^{HP}\rangle = \Big( \sum_{i=1}^N \varepsilon_i \Big) |K^{HP}\rangle $.
%		\[
%		\begin{aligned}
%			\therefore \langle K | \mathscr{H} | L \rangle =& \sum_{i=1}^N \varepsilon_i \langle K | L \rangle \\
%			=& \sum_{i=1}^N \varepsilon_i \frac{1}{\sqrt{N!}} \sum_{i_1i_2\cdots i_N} (-1)^{\tau(i_1i_2\cdots i_N)}\langle i_1i_2\cdots i_N | L \rangle \\
%			=& \sum_{i=1}^N \varepsilon_i \frac{1}{\sqrt{N!}} \sum_{i_1i_2\cdots i_N} (-1)^{\tau(i_1i_2\cdots i_N)} \delta_{i_1i_2\cdots i_N}^L.
%		\end{aligned}
%		\]
%		
%		若$|K^{HP}\rangle$与$|L\rangle$不相同, 即$\langle K^{HP} | L \rangle = 0$. 则由相同种类轨道构成的$|K\rangle$必然有$\langle K|L\rangle = 0$; 反之, 若$\langle K^{HP}|L\rangle = 1$或$\langle K^{HP}|L\rangle = -1$, 则$ \sum_{i_1i_2\cdots i_N} (-1)^{\tau(i_1i_2\cdots i_N)}\delta_{i_1i_2\cdots i_N}^L $中有且仅有一项为1, 其余为0,从而
%		\[
%			\lrs{K}{\mathscr{H}}{L} = \frac{1}{\sqrt{N!}} \sum_{i=1}^N \varepsilon_i \langle K_{i}^{HP} | L \rangle = \frac{1}{\sqrt{N!}} \langle K^{HP} | L \rangle.
%		\]
%		
%		证明单电子积分的运算规则的可参考个人读后感2.3.4节,有详细同思路证明。
	
%	% 2.17
%	\[
%		\begin{aligned}
%			&H(1,1) = \la 1|h|1 \ra + \la 2|h|2 \ra + \la 12|12 \ra + \la 12 | 21\ra \\
%			&= [1|h|1] + [\bar 1|h|\bar 1] + [11|\bar1 \bar1] + [1\bar1 | \bar11] \\
%			&= (1|h|1) + (1|h|1) + (11|11) = 2(1|h|1) + (11|11).
%		\end{aligned}
%		\]
%		\[
%			H(1,2) = \la12|34\ra - \la12|43\ra = [12|\bar1\bar2] - [1\bar2|\bar12] = (12|12).
%		\]
%		\[
%			H(2,1) = \la34|12\ra - \la34|21\ra = [21|\bar2\bar1] - [2\bar1|\bar21] = (21|21).
%		\]
%		\[
%		\begin{aligned}
%			&H(2,2) = \la3|h|3\ra + \la4|h|4\ra + \la34|34\ra - \la34|43\ra \\
%			&= [2|h|2] + [\bar2|h|\bar2] + [22|\bar2\bar2] - [2\bar2|\bar22] \\
%			&= (2|h|2) + (2|h|2) + (22|22) = 2(2|h|2) + (22|22).
%		\end{aligned}
%		\]
%
%
%	2.18
%	类比于教材中将电子傀标在闭壳层结构下折半为电子对傀标,我也如此处理。此外,由于对闭壳层系统,同一能级上电子能量(不论其自旋)相同,即成立$\varepsilon_i = \varepsilon_{i'}$,从而求和范围折半后,和表达式的分母不变,从而在不改变结果的情况下,我将求和过程中出现的分母略去不写,以使过程简便,在最后结果中再加上它。那么,先分析求和范围变化,原求和裂为$2^4 = 16$组求和,而由求和表达式
	
	

\end{document}