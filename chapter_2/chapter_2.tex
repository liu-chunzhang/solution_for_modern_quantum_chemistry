\documentclass[a4paper]{book}
\special{dvipdfmx:config z 0} %取消PDF压缩,加快速度,最终版本生成的时候最好把这句话注释掉

\usepackage{amssymb}
\usepackage[hypcap=false]{caption}
\usepackage{geometry}
\geometry{
	left=2cm,
	right=2cm,
	top=2cm,
	bottom=2cm,
}
\usepackage{hyperref}
\hypersetup{
    colorlinks=true,            %链接颜色
    linkcolor=blue,             %内部链接
    filecolor=magenta,          %本地文档
    urlcolor=cyan,              %网址链接
}
\usepackage[none]{hyphenat}		% 阻止长单词分在两行
\usepackage{mathrsfs}
\usepackage{mathtools}
\usepackage[version=4]{mhchem}
\usepackage{subcaption}
\usepackage{titlesec}

\RequirePackage[many]{tcolorbox}
\tcbset{
    boxed title style={colback=magenta},
	breakable,
	enhanced,
	sharp corners,
	attach boxed title to top left={yshift=-\tcboxedtitleheight,  yshifttext=-.75\baselineskip},
	boxed title style={boxsep=1pt,sharp corners},
    fonttitle=\bfseries\sffamily,
}

\definecolor{skyblue}{rgb}{0.54, 0.81, 0.94}

\newcounter{exercise}[chapter]
\newcounter{solution}[chapter]
\newcounter{eqs}[solution]

\newenvironment{sequation}
  {\begin{equation}\stepcounter{eqs}\tag{\thesolution-\theeqs}}
  {\end{equation}}

\newtcolorbox[use counter=exercise, number within=chapter, number format=\arabic]{exercise}[1][]{
    title={Exercise~\thetcbcounter},
    colframe=skyblue,
    colback=skyblue!12!white,
    boxed title style={colback=skyblue},
    overlay unbroken and first={
        \node[below right,font=\small,color=skyblue,text width=.8\linewidth]
        at (title.north east) {#1};
    }
    label={\unskip},
    before upper={
        \phantomsection
        \addcontentsline{toc}{subsubsection}{Exercise\hspace{1em}\thetcbcounter}
    },
}

\newtcolorbox[use counter=solution, number within=chapter, number format=\arabic]{solution}[1][]{
    title={Solution~\thetcbcounter},
    colframe=teal!60!green,
    colback=green!12!white,
    boxed title style={colback=teal!60!green},
    overlay unbroken and first={
        \node[below right,font=\small,color=red,text width=.8\linewidth]
        at (title.north east) {#1};
    }
}

% special new commands for common symbols used in the article
\newcommand\lr[2]{\langle#1\|#2\rangle}
\newcommand\tr[1]{\mathrm{tr(#1)}}
\newcommand*{\dif}{\mathop{}\!\mathrm{d}}
\renewcommand\det[1]{\mathrm{det\left(#1\right)}}
\newcommand{\bfr}{{\bf r}}
\newcommand{\bfx}{{\bf x}}
\newcommand{\HP}{{\rm HP}}

\titleformat{\chapter}[display]
  {\bfseries\Large}
  {\filright\MakeUppercase{\chaptertitlename} \Huge\thechapter}
  {1ex}
  {\titlerule\vspace{1ex}\filleft}
  [\vspace{1ex}\titlerule]
 

\newcommand\Figref[1]{Fig \ref{#1}}
\newcommand\Tableref[1]{Table \ref{#1}}
 
\allowdisplaybreaks

\begin{document}

	\stepcounter{chapter}

	\chapter{Many Electron Wave Functions and Operators}
	
	\section{The Electron Problem}
	
	\subsection{Atomic Units}
	
	\subsection{The Born-Oppenheimer Approximation}
	
	\subsection{The Antisymmetry or Pauli Exclusion Principle}
	
	\section{Orbitals, Slater Determinants, and Basis Functions}
	
	\subsection{Spin Orbitals and Spatial Orbitals}
	
	% 2.1
	\begin{exercise}
	Given a set of $K$ orthonormal spatial functions, $\{\psi^\alpha_i(\bfr)\}$, and another set of $K$ orthonormal functions, $\{\psi^\beta_i(\bfr)\}$, such that the first set is not orthogonal to the second set, i.e., 
	\[
		\int \dif \bfr \, \psi^{\alpha*}_i( \bfr ) \psi^\beta_j( \bfr ) = S_{ij}
	\]
	where ${\bf S}$ is an overlap matrix, show that the set $\{ \chi_i \}$ of $2K$ spin orbitals, formed by multiplying $\psi^\alpha_i( \bfr )$ by the $\alpha$ spin function and $\psi^\beta_i( \bfr )$ by the $\beta$ spin function, i.e.,
	\[ \left.
	\begin{rgathered}
		\chi_{2i-1}(\bfx) = \psi^\alpha_i( \bfr ) \alpha( \omega ) \quad \\
		\chi_{2i}(\bfx) = \psi^\beta_i( \bfr ) \beta( \omega ) \quad
	\end{rgathered} \right\} i = 1, 2, \ldots, K
	\]
	is an orthonormal set.
	\end{exercise}
	
	\begin{solution}
	It is easy to verify the normalization of any $\chi_{2i-1}$ or $\chi_{2j}$, where $i = 1,2,\ldots,K$ and $j = 1,2,\ldots, K$,
	\begin{align*}
		\langle \chi_{2i-1} | \chi_{2j-1} \rangle &= \int \dif \bfx \, \chi^*_{2i-1}( \bfx ) \chi_{2j-1}( \bfx ) = \int \dif \bfr \, \psi^{\alpha*}_i( \bfr ) \psi^\alpha_j( \bfr ) \int \dif \omega \, \alpha^*( \omega ) \alpha( \omega) = \delta_{ij} \times 1 = \delta_{ij} , \\
		\langle \chi_{2i} | \chi_{2j} \rangle &= \int \dif \bfx \, \chi^*_{2i}( \bfx ) \chi_{2j}( \bfx ) = \int \dif \bfr \, \psi^{\beta*}_i( \bfr ) \psi^\beta_j( \bfr ) \int \dif \omega \, \beta^*( \omega ) \beta( \omega) = \delta_{ij} \times 1 = \delta_{ij}.
	\end{align*}
	and the orthogonality between $\chi_{2i-1}$ and $\chi_{2j}$, where $i = 1,2,\ldots,K$ and $j = 1,2,\ldots, K$,
	\begin{align*}
		\langle \chi_{2i-1} | \chi_{2j} \rangle &= \int \dif \bfx \, \chi^*_{2i-1}( \bfx ) \chi_{2j}( \bfx ) = \int \dif \bfr \, \psi^{\alpha*}_i( \bfr ) \psi^\beta_j( \bfr ) \int \dif \omega \, \alpha^*( \omega ) \beta( \omega) = S_{ij} \times 0 = 0 , \\
		\langle \chi_{2i} | \chi_{2j-1} \rangle &= \int \dif \bfx \, \chi^*_{2i}( \bfx ) \chi_{2j-1}( \bfx ) = \int \dif \bfr \, \psi^{\beta*}_i( \bfr ) \psi^\alpha_j( \bfr ) \int \dif \omega \, \beta^*( \omega ) \alpha( \omega) = S^*_{ji} \times 0 = 0 .
	\end{align*}	
	
	Thus, we can that the set $\{ \chi_i \}$ of $2K$ spin orbitals is an orthonormal set.	
	\end{solution}
	
	\subsection{Hartree Products}
	
	% 2.2
	\begin{exercise}
	Show that the Hartree product of (2.30) is an eigenfunction of $\mathscr{H} = \displaystyle\sum_{i=1}^N h(i)$ with an eigenvalue given by (2.32).
	\end{exercise}
	
	\begin{solution}
	
	The verification is easy. With (2.29), we find that
	\begin{sequation}
		\mathscr{H} \Psi^\HP = \left( \sum_{i=1}^N h(i) \right) \left[ \prod_{ j=1 }^N \chi_{j^\prime}( \bfx_j ) \right] = \sum_{i=1}^N \prod_{ j=1 }^N h(i) \chi_{j^\prime}( \bfx_j ) = \sum_{i=1}^N \prod_{ j=1 }^N \varepsilon_i \chi_{j^\prime}( \bfx_j ) = \sum_{i=1}^N \varepsilon_i \prod_{ j=1 }^N  \chi_{j^\prime}( \bfx_j ).
	\end{sequation}		
	
	\end{solution}
	
	\subsection{Slater Determinants}
	
	% 2.3	
	\begin{exercise}
	Show that $\Psi({\bf x}_1,{\bf x}_2)$ of Eq.(2.34) is normalized.
	\end{exercise}
	
	\begin{solution}
	The verification is direct, viz.,
	\begin{align*}
		&\hspace{1.4em}\langle \Psi | \Psi \rangle = \int \dif \vec{\bfx} \, \langle \Psi | \vec{\bfx} \rangle \langle \vec{\bfx} | \Psi \rangle \\
		&= \int \dif \bfx_1 \int \dif \bfx_2 \, \frac{1}{\sqrt{2}} \left[ \chi_i(\bfx_1) \chi_j(\bfx_2) - \chi_j(\bfx_1) \chi_i(\bfx_2) \right]^* \frac{1}{\sqrt{2}} \left[ \chi_i(\bfx_1) \chi_j(\bfx_2) - \chi_j(\bfx_1) \chi_i(\bfx_2) \right] \\
		&= \frac{1}{2} \left( \int \dif \bfx_1 \int \dif \bfx_2 \chi^*_i(\bfx_1) \chi^*_j(\bfx_2) \chi_i(\bfx_1) \chi_j(\bfx_2) - \int \dif \bfx_1 \int \dif \bfx_2 \chi^*_i(\bfx_1) \chi^*_j(\bfx_2) \chi_j(\bfx_1) \chi_i(\bfx_2) \right. \\
		&\hspace{4em} \left. - \int \dif \bfx_1 \int \dif \bfx_2  \chi_j^*(\bfx_1) \chi_i^*(\bfx_2) \chi_i(\bfx_1) \chi_j(\bfx_2) + \int \dif \bfx_1 \int \dif \bfx_2 \chi_j^*(\bfx_1) \chi_i^*(\bfx_2) \chi_j(\bfx_1) \chi_i(\bfx_2) \right) \\
		&= \frac{1}{2} ( 1 - 0 - 0 + 1 ) = 1.
	\end{align*}
	
	\end{solution}
	
	% 2.4
	\begin{exercise}
	Suppose the spin orbitals $\chi_i$ and $\chi_j$ are eigenfunctions of a one-electron operator $h$ with eigenvalues $\varepsilon_i$ and $\varepsilon_j$ as in Eq.(2.29). Show that the Hartree products in Eqs.(2.33a, b) and the antisymmetrized wave function in Eq.(2.34) are eigenfunctions of the independent-particle Hamiltonian $\mathscr{H} = h(1) + h(2)$ (c.f. Eq.(2.28)) and have the same eigenvalue namely, $\varepsilon_i + \varepsilon_j$. 
	\end{exercise}
	
	\begin{solution}
	Firstly, we check the Hartree products of $\chi_i$ and $\chi_j$. With the conclusion of Exercise 2.2, we get that
	\begin{align*}
		\mathscr{H} | \Psi^\HP_{12} \rangle &= ( \varepsilon_i + \varepsilon_j ) | \Psi^\HP_{21} \rangle, \\
		\mathscr{H} | \Psi^\HP_{21} \rangle &= ( \varepsilon_j + \varepsilon_i ) | \Psi^\HP_{21} \rangle = ( \varepsilon_i + \varepsilon_j ) | \Psi^\HP_{21} \rangle.
	\end{align*}
	Thus, the eigenvalue of the Hartree product of $\chi_i$ and $\chi_j$ is irrelevant to their order. Note that
	\[
		\Psi = \frac{1}{\sqrt{2}} \left[ \chi_i(\bfx_1) \chi_j(\bfx_2) - \chi_j(\bfx_1) \chi_i(\bfx_2) \right] = \frac{1}{\sqrt{2}} \left( \Psi^\HP_{12} - \Psi^\HP_{21} \right),
	\]
	we find that
	\begin{align*}
		\mathscr{H} | \Psi \rangle &= \mathscr{H} \frac{1}{\sqrt{2}} \left( | \Psi^\HP_{12} \rangle - | \Psi^\HP_{21} \rangle \right) = \frac{1}{\sqrt{2}} \left( \mathscr{H} | \Psi^\HP_{12} \rangle - \mathscr{H} | \Psi^\HP_{21} \rangle \right) \\
		&= \frac{1}{\sqrt{2}} \left[ ( \varepsilon_i + \varepsilon_j ) | \Psi^\HP_{12} \rangle - ( \varepsilon_i + \varepsilon_j ) | \Psi^\HP_{21} \rangle \right] \\
		&= ( \varepsilon_i + \varepsilon_j ) \frac{1}{\sqrt{2}} \left( | \Psi^\HP_{12} \rangle - | \Psi^\HP_{21} \rangle \right) = ( \varepsilon_i + \varepsilon_j ) | \Psi \rangle.
	\end{align*}
	Thus, we have proved that the Hartree products in Eqs.(2.33a, b) and the antisymmetrized wave function in Eq.(2.34) are eigenfunctions of the independent-particle Hamiltonian $\mathscr{H} = h(1) + h(2)$ and have the same eigenvalue $\varepsilon_i + \varepsilon_j$. 
	\end{solution}
	
	% 2.5
	\begin{exercise}
	Consider the Slater determinants
	\[
		| K \rangle = | \chi_i \chi_j \rangle, \quad | L \rangle = | \chi_k \chi_l \rangle.
	\]
	Show that
	\[
		\langle K | L \rangle = \delta_{ik} \delta_{jl} - \delta_{il} \delta_{jk}.
	\]
	Note that the overlap is zero unless: 1) $k=i$ and $l=j$, in which case $|L\rangle$ = $|K\rangle$ and the overlap is unity and 2) $k=j$ and $l=i$ in which case $|L\rangle= |\chi_j \chi_i \rangle = -|K\rangle$ and the overlap is minus one.
	\end{exercise}
	
	\begin{solution}
	We calculate the inner product firstly,
	\begin{align*}
		\langle K | L \rangle &= \int \dif \vec{\bfx} \, \langle K | \vec{\bfx} \rangle \langle \vec{\bfx} | L \rangle \\
		&= \int \dif \bfx_1 \int \dif \bfx_2 \, \frac{1}{\sqrt{2}}\left[ \chi_i(\bfx_1) \chi_j(\bfx_2) - \chi_j(\bfx_1) \chi_i(\bfx_2) \right]^* \frac{1}{\sqrt{2}}\left[ \chi_k(\bfx_1) \chi_l(\bfx_2) - \chi_l(\bfx_1) \chi_k(\bfx_2) \right] \\
		&= \frac{1}{2} \left[ \int \dif \bfx_1 \int \dif \bfx_2 \, \chi^*_i(\bfx_1) \chi^*_j(\bfx_2) \chi_k(\bfx_1) \chi_l(\bfx_2) - \int \dif \bfx_1 \int \dif \bfx_2 \, \chi^*_i(\bfx_1) \chi^*_j(\bfx_2) \chi_l(\bfx_1) \chi_k(\bfx_2)\right. \\
		&\hspace{4em} \left. - \int \dif \bfx_1 \int \dif \bfx_2 \, \chi^*_j(\bfx_1) \chi^*_i(\bfx_2) \chi_k(\bfx_1) \chi_l(\bfx_2) + \int \dif \bfx_1 \int \dif \bfx_2 \, \chi^*_j(\bfx_1) \chi^*_i(\bfx_2) \chi_l(\bfx_1) \chi_k(\bfx_2) \right] \\
		&= \frac{1}{2} \left( \delta_{ik} \delta_{jl} - \delta_{il}\delta_{jk} - \delta_{jk} \delta_{il} + \delta_{jl}\delta_{ik}\right) = \delta_{ik}\delta_{jl} - \delta_{jk} \delta_{il}.
	\end{align*}
	The conclusion is obvious. 
	\begin{itemize}
	
	\item When $k=i$ and $l=j$, in which case $|L\rangle$ = $|K\rangle$ and the overlap is $1$.
	
	\item When $k=j$ and $l=i$ in which case $|L\rangle = | \chi_k \chi_l \rangle = |\chi_j \chi_i \rangle = -|K\rangle$ and the overlap is $-1$.
	
	\item Otherwise, the overlap is $0$.
	
	\end{itemize}
	\end{solution}
	
	\subsection{The Hartree-Fock Approximation}
	
	\subsection{The Minimal Basis \texorpdfstring{$\ce{H2}$}- Model}
	
	% 2.6
	\begin{exercise}
	Show that $\psi_1$ and $\psi_2$ form an orthonormal set.
	\end{exercise}
	
	\begin{solution}
	Similar to Exercise 2.1, verify the normalization of $\psi_1$ or $\psi_2$, with $S_{12} = S_{21}$,
	\begin{align*}
		\langle \psi_1 | \psi_1 \rangle &= \int \dif \bfr \, \psi^*_1( \bfr ) \psi_1( \bfr ) = \int \dif \bfr \,  \frac{1}{ \sqrt{ 2 ( 1 + S_{12} ) } } \left( \phi_1(\bfr) + \phi_2(\bfr) \right)^* \frac{1}{ \sqrt{ 2 ( 1 + S_{12} ) } } \left( \phi_1(\bfr) + \phi_2(\bfr) \right) \notag \\
		&= \frac{1}{ 2 ( 1 + S_{12} ) } ( 1 + S_{12} + S_{21} + 1 ) = 1, \\
		\langle \psi_2 | \psi_2 \rangle &= \int \dif \bfr \, \psi^*_2( \bfr ) \psi_2( \bfr ) = \int \dif \bfr \,  \frac{1}{ \sqrt{ 2 ( 1 - S_{12} ) } } \left( \phi_1(\bfr) - \phi_2(\bfr) \right)^* \frac{1}{ \sqrt{ 2 ( 1 - S_{12} ) } } \left( \phi_1(\bfr) - \phi_2(\bfr) \right) \notag \\
		&= \frac{1}{ 2 ( 1 - S_{12} ) } ( 1 - S_{12} - S_{21} + 1 ) = 1,
	\end{align*}
	and the orthogonality between $\psi_1$ and $\psi_2$,
	\begin{align*}
		\langle \psi_1 | \psi_2 \rangle &= \int \dif \bfr \, \psi^*_1( \bfr ) \psi_2( \bfr ) = \int \dif \bfr \,  \frac{1}{ \sqrt{ 2 ( 1 + S_{12} ) } } \left( \phi_1(\bfr) + \phi_2(\bfr) \right)^* \frac{1}{ \sqrt{ 2 ( 1 - S_{12} ) } } \left( \phi_1(\bfr) - \phi_2(\bfr) \right) \notag \\
		&= \frac{1}{ 2 \sqrt{  ( 1 - (S_{12})^2 ) } } ( 1 - S_{12} + S_{21} - 1 ) = 0 = \langle \psi_2 | \psi_1 \rangle^*.
	\end{align*}	
	
	Thus, we conclude that $\psi_1$ and $\psi_2$ form an orthonormal set.
	\end{solution}
	
	\subsection{Excited Determinants}
	
	\subsection{Form of the Exact Wave Function and Configuration Interaction}
	
	% 2.7
	\begin{exercise}
	A minimal basis set for benzene consists of 72 spin orbitals. Calculate the size of the full CI matrix if it would be formed from determinants. How many singly excited determinants are there? How many doubly excited determinants are there?
	\end{exercise}
	
	\begin{solution}
	
	To begin with, a benzene molecule consists of 6 carbon atoms, each contributing 6 electrons, and 6 hydrogen atoms, each contributing 1 electron. Consequently, the total number of electrons in a benzene molecule is calculated as $6 \times 6 + 6 \times 1 = 42$ electrons. Namely, $N=42$.
	
	Secondly, the minimal basis set of benzene includes 36 spatial orbitals. Each carbon atom provides its $1s$, $2s$, and three $2p$ orbitals (i.e., $2p_x$, $2p_y$, and $2p_z$), while each hydrogen atom contributes its $1s$ orbital. These 36 spatial orbitals can be used to construct 72 spin orbitals. Namely, $2K=72$.
	
	Thus, there are $\binom{72}{42}$ = 164307576757973059488 determinants in full CI calculation. Besides, there are $\binom{42}{1} \binom{30}{1}$ = 1260 singly excited determinants and $\binom{42}{2} \binom{30}{2}$ = 374535 doubly excited determinants.
	\end{solution}
	
	\section{Operators and Matrix Elements}
	
	\subsection{Minimal Basis \texorpdfstring{$\ce{H2}$}- Matrix Elements}
	
	% 2.8
	\begin{exercise}
	Show that
	\[
		\langle \Psi^{34}_{12} | \mathscr{O}_1 | \Psi^{34}_{12} \rangle = \langle 3 | h | 3 \rangle + \langle 4 | h | 4 \rangle
	\]
	and
	\[
		\langle \Psi_0 | \mathscr{O}_1 | \Psi^{34}_{12} \rangle = \langle \Psi^{34}_{12} | \mathscr{O}_1 | \Psi_0 \rangle = 0.
	\]
	\end{exercise}
	
	\begin{solution}
	For $\langle \Psi_{12}^{34} | \mathscr{O}_1 | \Psi_{12}^{34} \rangle$, it can be divided into two parts, too, viz.,
	\[
		\langle \Psi_{12}^{34} | \mathscr{O}_1 | \Psi_{12}^{34} \rangle = \langle \Psi_{12}^{34} | h(1) | \Psi_{12}^{34} \rangle + \langle \Psi_{12}^{34} | h(2) | \Psi_{12}^{34} \rangle.
	\]
	Its first part is
	\begin{align*}
		&\hspace{1.4em}\langle \Psi_{12}^{34} | h(1) | \Psi_{12}^{34} \rangle \\
		&=  \int \dif \bfx_1 \int \dif \bfx_2 \, \frac{1}{\sqrt{2}}[ \chi_3(1) \chi_4(2) - \chi_4(1) \chi_3(2) ]^* h(1) \frac{1}{\sqrt{2}}[ \chi_3(1) \chi_4(2) - \chi_4(1) \chi_3(2) ] \\
		&= \frac{1}{2} \left[ \int \dif \bfx_1 \, \chi_3^*(1) h(1) \chi_3(1) \int \dif \bfx_2 \, \chi^*_4(2) \chi_4(2) - \int \dif \bfx_1 \, \chi_3^*(1) h(1) \chi_4(1) \int \dif \bfx_2 \, \chi^*_4(2) \chi_3(2) \right. \\
		&\hspace{2em} \left. - \int \dif \bfx_1 \, \chi_4^*(1) h(1) \chi_3(1) \int \dif \bfx_2 \, \chi^*_3(2) \chi_4(2) + \int \dif \bfx_1 \, \chi_4^*(1) h(1) \chi_4(1) \int \dif \bfx_2 \, \chi^*_3(2) \chi_3(2) \right] \\
		&= \frac{1}{2} \left[ \int \dif \bfx_1 \, \chi_3^*(1) h(1) \chi_3(1) + 0 + 0 + \int \dif \bfx_1 \, \chi_4^*(1) h(1) \chi_4(1) \right] = \frac{1}{2} \left( \langle 3 | h | 3 \rangle + \langle 4 | h | 4 \rangle \right).
	\end{align*}
	Similarly, we can obtain that
	\[
		\langle \Psi_{12}^{34} | h(2) | \psi_{12}^{34} \rangle = \frac{1}{2} \left( \langle 3 | h | 3 \rangle + \langle 4 | h | 4 \rangle \right).
	\]
	Thus,
	\begin{sequation}
		\langle \Psi_{12}^{34} | \mathscr{O}_1 | \Psi_{12}^{34} \rangle = \frac{1}{2} \left( \langle 3 | h | 3 \rangle + \langle 4 | h | 4 \rangle \right) + \frac{1}{2} \left( \langle 3 | h | 3 \rangle + \langle 4 | h | 4 \rangle \right) = \langle 3 | h | 3 \rangle + \langle 4 | h | 4 \rangle .
	\end{sequation}
	Besides, in the same way, we obtain that
	\begin{align*}
		&\hspace{1.4em}\langle \Psi_{12}^{34} | h(1) | \Psi_0 \rangle \\
		&=  \int \dif \bfx_1 \int \dif \bfx_2 \, \frac{1}{\sqrt{2}}[ \chi_3(1) \chi_4(2) - \chi_4(1) \chi_3(2) ]^* h(1) \frac{1}{\sqrt{2}}[ \chi_1(1) \chi_2(2) - \chi_2(1) \chi_1(2) ] \\
		&= \frac{1}{2} \left[ \int \dif \bfx_1 \, \chi_3^*(1) h(1) \chi_1(1) \int \dif \bfx_2 \, \chi^*_4(2) \chi_2(2) - \int \dif \bfx_1 \, \chi_3^*(1) h(1) \chi_2(1) \int \dif \bfx_2 \, \chi^*_4(2) \chi_1(2) \right. \\
		&\hspace{2em} \left. - \int \dif \bfx_1 \, \chi_4^*(1) h(1) \chi_1(1) \int \dif \bfx_2 \, \chi^*_3(2) \chi_2(2) + \int \dif \bfx_1 \, \chi_4^*(1) h(1) \chi_2(1) \int \dif \bfx_2 \, \chi^*_3(2) \chi_1(2) \right] \\
		&= \frac{1}{2} \left[ 0 - 0 - 0 + 0 \right] = 0.
	\end{align*}
	and
	\[
		\langle \Psi_{12}^{34} | h(2) | \Psi_0 \rangle = 0,
	\]
	Therefore,
	\begin{sequation}
		\langle \Psi_{12}^{34} | h(1) | \Psi_{12}^{34} \rangle = \langle \Psi_{12}^{34} | h(1) | \Psi_0 \rangle + \langle \Psi_{12}^{34} | h(2) | \Psi_0 \rangle = 0 + 0 = 0,
	\end{sequation}
	and
	\begin{sequation}
		\langle \Psi_0 | \mathscr{O}_1 | \Psi^{34}_{12} \rangle = \langle \Psi^{34}_{12} | \mathscr{O}_1 | \Psi_0 \rangle^* = 0^* = 0.
	\end{sequation}
	
	\end{solution}
	
	% 2.9
	\begin{exercise}
	Using the above approach, show that the full CI matrix for minimal basis $\ce{H2}$ is
	\[
		\mathscr{H} = \begin{pmatrix}
			\langle 1 | h | 1 \rangle + \langle 2 | h | 2 \rangle + \langle 12 | 12 \rangle - \langle 12 | 21 \rangle & \langle 12 | 34 \rangle - \langle 12 | 43 \rangle \\
			\langle 34 | 12 \rangle - \langle 34 | 21 \rangle & \langle 3 | h | 3 \rangle + \langle 4 | h | 4 \rangle + \langle 34| 34 \rangle - \langle 34 | 43 \rangle
		\end{pmatrix}.
	\]
	and that it is Hermitian.
	\end{exercise}
	
	\begin{solution}
	From (2.92), we know that
	\begin{equation}
		\langle \Psi_0 | \mathscr{H} | \Psi_0 \rangle = \langle 1 | h | 1 \rangle + \langle 2 | h | 2 \rangle + \langle 12 | 12 \rangle - \langle 12 | 21 \rangle.
	\end{equation}
	For $\langle \Psi_0 | \mathscr{H} | \Psi_{12}^{34} \rangle$, from Exercise 2.8, we know
	\[
		\langle \Psi_0 | \mathscr{O}_1 | \Psi^{34}_{12} \rangle = 0.
	\]
	Besides,
	\begin{align*}
		&\hspace{1.4em}\langle \Psi_0 | \mathscr{O}_2 | \Psi_{12}^{34} \rangle \\
		&=  \int \dif \bfx_1 \int \dif \bfx_2 \, \frac{1}{\sqrt{2}}[ \chi_1(1) \chi_2(2) - \chi_2(1) \chi_1(2) ]^* r^{-1}_{12} \frac{1}{\sqrt{2}}[ \chi_3(1) \chi_4(2) - \chi_4(1) \chi_3(2) ] \\
		&= \frac{1}{2} \left[ \int \dif \bfx_1 \int \dif \bfx_2 \,\chi_1^*(1) \chi^*_2(2) r^{-1}_{12} \chi_3(1) \chi_4(2) - \int \dif \bfx_1 \int \dif \bfx_2 \, \chi_1^*(1) \chi^*_2(2) r^{-1}_{12} \chi_4(1) \chi_3(2) \right. \\
		&\hspace{2em} \left. -\int \dif \bfx_1 \int \dif \bfx_2 \,\chi_2^*(1) \chi^*_1(2) r^{-1}_{12} \chi_3(1) \chi_4(2) + \int \dif \bfx_1 \int \dif \bfx_2 \, \chi_2^*(1) \chi^*_1(2) r^{-1}_{12} \chi_4(1) \chi_3(2) \right] \\
		&= \frac{1}{2} \left( \langle 12 | 34 \rangle - \langle 12 |  43 \rangle - \langle 21 | 34 \rangle + \langle 21 | 43 \rangle \right) = \frac{1}{2} \left( \langle 12 | 34 \rangle - \langle 12 |  43 \rangle - \langle 12 | 43 \rangle + \langle 12 | 34 \rangle \right) \\
		&= \langle 12 | 34 \rangle - \langle 12 | 43 \rangle .
	\end{align*}
	Thus, we know that
	\begin{sequation}
		\langle \Psi_0 | \mathscr{H} | \Psi^{34}_{12} \rangle = \langle \Psi_0 | \mathscr{O}_1 | \Psi^{34}_{12} \rangle + \langle \Psi_0 | \mathscr{O}_2 | \Psi^{34}_{12} \rangle = 0 + \langle 12 | 34 \rangle - \langle 12 |  43 \rangle = \langle 12 | 34 \rangle - \langle 12 | 43 \rangle,
	\end{sequation}
	and
	\begin{sequation}
		\langle \Psi^{34}_{12} | \mathscr{H} | \Psi_0  \rangle = ( \langle \Psi_0 | \mathscr{H} | \Psi^{34}_{12} \rangle )^* = \langle 34 | 12 \rangle - \langle 34 | 21 \rangle.
	\end{sequation}
	
	At last, for $\langle \Psi^{34}_{12} | \mathscr{H} | \Psi^{34}_{12} \rangle$, from Exercise 2.8,
	\[
		\langle \Psi^{34}_{12} | \mathscr{O}_1 | \Psi^{34}_{12} \rangle = \langle 3 | h | 3 \rangle + \langle 4 | h | 4 \rangle.
	\]
	Moreover,
	\begin{align*}
		&\hspace{1.4em}\langle \Psi^{12}_{34} | \mathscr{O}_2 | \Psi_{12}^{34} \rangle \\
		&=  \int \dif \bfx_1 \int \dif \bfx_2 \, \frac{1}{\sqrt{2}}[ \chi_3(1) \chi_4(2) - \chi_4(1) \chi_3(2) ]^* r^{-1}_{12} \frac{1}{\sqrt{2}}[ \chi_3(1) \chi_4(2) - \chi_4(1) \chi_3(2) ] \\
		&= \frac{1}{2} \left[ \int \dif \bfx_1 \int \dif \bfx_2 \,\chi_3^*(1) \chi^*_4(2) r^{-1}_{12} \chi_3(1) \chi_4(2) - \int \dif \bfx_1 \int \dif \bfx_2 \, \chi_3^*(1) \chi^*_4(2) r^{-1}_{12} \chi_4(1) \chi_3(2) \right. \\
		&\hspace{2em} \left. -\int \dif \bfx_1 \int \dif \bfx_2 \,\chi_4^*(1) \chi^*_3(2) r^{-1}_{12} \chi_3(1) \chi_4(2) + \int \dif \bfx_1 \int \dif \bfx_2 \, \chi_4^*(1) \chi^*_3(2) r^{-1}_{12} \chi_4(1) \chi_3(2) \right] \\
		&= \frac{1}{2} \left( \langle 34 | 34 \rangle - \langle 34 |  43 \rangle - \langle 43 | 34 \rangle + \langle 43 | 43 \rangle \right) = \frac{1}{2} \left( \langle 34 | 34 \rangle - \langle 34 |  43 \rangle - \langle 34 | 43 \rangle + \langle 34 | 34 \rangle \right) \\
		&= \langle 34 | 34 \rangle - \langle 34 | 43 \rangle .
	\end{align*}
	Hence,
	\begin{sequation}
		\langle \Psi^{34}_{12} | \mathscr{H} | \Psi^{34}_{12} \rangle = \langle \Psi^{34}_{12} | \mathscr{O}_1 | \Psi^{34}_{12} \rangle + \langle \Psi^{34}_{12} | \mathscr{O}_2 | \Psi^{34}_{12} \rangle = \langle 3 | h | 3 \rangle + \langle 4 | h | 4 \rangle + \langle 34 | 34 \rangle - \langle 34 |  43 \rangle.
	\end{sequation}
	In conclusion, we have proved that
	\begin{sequation}
		\mathscr{H} = \begin{pmatrix}
			\langle 1 | h | 1 \rangle + \langle 2 | h | 2 \rangle + \langle 12 | 12 \rangle - \langle 12 | 21 \rangle & \langle 12 | 34 \rangle - \langle 12 | 43 \rangle \\
			\langle 34 | 12 \rangle - \langle 34 | 21 \rangle & \langle 3 | h | 3 \rangle + \langle 4 | h | 4 \rangle + \langle 34| 34 \rangle - \langle 34 | 43 \rangle
		\end{pmatrix}.
	\end{sequation}
	Obviously, it is Hermitian.
	\end{solution}
	
	\subsection{Notations for One- and Two-Electron Integrals}
	
	\subsection{General Rules for Matrix Elements}

\end{document}