\documentclass[a4paper]{book}

\usepackage{amsmath}
\usepackage{amssymb}
\usepackage[hypcap=false]{caption}
\usepackage{enumitem}	% 定制enumerate标号
\usepackage{geometry}
\geometry{
	left=2cm,
	right=2cm,
	top=2cm,
	bottom=2cm,
}
\usepackage{hyperref}
\hypersetup{
    colorlinks=true,            %链接颜色
    linkcolor=blue,             %内部链接
    filecolor=magenta,          %本地文档
    urlcolor=cyan,              %网址链接
}
\usepackage[none]{hyphenat}		% 阻止长单词分在两行
\usepackage{mathrsfs}
\usepackage[version=4]{mhchem}
\usepackage{subcaption}
\usepackage{titlesec}

\RequirePackage[many]{tcolorbox}
\tcbset{
    boxed title style={colback=magenta},
	breakable,
	enhanced,
	sharp corners,
	attach boxed title to top left={yshift=-\tcboxedtitleheight,  yshifttext=-.75\baselineskip},
	boxed title style={boxsep=1pt,sharp corners},
    fonttitle=\bfseries\sffamily,
}

\definecolor{skyblue}{rgb}{0.54, 0.81, 0.94}

\newtcolorbox[auto counter, number within=chapter, number format=\arabic]{exercise}[1][]{
    title={Exercise~\thetcbcounter},
    colframe=skyblue,
    colback=skyblue!12!white,
    boxed title style={colback=skyblue},
    overlay unbroken and first={
        \node[below right,font=\small,color=skyblue,text width=.8\linewidth]
        at (title.north east) {#1};
    }
}

\newtcolorbox[auto counter, number within=chapter, number format=\arabic]{solution}[1][]{
    title={Solution~\thetcbcounter},
    colframe=teal!60!green,
    colback=green!12!white,
    boxed title style={colback=teal!60!green},
    overlay unbroken and first={
        \node[below right,font=\small,color=red,text width=.8\linewidth]
        at (title.north east) {#1};
    }
}

% special new commands for common symbols used in the article
\newcommand\la{\langle}
\newcommand\ra{\rangle}
\newcommand\lr[2]{\langle#1\|#2\rangle}
\newcommand\tr[1]{\mathrm{tr(#1)}}
\newcommand\Tr[3]{#1\mathrm\#2#3}
\newcommand*{\dif}{\mathop{}\!\mathrm{d}}
\renewcommand\det[1]{\mathrm{det\left(#1\right)}}
\newcommand{\HF}{{\rm HF}}

\newcommand{\A}{{\bf A}}
\newcommand{\B}{{\bf B}}
\newcommand{\C}{{\bf C}}
\newcommand{\I}{{\bf 1}}
\newcommand{\U}{{\bf U}}
\newcommand{\Op}{{\bf O}}

\titleformat{\chapter}[display]
  {\bfseries\Large}
  {\filright\MakeUppercase{\chaptertitlename} \Huge\thechapter}
  {1ex}
  {\titlerule\vspace{1ex}\filleft}
  [\vspace{1ex}\titlerule]
  
\allowdisplaybreaks

\begin{document}

	\stepcounter{chapter}

	\chapter{Many Electron Wave Functions and Operators}
	
	\section{The Electron Problem}
	
	\subsection{Atomic Units}
	
	\subsection{The Born-Oppenheimer Approximation}
	
	\subsection{The Antisymmetry or Pauli Exclusion Principle}
	
	\section{Orbitals, Slater Determinants, and Basis Functions}
	
	\subsection{Spin Orbitals and Spatial Orbitals}
	
	\begin{exercise}
	111
	\end{exercise}
	
	\begin{solution}
		2-1 so
	\end{solution}
	
	\subsection{Hartree Products}
	
	\begin{exercise}
	111
	\end{exercise}
	
	\begin{solution}
		2-2 so
	\end{solution}
	
	\subsection{Slater Determinants}
	
	\begin{exercise}
	111
	\end{exercise}
	
	\begin{solution}
		2-3 so
	\end{solution}

	\begin{exercise}
	111
	\end{exercise}
	
	\begin{solution}
		2-4 so
	\end{solution}
	
	\begin{exercise}
	111
	\end{exercise}
	
	\begin{solution}
		2-5 so
	\end{solution}
	
	\subsection{The Hartree-Fock Approximation}
	
	\subsection{The Minimal Basis \texorpdfstring{$\ce{H2}$}- Model}
	
	\begin{exercise}
	Show that $\psi_1$ and $\psi_2$ form an orthonormal set.
	\end{exercise}
	
	\begin{solution}
		2-6 so
	\end{solution}
	
	\subsection{Excited Determinants}
	
	\subsection{Form of the Exact Wave Function and Configuration Interaction}
	
	\begin{exercise}
	111
	\end{exercise}
	
	\begin{solution}
		2-7 so
	\end{solution}
	
	\section{Operators and Matrix Elements}
	
	\subsection{Minimal Basis \texorpdfstring{$\ce{H2}$}- Matrix Elements}
	
	\begin{exercise}
	111
	\end{exercise}
	
	\begin{solution}
		2-8 so
	\end{solution}
	
	\begin{exercise}
	111
	\end{exercise}
	
	\begin{solution}
		2-9 so
	\end{solution}
	
	\subsection{Notations for One- and Two-Electron Integrals}
	
	\subsection{General Rules for Matrix Elements}
	
	\begin{exercise}
	111
	\end{exercise}
	
	\begin{solution}
		2-10 so
	\end{solution}
	
	\begin{exercise}
	111
	\end{exercise}
	
	\begin{solution}
		2-11 so
	\end{solution}
	
	\begin{exercise}
	111
	\end{exercise}
	
	\begin{solution}
		2-12 so
	\end{solution}
	
	\begin{exercise}
	111
	\end{exercise}
	
	\begin{solution}
		2-13 so
	\end{solution}
	
	\begin{exercise}
	111
	\end{exercise}
	
	\begin{solution}
		2-14 so
	\end{solution}
	
	\subsection{Derivation of the Rules for Matrix Elements}
	
	\begin{exercise}
	111
	\end{exercise}
	
	\begin{solution}
		2-15 so
	\end{solution}
	
	\begin{exercise}
	111
	\end{exercise}
	
	\begin{solution}
		2-16 so
	\end{solution}
	
	\subsection{Transition from Spin Orbitals to Spatial Orbitals}
	
	\begin{exercise}
	111
	\end{exercise}
	
	\begin{solution}
		2-17 so
	\end{solution}
	
	\begin{exercise}
	111
	\end{exercise}
	
	\begin{solution}
		2-18 so
	\end{solution}
	
	\subsection{Coulomb and Exchange Integrals}

	\begin{exercise}
	111
	\end{exercise}
	
	\begin{solution}
		2-19 so
	\end{solution}
	
	\begin{exercise}
	Show that for {\it real} spatial orbitals
	\[
		K_{ij} = (ij|ij) = (ji|ji) = \langle ii | jj \rangle = \langle jj | ii \rangle.
	\]
	\end{exercise}
	
	\begin{solution}
		2-20 so
	\end{solution}
	
	\begin{exercise}
	111
	\end{exercise}
	
	\begin{solution}
		2-21 so
	\end{solution}
	
	\begin{exercise}
	111
	\end{exercise}
	
	\begin{solution}
		2-22 so
	\end{solution}
	
	\subsection{Pseudo-Classical Interpretation of Determinantal Energies}
	
	\begin{exercise}
	111
	\end{exercise}
	
	\begin{solution}
		2-23 so
	\end{solution}
	
	\section{Second Quantization}
	
	\subsection{Creation and Annihilation Operators and Their Anticommutation Relations}
	
	\begin{exercise}
	111
	\end{exercise}
	
	\begin{solution}
		2-24 so
	\end{solution}
	
	\begin{exercise}
	111
	\end{exercise}
	
	\begin{solution}
		2-25 so
	\end{solution}
	
	\begin{exercise}
	111
	\end{exercise}
	
	\begin{solution}
		2-26 so
	\end{solution}

	\begin{exercise}
	111
	\end{exercise}
	
	\begin{solution}
		2-27 so
	\end{solution}
	
	\begin{exercise}
	111
	\end{exercise}
	
	\begin{solution}
		2-28 so
	\end{solution}
	
	\subsection{Second-Quantized Operators and Their Matrix Elements}
	
	\begin{exercise}
	111
	\end{exercise}
	
	\begin{solution}
		2-29 so
	\end{solution}
	
	\begin{exercise}
	111
	\end{exercise}
	
	\begin{solution}
		2-30 so
	\end{solution}
	
	\begin{exercise}
	111
	\end{exercise}
	
	\begin{solution}
		2-31 so
	\end{solution}
	
	\section{Spin-Adapted Configurations}
	
	\subsection{Spin Operators}
	
	\begin{exercise}
	111
	\end{exercise}
	
	\begin{solution}
		2-32 so
	\end{solution}
	
	\begin{exercise}
	111
	\end{exercise}
	
	\begin{solution}
		2-33 so
	\end{solution}
	
	\begin{exercise}
	111
	\end{exercise}
	
	\begin{solution}
		2-34 so
	\end{solution}
	
	\begin{exercise}
	111
	\end{exercise}
	
	\begin{solution}
		2-35 so
	\end{solution}
	
	\begin{exercise}
	111
	\end{exercise}
	
	\begin{solution}
		2-36 so
	\end{solution}
	
	\begin{exercise}
	111
	\end{exercise}
	
	\begin{solution}
		2-37 so
	\end{solution}
	
	\subsection{Restricted Determinants and Spin-Adapted Configurations}
	
	\begin{exercise}
	111
	\end{exercise}
	
	\begin{solution}
		2-38 so
	\end{solution}
	
	\begin{exercise}
	111
	\end{exercise}
	
	\begin{solution}
		2-39 so
	\end{solution}
	
	\begin{exercise}
	111
	\end{exercise}
	
	\begin{solution}
		2-40 so
	\end{solution}
	
	\subsection{Unrestricted Determinants}
	
	\begin{exercise}
	Consider the determinant $| K \rangle = |\psi_1^\alpha \bar{\psi}^\beta_1 \rangle$ formed from {\it nonorthogonal} spatial orbitals, $\langle \psi^\alpha_1 | \psi^\beta_1 \rangle = S^{\alpha\beta}_{11}$. 
	\begin{enumerate}
	
	\item[a.] Show that $| K \rangle$ is an eigenfunction of $\mathscr{S}^2$ only if $\psi^\alpha_1 = \psi^\beta_1$.
	
	\item[b.] Show that $\langle K | \mathscr{S}^2 | K \rangle$ = 1 - $|S^{\alpha\beta}_{11}|^2$ in agreement with Eq.(2.271).	
	
	\end{enumerate}
	\end{exercise}
	
	\begin{solution}
		2-41 so
	\end{solution}
	


\end{document}